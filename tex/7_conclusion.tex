%!TEX root = ../document.tex
\chapter{Concluding remarks}
Throughout this thesis we have presented our plant monitoring system Monoplant and how high-school students interacted with the system while trying to answer questions related to photosynthesis and germination. 

In order to answer our research questions, we gathered data in a biology class at a high-scool in Oslo. Our primary data consisted of video of four students during a class session. This was transcribed and analyzed using interaction analysis techniques. In addition we used notes from earlier observations of the class, and written documents produced by some of the students. 

The thesis is framed within a sociocultural perspective, leading to a focus on how the students interacted with each other and Monoplant, and how the context of the school affected the students inquiry process. The analytical perspective has been \emph{dialogic}, meaning that we have looked at how Monoplant has provided a context for social interaction. 

The first research question was as follows: \emph{"What characterizes the students’ inquiry in interaction with Monoplant?"} 

The second question we sought to answer was: \emph{"How does Monoplant, by presenting photosynthesis differently from the text book, support the inquiry process?"}. Here we saw evidence of one of the students discoveries leading to conceptual understanding. As Monoplant provided real life experiences related to the phenomenon of photosynthesis, and the textbook provided scientific concepts, they complemented each other, leading to stronger concept comprehension. 

We also found that the students had problems understanding the link between the different representations. Especially the link between the textbook model of photosynthesis and the Monoplant model. So while multiple representations may display different aspects of the phenomenon under study, the students are faced with an extra task. The process of linking should therefore be scaffolded so we can have the benefits of multiple representations while reducing the cost. 

%misconception

Our third research question was: \emph{"How is scaffolding operationalized in the environment?"}. Here we looked at how \citeauthor{wood1976role}'s (\citeyear{wood1976role}) six steps of scaffolding was used by the instructors during the session. We also saw how there were multiple opportunities for employing a scaffold, and that scaffolds should be personally adjusted. 

The main finding in this section was how the inquiry process lead to misconceptions among the students. The problem then becomes to scaffold in a way that the students are lead toward the fruitful discoveries, but at the same time have the freedom to explore and not feel commanded in any way. Computer based scaffolds can be used for this task, as they can be programmed to instruct the students with the procedural aspects of inquiry learning. 

The fourth and last question was: \emph{"How does the institutional setting frame the students' inquiry process?"}.