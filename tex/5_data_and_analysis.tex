%!TEX root = ../document.tex
\chapter{Data \& Analysis}
In this chapter we will present the findings from our case study with a focus on themes relevant to our research questions. Each of the themes contain at least one excerpt with a context description, excerpt from the transcript, and an analysis of the unfolding events. 

The first theme (\ref{cha:hypothesisgeneration}) is named \textit{Hypothesis generation and testing}. Here we follow a hypothesis from generation to falsification to a new improved hypothesis. Then we move on to \textit{Misconception} (\ref{cha:guidedinquiry}) where we show examples of how misconceptions can be addressed successfully or unsuccessfully by the teacher, and how it can lead to hypothesis generation based on false premises. The third theme (\ref{cha:conceptualization}) is dubbed \textit{conceptualization} and presents three excerpts regarding scientific and everyday language. The last theme (\ref{cha:linking}), \textit{linking between representations}, aims to show how the students relate the digital representation to the physical world (biology of plant).  

For the sake of simplicity the first experiment, where the plant was located in the window has been named plant A, and the second experiment where the plant was located in the cabinet, plant B. 


\begin{table}[H]
\begin{center}
	\begin{tabular}{l r r } \toprule
	Who &  Interactions  & Percentage\\ \midrule  
	Linda &	 14  & 3.67\% \\
	Nora&	118 & 30,97\% \\ 
	Siri& 	182 & 47.77\% \\
	Fredrik& 67 & 17.59\% \\ \midrule
	All &	381 & 100\%\\
	\bottomrule
	\end{tabular}
\end{center}
\caption{Verbal interactions by participant}
\end{table}

\section{Hypothesis generation and testing}
\label{cha:hypothesisgeneration}
\subsection{First claim}
\subsubsection*{Context}
\label{firsthypothesis}
We enter the situation at the beginning of class. The students have been divided into groups, and they are approximately two minutes into the task. Preceding this discussion, the students have tried for about one minute to figure out what the task is about, and what the two experiments involved. Siri has read out loud the first question in the assignment: "what did you expect would happen?" (in the experiment), and they have rehearsed some of the theories presented in previous lectures (e.g soil moisture decreasing over time). Prior to the excerpt, the students have appeared a bit insecure about the task. But as we enter the setting they seem more focused and interested, and the discussion has changed from making general observations to generating hypotheses.

\subsubsection*{Excerpt 1}\label{ex:excerpt1}
\begin{table}[H]
	\begin{center}
		\begin{tabular}{r l p{7cm} p{3cm} } \toprule
				Time &  Who &  Speech  & Action \\ \midrule 

			2:04 %time
			&Nora %name
			&\parbox[t]{7cm}{\raggedright hehe.. mm.. hmhm .. når den stod i skapet så.. jeg visste ... %speech 
			}&\parbox[t]{3cm}{\raggedright  %action 
			}
			\\

			2:13 %time
			&Siri %name
			&\parbox[t]{7cm}{\raggedright ... neddi skapet ... %speech 
			}&\parbox[t]{3cm}{\raggedright  %action 
			}
			\\

			2:13 %time
			&Nora %name
			&\parbox[t]{7cm}{\raggedright eller jeg visste ikke helt hva den skull.. hva som skulle skje da egentlig .. %speech 
			}&\parbox[t]{3cm}{\raggedright  %action 
			}
			\\
			2:16 %time
			&Siri %name
			&\parbox[t]{7cm}{\raggedright .. det var det planten stod i skapet også skulle det være bare grønt lys på den ... men det kan jo hende for eksempel at det kom litt annet lys inn i skapet også .. så da er det ikke sikkert at det bare var grønt lys ..  %speech 
			}&\parbox[t]{3cm}{\raggedright peker på skapet %action 
			}
			\\

			2:31 %time
			&Nora %name
			&\parbox[t]{7cm}{\raggedright  %speech 
			}&\parbox[t]{3cm}{\raggedright nikker %action 
			} 
			\\

			2:31 %time
			&Siri %name
			&\parbox[t]{7cm}{\raggedright og planten tar jo opp littegrann grønt lys også, men ikke så mye .. så derfor kunne det hende atte den ikke vokste like my.. eller jeg trodde at den ikke ville vokse like mye i skapet .. siden da fikk den bare grønt lys ...  %speech 
			}&\parbox[t]{3cm}{\raggedright  %action 
			}
			\\
			2:46 %time
			&Nora %name
			&\parbox[t]{7cm}{\raggedright ... mmm ... %speech 
			}&\parbox[t]{3cm}{\raggedright  nikker%action 
			}
			\\
		\end{tabular}
	\end{center}
	\caption{First hypothesis}
	\label{excerpt:hypothesisgeneration}
\end{table}

\subsubsection*{Analysis}
\begin{figure}
\centering
\includegraphics[width=\textwidth]{img/dataandanalasys/absorbtion.png}
\caption{Absorption of wavelengths by pigments \citep{bios}}
\label{fig:absorption}
\end{figure}
At first, Nora is not sure what would happen to the plant given green light in the cabinet (plant B). Siri, as one who thinks out loud, promptly starts reflecting on what could have happened. First she proposes that the plant was given more than green light, indicating that there could be error sources to the experiment. This is acknowledged by a slight nod from Nora. Then she goes on to reflect on the wavelengths plants absorb, agreeing that they only absorb a small amount of green light. They conclude that the plant in the cabinet would not grow as much as plant A. Nora agrees with this hypothesis by nodding and saying "mmm". 

The basis for the statement that plants only absorb a small amount of green light can be found in the textbook: "reflected and transmitted light can hit our eyes and give the object color" \citep[pg. 103]{bios}. The book also contains a graph of the different pigments according to the wavelengths of light they absorb (see fig. \ref{fig:absorption}), clearly showing that chlorophyll absorbs little of green light. In addition, the teacher has used this as a discussion point in earlier lectures, asking why plants' leaves appear green. 

\subsection{Claim refuted}

\subsubsection*{Context}
We enter the setting immediately after the excerpt explained in the previous section. Siri has generated a hypothesis, that she wants to test out. The mood in the group has now gone from laughter and insecurity about the task to concentration and goal-driven work. The overall noise level in the class room has also fallen significantly. 

\subsubsection*{Excerpt 2}\label{ex:excerpt2}
\begin{table}[H]
	\begin{center}
		\begin{tabular}{r l p{7cm} p{3cm} } \toprule
			Time &  Who &  Speech  & Action \\ \midrule 

			2:47 %time
			&Siri %name
			&\parbox[t]{7cm}{\raggedright ...eller nesten bare grønt lys ihvertfall ... men hvor mye vokste den egentlig? er det den ((refererer til planten på bordet)) som stod i skapet? %speech 
			}&\parbox[t]{3cm}{\raggedright peker på planten som står på pulten %action 
			}\\

			2:52 %time
			&Sjur %name
			&\parbox[t]{7cm}{\raggedright ja %speech 
			}&\parbox[t]{3cm}{\raggedright  %action 
			}\\

			2:53 %time
			&Nora %name
			&\parbox[t]{7cm}{\raggedright OJ(!) %speech 
			}&\parbox[t]{3cm}{\raggedright  %action 
			}\\

			2:53 %time
			&Siri %name
			&\parbox[t]{7cm}{\raggedright Den har jo vokst ganske mye %speech 
			}&\parbox[t]{3cm}{\raggedright smiler %action 
			}\\
			2:59 %time
			&Siri %name
			&\parbox[t]{7cm}{\raggedright men var stilkene på den som stod i vinduet var de også hvite? %speech 
			}&\parbox[t]{3cm}{\raggedright Peker mot vinduet %action 
			}\\
		\end{tabular}
	\end{center}
	\caption{Claim refuted}
	\label{excerpt:testinghypothesis}
\end{table}
\subsubsection*{Analysis}
After Siri proposed that plant B would not grow as much as plant A, she wants to find out if it holds. Suddenly she notices the plant, which is placed on the table in front of them, and exclaims, "is it that one(!)?". When Sjur confirms, the whole group and especially Siri look surprised. It seems like they all firmly believed that the hypothesis Siri presented earlier (see section \ref{firsthypothesis}) should hold true. Their knowledge of photosynthesis would also point to the plant not growing as much as it had. Thus, the first hypothesis generated by the group has now been falsified. 

As a reaction to this Siri stops to think for a few seconds before she points at the window and asks: "were the stems on the one in the window also white?". This is a very appropriate scientific question, as a plant with absolutely no photosynthesis would most likely be white, as a result of having no pigments. The reason for her asking this may be related to a comment made by another student in a previous lecture. He had observed that when they put plants in the basement for winter storage, the leaves would turn white. 

\subsection{A new claim}
\subsubsection*{Context}
This next excerpt is from a situation occurring only a few seconds later. The group has been instructed to interact with the system on the computer in front of them to find the answer to the question asked at 2:59 \emph{"were the stems on the one in the window also white?"}. When we enter the situation they have a video of plant A on the screen in front of them, dated 31st of October, ready to play. 

\subsubsection*{Excerpt 3}\label{ex:excerpt3}
\begin{table}[H]
		\begin{center}
			\begin{tabular}{r l p{7cm} p{3cm} } \toprule
					Time &  Who &  Speech  & Action \\ \midrule 
				3:21 %time
				&Nora %name
				&\parbox[t]{7cm}{\raggedright Ja for karse har jo hvit stilk %speech 
				}&\parbox[t]{3cm}{\raggedright  %action 
				}\\

				3:23 %time
				&Siri %name
				&\parbox[t]{7cm}{\raggedright Ja det de har hvit stilk de også %speech 
				}&\parbox[t]{3cm}{\raggedright  %action 
				}\\

				3:24 %time
				&Fredrik %name
				&\parbox[t]{7cm}{\raggedright mhm ... mmja så da er det jo egentlig ganske ... ja ikke så stor forskjell da på de som stod ...  i skapet ((peker på planten på border)) og de som stod i vinduskarmen hvis man bare ser på ...  utseende %speech 
				}&\parbox[t]{3cm}{\raggedright Dette sies mens Siri starter videoen, hun stopper også videoen før de har sett den halvferdig. %action 
				}\\

				3:37 %time
				&Siri %name
				&\parbox[t]{7cm}{\raggedright ja .. men da ville jeg kanskje tenke at det kan hende at det kom inn annet lys enn det grønne lyset også. siden de har vokst så bra, og at de vokser bedre hvis de får flere.. lys i flere bølgelengder enn bare grønt lys %speech 
				}&\parbox[t]{3cm}{\raggedright Stemmeleiet går opp mot slutten av setningen, og løfter blikket fra arket for å få bekreftelse
				 %action 
				}\\
			\end{tabular}
		\end{center}
	\caption{A new claim}
	\label{excerpt:newhypothesis}
\end{table}
\subsubsection*{Analysis}
Here Nora and Siri find that the stem of plant A is white as well. Fredrik then says that there is not much difference between the two plants if they consider just their looks. Since Siri got that answer to her question about the stems, she has ruled out that photosynthesis is not happening to the plant in the cabinet. Thus she formulates a new hypothesis, which presumes an error source in the experiment: the plant has grown as much as it did because light of other wavelengths than green has entered the cabinet.
This hypothesis would also explain why her first hypothesis, that the plant would not grow as much as the other, failed. It is also worth noting that monoplant does not provide a means of observing the wavelength of light, but we did however provide the students with a spectrometer image of the green light as shown in INSERT FIGREF HERE.

\section{Misconception}
\label{cha:guidedinquiry}


\subsection{Assumptions based on a misconception}

\subsubsection*{Context}
Prior to the following excerpt, the students have been looking at the movements of the two plants, and observed that plant A is moving towards the sun, a phenomenon called \emph{heliotropism}. They are now observing the movement of plant B. Fredrik has just pointed out that it is growing straight up without any skewed movement like plant A. As we enter the setting, all the students are concentrated and watching a video of plant B from the 4th of November.


\subsubsection*{Excerpt 4}\label{ex:excerpt4}

\def\arraystretch{1.5}
\begin{table}[H]
	\begin{adjustwidth}{-4em}{-4em}
		\begin{center}
			\begin{tabular}{r l p{7cm} p{3cm} } \toprule
				Time &  Who &  Speech  & Action\\ \midrule  

				7:46 %time
				&Nora %name
				&\parbox[t]{7cm}{\raggedright Jeg føler at de vokser veldig mye inni ... skapet eller er det? ... %speech 
				}&\parbox[t]{3cm}{\raggedright  %action 
				}\\

				7:51 %time
				&Siri %name
				&\parbox[t]{7cm}{\raggedright Ja det virka som om de vokste ... %speech 
				}&\parbox[t]{3cm}{\raggedright  %action 
				}\\

				7:53 %time
				&Nora %name
				&\parbox[t]{7cm}{\raggedright ... ser ut som de ble lenger lissom ... %speech 
				}&\parbox[t]{3cm}{\raggedright  %action 
				}\\

				7:53 %time
				&Siri %name
				&\parbox[t]{7cm}{\raggedright ... enda mer der. %speech 
				}&\parbox[t]{3cm}{\raggedright  %action 
				}\\

				7:54 %time
				&Fredrik %name
				&\parbox[t]{7cm}{\raggedright ja %speech 
				}&\parbox[t]{3cm}{\raggedright  %action 
				}\\

				7:56 %time
				&Siri %name
				&\parbox[t]{7cm}{\raggedright ... enn ute, at de ble mye lengre. %speech 
				}&\parbox[t]{3cm}{\raggedright  %action 
				}\\

				7:59 %time
				&Fredrik %name
				&\parbox[t]{7cm}{\raggedright mhm. %speech 
				}&\parbox[t]{3cm}{\raggedright  %action 
				}\\

				8:01 %time
				&Siri %name
				&\parbox[t]{7cm}{\raggedright Kanskje de fokuserer veldig på å vokse oppover når lyset er rett over dem.. at de vokser rett oppover ((fører hånden oppover)) i stedet for å følge lyset og gå lissom sånn sakte oppover ((snurrer hånden sakte oppover)) %speech 
				}&\parbox[t]{3cm}{\raggedright  %action 
				}\\
				
				
				\bottomrule
			\end{tabular}
		\end{center}
	\end{adjustwidth}
	\caption{Assumption based on a misconception}
	\label{excerpt:disconfirmation}
\end{table}

\subsubsection*{Analysis}
When Nora says that the plant is growing taller in the cabinet, she is very cautious when introducing the idea, as it seems like an unlikely observation according to their hypothesis. Siri approves and states that it is indeed growing more than plant A. Fredrik agrees and they all seem a bit puzzled by this observation.

Siri starts to formulate a new hypothesis for why plant B grows more than plant A. Her reasoning is that heliotropism makes plant A grow slower because it has to move after the sun, and since plant B can grow straight up without following the sun, it grows faster. 

There is no indication that this hypothesis relates to anything she has read in the textbook or learned in class, so it seems like her hypothesis is based on what she has observed: plant A grows slowly and follows the sun, whereas plant B grows faster and more upright. Since the students can't explain the phenomena with their current knowledge of photosynthesis, Siri proposes a hypothesis based on empirical data. However, as we will show in the next excerpt, the students have also created a misconception, which is that seeds need photosynthesis to grow.


\subsection{Scaffolding to repair misconception}

\subsubsection*{Context}
When we enter the situation, the teacher has been talking with the group for a couple of minutes. They have discussed that plant B grew taller than plant A. The teacher wants to know how they explain this, because they all thought the outcome would be the opposite (see section \ref{firsthypothesis} on page \pageref{firsthypothesis}). Siri has explained her favorite hypotheses, that plant B might have received more than just green light, because if it only got green light it would probably not grow as much.  It is at this point we are entering the setting.
 
\subsubsection*{Excerpt 5}\label{ex:excerpt5}

\def\arraystretch{1.5}
\begin{table}[H]
	\begin{adjustwidth}{-4em}{-4em}
		\begin{center}
		\begin{tabular}{r l p{7cm} p{3cm} } \toprule
			Time &  Who &  Speech  & Action\\ \midrule  

			13:44 %time
			&Lærer %name
			&\parbox[t]{7cm}{\raggedright ja.. så altså dere tenker at .. sammenhengen mellom \underline{vekst} og fotosyntese den er helt klar ... du kan ikke du tenker at du kan ik et \textbf{frø} kan ikke \textbf{spire} og vokse og bli en plante uten at drives fotosyntese.. tenker dere alle det? %speech 
			}&\parbox[t]{3cm}{\raggedright  %action 
			}\\

			14:00 %time
			&Fredrik %name
			&\parbox[t]{7cm}{\raggedright Det er jo noen planter som ikke har fotosyntese ... og de spirer jo og fordet ikkesant.. det er vel en liten energipakke på en måte i  frøet da? er det ikke det da? %speech 
			}&\parbox[t]{3cm}{\raggedright  %action 
			}\\

			14:14 %time
			&Lærer %name
			&\parbox[t]{7cm}{\raggedright okei, er det? %speech 
			}&\parbox[t]{3cm}{\raggedright  %action 
			}\\

			14:14 %time
			&Nora %name
			&\parbox[t]{7cm}{\raggedright Ja %speech 
			}&\parbox[t]{3cm}{\raggedright nikker annerkjennende %action 
			}\\
			
			\bottomrule
		\end{tabular}
		\end{center}
	\end{adjustwidth}
	\caption{Teacher scaffolding to repair misconception}
	\label{excerpt:teachertalk}
\end{table}

\subsubsection*{Analysis}
The excerpt starts with the teacher formulating a question in which he says: "a \emph{seed} can't \emph{germinate} and grow to become a plant without photosynthesis.. do you all think that?". In this sentence the teacher says what Siri indicated in a way which leads the group to think outside the textbook model of photosynthesis. By using the words "\emph{seed}"" and "\emph{germination}"" (bold text in excerpt) the teacher hints to the germination process. 

When the teacher have asked if this is what they all think, Fredrik starts answering right away. He introduces the notion that there are plants that do not have photosynthesis, but can nevertheless grow from a seed. Hence that the seed have an energy pack. This notion lays the basis for a discussion to which the teacher leads the students to find out that seeds have starch as a food reserve, which makes it possible for them to grow (germinate). 

Up till this point in the session, the students have tried to generate and test hypotheses with what they know about photosynthesis, or what they have observed in Monoplant. Despite of this, they fail to generate valid hypotheses for why plant B has grown more than plant A. They are hampered because they think that seeds need photosynthesis to grow. This misconception is repaired due to teacher intervention, and at this point the students know that a seed can grow without photosynthesis and therefore without light.

\begin{figure}
\centering
\includegraphics[width=0.4\textwidth]{img/data_analysis/light_dependent_detail.png}
\caption{Detail from the illustration of the light-dependent reaction \citep{bios}}
\label{fig:lightdependentdetail}
\end{figure}

\subsection{Misconception not followed up}

\subsubsection*{Context}
The teacher is standing in front of the group asking them questions to make them reflect on different aspects of the photosynthesis. The conversation follows a pattern where the teacher asks a question, and the students answer. As we enter the setting, Siri has just presented a hypothesis. As the teacher asks for other explanations, all of the students are looking down on the textbook illustration of the light-dependent reaction placed on the table in front of them (see figure \ref{fig:lightdependentdetail}). 

\subsubsection*{Excerpt 6}\label{ex:excerpt6}
\begin{table}[H]
	\begin{center}
		\begin{tabular}{r l p{7cm} p{3cm} } \toprule
			Time &  Who &  Speech  & Action \\ \midrule 
			12:34 %time
			&Lærer %name
			&\parbox[t]{7cm}{\raggedright ja det er et alternativ en alterna har dere noen andre eventuelle forklaringer? det kunne være andre forklaringer? %speech 
			}&\parbox[t]{3cm}{\raggedright  %action 
			}\\

			12:42 %time
			&Nora %name
			&\parbox[t]{7cm}{\raggedright kan jeg bar sp.. solener.. ehh kan det bare være lys også? %speech 
			}&\parbox[t]{3cm}{\raggedright Peker på ordet "solenergi" på modellen på arket %action 
			}\\

			12:45 %time
			&Lærer %name
			&\parbox[t]{7cm}{\raggedright Hva sier du %speech 
			}&\parbox[t]{3cm}{\raggedright bøyer seg frem for å høre bedre %action 
			}\\

			12:46 %time
			&Nora %name
			&\parbox[t]{7cm}{\raggedright Kan lys forårsake eksit.... at det eksiterer? eller bare sol? %speech 
			}&\parbox[t]{3cm}{\raggedright Tar fingeren langs pilen i modellen hvor det står "solenergi", og illustrerer at solenergi kommer inn til klorofyllmolekylene %action 
			}\\

			12:50 %time
			&Lærer %name
			&\parbox[t]{7cm}{\raggedright vanlig lys.. åja du mener lampe altså sånn grønt lys? %speech 
			}&\parbox[t]{3cm}{\raggedright  %action 
			}\\

			12:54 %time
			&Nora %name
			&\parbox[t]{7cm}{\raggedright mhm %speech 
			}&\parbox[t]{3cm}{\raggedright  %action 
			}\\

			12:55 %time
			&Lærer %name
			&\parbox[t]{7cm}{\raggedright Altså det er jo spørsmålet...  %speech 
			}&\parbox[t]{3cm}{\raggedright  %action 
			}\\

			12:57 %time
			&Nora %name
			&\parbox[t]{7cm}{\raggedright eller jeg mente ehh.. lys  %speech 
			}&\parbox[t]{3cm}{\raggedright peker opp mot lampene i taket %action 
			}\\
			12:57 %time
			&Siri %name
			&\parbox[t]{7cm}{\raggedright ... det var jo det de gjorde i skapet %speech 
			}&\parbox[t]{3cm}{\raggedright peker mot skapet %action 
			}\\

			12:58 %time
			&Lærer %name
			&\parbox[t]{7cm}{\raggedright Åja her inne? jammen få.. fikk de det inne i skapet? %speech 
			}&\parbox[t]{3cm}{\raggedright  %action 
			}\\

			13:00 %time
			&Nora %name
			&\parbox[t]{7cm}{\raggedright Nei jeg bare lurer jeg mm. %speech 
			}&\parbox[t]{3cm}{\raggedright  %action 
			}\\
		\end{tabular}
	\end{center}
	\caption{Misconception not followed up}
	\label{excerpt:misconceptionnotfollowed}
\end{table}

\subsubsection*{Analysis}
After the teacher has asked if there can be any other explanations, Nora takes the opportunity to ask the question: "...ehh can it be light as well?". As she asks the question, she points at the word "solar energy" in the illustration of the light-dependent reaction (see fig \ref{fig:lightdependentdetail} on page \pageref{fig:lightdependentdetail}). The teacher does not quite understand what she is asking, and therefore leans in and ask her to repeat the question. She reformulates her question in a more scientific language, asking if only sunlight can excite chlorophyll, and not artificial light. As she says the word "excite", she is pointing at the illustration of the chlorophyll molecule, and as she says "sun", she is pointing at the word "solar energy".

When Nora asks these questions, she refers to the illustration in front of her (as indicated by her pointing gesture). The reason for Nora asking is that in the illustration, photons are labeled as "solar energy" . This is probably done by the authors of the textbook to simplify the model as the audience is high school students, but in this case it leads to a big misconception. As we can see from her questions, she is unsure if artificial light can cause photosynthesis (which it can). If this were the case, Nora could rule out photosynthesis as the cause of the cabinet plant growing more than the window plant.

The teacher then proceeds to ask her if she means a lamp with green light, whereupon she confirms by saying "mmm". When the teacher replies that it is the question they are supposed to answer, she quickly replies that she meant artificial light, while pointing to the fluorescent ceiling lighting in the class room. The teacher then misinterprets her question, and think she is speaking of the specific lighting in the classroom, and not artificial light in general.

After Nora's question regarding the "erroneous" representation in the model, and the teacher's failure to understand the motivation behind the question, the discussion quickly takes another turn. The question is left hanging, it is not followed up later in the session.

\sout{Indicate that this will be followed up later?}


\section{Conceptualization}
\label{cha:conceptualization}

\subsection{Everyday language}

\subsubsection*{Context}
When we enter the setting, the teacher has just left the group. Morten has asked the students to look at the videos of the two different experiments and see if there are any differences in their appearance. The students have looked at plant B and found that it is mostly the stem that grows, not the leaves. Fredrik has requested that they should check plant A to compare the two, and Siri has just started the video from 29th of October, showing plant A. \sout{The group seems more motivated than in the next excerpt (\ref{excerpt:teacherintervention}), \textit{teacher intervention}.}


\subsubsection*{Excerpt 7}\label{ex:excerpt7}

\def\arraystretch{1.5}
\begin{table}[H]
	\begin{adjustwidth}{-4em}{-4em}
		\begin{center}
		\begin{tabular}{r l p{7cm} p{3cm} } \toprule
			Time &  Who &  Speech  & Action\\ \midrule  

			17:12 %time
			&Siri %name
			&\parbox[t]{7cm}{\raggedright Der åpner jo bladene seg med en gang nesten %speech 
			}&\parbox[t]{3cm}{\raggedright Nora ser mot planten på bordet %action 
			}\\

			17:15 %time
			&Fredrik %name
			&\parbox[t]{7cm}{\raggedright ja ... ((stillhet, venter til video er ferdig)) det kan jo ha noe med at her trenger den jo bladene for å ((tar hånden over bordet og beveger den raskt oppover som om han tar i mot noe)) \textbf{fange} lyset da, mens ((nikker mot skapet)) den trenger jo ikke det så mye inni skapet.. eh kanskje %speech 
			}&\parbox[t]{3cm}{\raggedright   %action 
			}\\

			17:34 %time
			&Siri %name
			&\parbox[t]{7cm}{\raggedright at den \textbf{bruker} næringen fra jorda og frøet mer i skapet? %speech 
			}&\parbox[t]{3cm}{\raggedright  %action 
			}\\

			17:37 %time
			&Fredrik %name
			&\parbox[t]{7cm}{\raggedright ehhhh.. ja. eller at den ikke utnytter den sol.. det \textbf{sollyset} inne i skapet så det den trenger jo ikke da også at bladene \textbf{spretter ut} så tidlig eller at... eh ja. %speech 
			}&\parbox[t]{3cm}{\raggedright  Gestikulerer med hånden som om den var planten som utnytter sol og vokser blader. %action 
			}\\

			\bottomrule
		\end{tabular}
		\end{center}
	\end{adjustwidth}
	\caption{Everyday language}
	\label{excerpt:everydaylanguage}
\end{table}

\subsubsection*{Analysis}
 First Siri mentions that the leaves are opening almost at once (compared to what they saw in the video from the cabinet). Fredrik approves, waits for the video to stop and then he says that plant A need leaves in order to "capture" light, while plant B does not need any leaves for that purpose. Siri asks if what he means is that plant B uses more food from the soil and the seed. Fredrik answers that plant B does not make use of the sunlight, hence it does not need leaves that "pops out" early. 

 The textbook Analysis of this phenomenon is that photosynthesis happens in the leaves. Photons become absorbed by different pigments that excite electrons, which again triggers the other parts of photosynthesis. Plants therefore need leaves in order to perform photosynthesis. Thus, the students are discussing a complex phenomenon using everyday language. Examples are (bold text in excerpt) \textit{capture} (fange) and \textit{use} (bruker) instead of \textit{absorb}, and \textit{sunlight} (sollyset) instead of \textit{photons}. 

\subsection{Teacher intervention}

\subsubsection*{Context}
The discussions preceding this excerpt has been a bit slow, leading us to intervene more in the situation, and asking more questions. The students still seem interested and concentrated, with Siri in the lead. The language used by the participants has up until this point been informal, and most utterances has been related to observations. A few seconds prior to excerpt 8 Sjur has instructed them to flip the task sheet, revealing an illustration from the textbook of the light-dependent reaction. INSERT FIGREF!!!

\subsubsection*{Excerpt 8}\label{ex:excerpt8}
\begin{table}[H]
	\begin{center}
		\begin{tabular}{r l p{7cm} p{3cm} } \toprule
			Time &  Who &  Speech  & Action \\ \midrule 
			11:20 %time
			&Lærer %name
			&\parbox[t]{7cm}{\raggedright Går det bra eller %speech 
			}&\parbox[t]{3cm}{\raggedright kommer bort til bordet og lener seg på det.%action 
			}\\

			11:23 %time
			&Siri %name
			&\parbox[t]{7cm}{\raggedright mmm, ja %speech 
			}&\parbox[t]{3cm}{\raggedright  alle nikker%action 
			}\\

			11:24 %time
			&Lærer %name
			&\parbox[t]{7cm}{\raggedright skjønner dere ... har dere funnet forklaring på alle spørsmålene? %speech 
			}&\parbox[t]{3cm}{\raggedright  %action 
			}\\

			11:26 %time
			&Alle jentene %name
			&\parbox[t]{7cm}{\raggedright *** vi prøver ... %speech 
			}&\parbox[t]{3cm}{\raggedright snakker i munnen på hverandre %action 
			}\\

			11:27 %time
			&Siri %name
			&\parbox[t]{7cm}{\raggedright Jeg tror kanskje jeg har en ide om det med at den her ute ((peker mot vinduet, refererer til planten i vinduet)) ikke vokser like høyt, eller så fort ihvertfall.. fordi atte når det kommer veldig mye sol så blir jo \textbf{klorofyllmolekylene eksitert}, men når alle ... alle \textbf{klorofyllene} blir \textbf{eksitert} i planten, sånn atte det ikke er flere som kan bli \textbf{eksitert} så hjelper det ikke om det er mere lys. %speech 
			}&\parbox[t]{3cm}{\raggedright  %action 
			}\\
		\end{tabular}
	\end{center}
	\caption{Teacher intervention}
	\label{excerpt:teacherintervention}
\end{table}

\subsubsection*{Analysis}
When the teacher approaches the group, Siri's language quickly change from explaining things in everyday terms to a more precise scientific language. After roughly 11 minutes of discussion, first occurrences of the words like \textit{excited}, \textit{chlorophyll}, and \textit{molecules} (bold text in excerpt) appear. 

One reason for the sudden change in language may be that only seconds before the excerpt, the students looked at the figure from the textbook, representing the light-dependent part of photosynthesis \sout{SEE FIGREF INSERT HERE}. This may have led Siri onto a more theoretical path of explanations, causing her to try and explain the phenomenon using scientific language. 

Another explanation of this phenomenon may be that when the teacher asks a question, the students think he will be assessing the answer. Thereby creating a test-like situation for the students, where Siri is eager to express her knowledge about the photosynthesis model as explained in the textbook. 

\sout{will be followed up as institutional stuff in discussion}

\subsection{Scientific language}
\subsubsection*{Context}
The students work with task 3 regarding soil moisture and differences in absorption rate. Most of the discussions have been concerned with making general observations, and they are struggling to form new hypotheses. The main observation is that there are major differences in the absorption rate in the two experiments. In an effort to push the discussion further, Sjur (researcher) has started to intervene, asking what it could mean in terms of photosynthesis that the soil moisture level drops less in the end of the experiment (see Figure \ref{fig:soilmoistscreenshot}). Approximately one minute before excerpt 9 starts, the teacher has tried to position himself discretely behind the group, but all the students except Linda has noticed him. As we enter the setting, Nora initiates the discussion.

\subsubsection*{Excerpt 9}\label{ex:excerpt9}
\begin{table}[H]
	\begin{center}
		\begin{tabular}{r l p{7cm} p{3cm} } \toprule
			Time &  Who &  Speech  & Action \\ \midrule 
			29:16:00 %time
			&Nora %name
			&\parbox[t]{7cm}{\raggedright men det er sånn...fordi vi har jo...det er jo den \textbf{lysuavhengige} delen av \textbf{fotosyntesen} også...jeg vet ikke om den har...\textbf{atp} og \textbf{nadph} fra f... %speech 
			}&\parbox[t]{3cm}{\raggedright ser mot Sjur mens hun snakker, vender seg mot Fredrik når han avbryter henne %action 
			}\\

			29:26:00 %time
			&Fredrik %name
			&\parbox[t]{7cm}{\raggedright ...den må jo ha den...først drive den lys... eller den må jo drive den \textbf{lysavhengige} også for å drive den \textbf{lysuavhengige} %speech 
			}&\parbox[t]{3cm}{\raggedright bruker hendene til å vise at den lysuavhengige reaksjonen er avhengig av den lysavhengige reaksjonen %action 
			}\\

			29:35:00 %time
			&Siri %name
			&\parbox[t]{7cm}{\raggedright mhm %speech 
			}&\parbox[t]{3cm}{\raggedright  %action 
			}\\

			29:36:00 %time
			&Fredrik %name
			&\parbox[t]{7cm}{\raggedright ...den har vel ikke \underline{atp} eller \underline{nadph} fra før av? %speech 
			}&\parbox[t]{3cm}{\raggedright alle ler %action 
			}\\

			29:44:00 %time
			&Nora %name
			&\parbox[t]{7cm}{\raggedright ja det var det jeg lurte på også %speech 
			}&\parbox[t]{3cm}{\raggedright  %action 
			}\\

			29:46:00 %time
			&Siri %name
			&\parbox[t]{7cm}{\raggedright nei det er vel den \textbf{lysavhengige} reaksjonen bruker til å danne det? %speech 
			}&\parbox[t]{3cm}{\raggedright  %action 
			}\\
		\end{tabular}
	\end{center}
	\caption{Scientific language}
	\label{excerpt:scientificlanguage}
\end{table}

\subsubsection*{Analysis}
After failed attempts to explain the observation of difference in absorption rate in the two experiments, Nora suddenly switches to a more scientific language than in the minutes preceding this discussion. The words emphasized in bold can be found both in the textbook and in the language used by the teacher in earlier presentations of the material.

There may be several reasons for this sudden change in language. \emph{First}: Sjur has asked an intervening question, and while she is answering this, she is looking at him as if he knows the answer, leading to a test-like situation. \emph{Second}: the teacher is standing behind her listening to the whole situation. \emph{Or third}: she is simply trying to bring in another representation as the students have not yet been able to explain the phenomena with the use the physical plant and the system. 

When Fredrik says \emph{".. it does not have \underline{atp} or \underline{nadph} from before?"}, everbody including himself laughs. It does not happen anything else at the moment, so it is apparent that the laugh comes as reaction to his statement. The laugh might happen because they have come to an extremity of their understanding and becomes uncertain. Alternatively it may be that they are laughing because they all know that they should have knowledge about this since it's part of the curriculum.


\section{Linking representations}
\label{cha:linking}
\subsection{Soil moisture representation}


\subsubsection*{Context}
The students have read the introduction to assignment 3: \emph{"Look at the soil moisture graph for the whole period of the experiment. Plant A was sown on 25th of October and plant B was sown 1st of November."} Siri has navigated to the soil moisture graph in the system (see Figure~\ref{fig:soilmoistscreenshot}), and got some help from Sjur to navigate the graph. Siri expanded the graph to include the lifespan of both plants and Sjur explained that this was indeed what they were looking at. Siri lets go of the mouse and keyboard to read assignment 3a: \emph{Is there any difference in the absorption rate?} It is at this point we enter excerpt 10



\subsubsection*{Excerpt 10}\label{ex:excerpt10}

\def\arraystretch{1.5}
\begin{table}[H]
	\begin{adjustwidth}{-4em}{-4em}
		\begin{center}
		\begin{tabular}{r l p{7cm} p{3cm} } \toprule
			Time &  Who &  Speech  & Action\\ \midrule  

			21:34 %time
			&Nora %name
			&\parbox[t]{7cm}{\raggedright Åja, fra de... den og den ((peker på høyre og venstre side av grafen)) %speech 
			}&\parbox[t]{3cm}{\raggedright Lager v-tegn med fingrene og viser hvilken periode i grafen planten var i vinduet, og hvilken periode den var i skapet %action 
			}\\

			21:36 %time
			&Sjur %name
			&\parbox[t]{7cm}{\raggedright ja. %speech 
			}&\parbox[t]{3cm}{\raggedright  %action 
			}\\

			21:37 %time
			&Siri %name
			&\parbox[t]{7cm}{\raggedright Åja, så det der er den ene planten og det der er den andre.. %speech 
			}&\parbox[t]{3cm}{\raggedright Peker først på venstre side av grafen, så på høyre %action 
			}\\

			21:41 %time
			&Nora %name
			&\parbox[t]{7cm}{\raggedright mhm, den der går litt brattere ned på ... %speech 
			}&\parbox[t]{3cm}{\raggedright Peker på området i grafen hvor planten sto i skapet %action 
			}\\

			21:44 %time
			&Fredrik %name
			&\parbox[t]{7cm}{\raggedright Ja, den går mye brattere ned. %speech 
			}&\parbox[t]{3cm}{\raggedright  %action 
			}\\

			21:46 %time
			&Siri %name
			&\parbox[t]{7cm}{\raggedright Kanskje det betyr at den der andre planten bruker mye mer fuktighet fra jorden %speech 
			}&\parbox[t]{3cm}{\raggedright Peker på området i grafen hvor planten sto i skapet %action 
			}\\

			\bottomrule
		\end{tabular}
		\end{center}
	\end{adjustwidth}
	\caption{Linking between representations}
	\label{excerpt:soilmoistureexcerpt}
\end{table}

\begin{figure}
	\centering
	\includegraphics[width=1.0\textwidth]{img/dataandanalasys/soilmoisturegraph.png}
	\caption{Screenshot from the soil moisture graph}
	\label{fig:soilmoistscreenshot}
\end{figure}

\subsubsection*{Analysis}
Here the students are looking at Monoplant's representation of the soil moisture over time (see figure~\ref{fig:soilmoistscreenshot}). At first they try to interpret what part of the graph is which plant. First, Nora shows by pointing with a V-shaped hand which part of the graph that represents plant A and plant B. Siri follows up and explains in an acknowledging way by pointing first to the left and then to the right. When this is confirmed and the students understand how the graph is divided between the two experiments, they start to interpret what the graph tells them. Nora observes and tell the others that the curves from the plant in the cabinet is much steeper than from the one in the window. The other students agree and Siri claims that the plant in the cabinet uses a lot more water. Hence it seems like the students are interpreting the graph to represent the water ($\text{H}_2\text{O}$) usage of the plant. 

This might be for several reasons. For example the textbook shows that plants use $\text{H}_2\text{O}$ in both the light dependent and the light independent reactions, so the students knows that $\text{H}_2\text{O}$ plays a central role in photosynthesis. There are also some constraints for interpretation in the system. First it is designed to represent a plant, hence it would be hard to interpret the soil moisture graph to not represent the life of the plant. Lastly there is the assignment: \emph{"Is there any difference in the \textbf{absorption} rate?"} By using the word absorption, we have constrained the interpretation of the graph, which lead the students to focus on the plants absorption of $\text{H}_2\text{O}$.
