%!TEX root = ../document.tex
\chapter{Discussion}
In this chapter we will discuss our research question by contextualizing our findings to the analytical concepts introduced earlier . As an overall theme we will look at the inquiry process of the students in interaction with Monoplant. This will manifest through 4 sections, the first being about the inquiry process. Further we will discuss how multiple External representations support the inquiry process of the students, how Scaffolding is instatiated in the environment and how the institutional setting frame the students inquiry process.

\section{Inquiry process}
In this section we will broadly address our first research question: What characterizes the students’ inquiry in interaction with monoplant? 
In the previous chapter we presented excerpts from the session where the students interacted with monoplant. We have seen that the students were generating hypotheses of what happened with the plant and why. We showed examples of explanations, discussion, misconceptions In excerpt 1, 3, 4 and 7 we showed that the students were generating hypotheses and in 
The experiments conducted and the session with the students can be categorized as \emph{guided inquiry} \citeauthor*{staver1987analysis} \citetext{\citeyear{staver1987analysis}, referenced in \citealp{prince2006inductive}}. The students are given a problem (the questions) but need to figure out solutions by themselves.  

The four \citep{de1998scientific}

Here we can talk about: 
presumptions, surprises, possible explanations, misconceptions, everyday language (empiric explanations) 


\begin{itemize}
\item{Inquiry learning and the problems. \citep{de1998scientific}}
\item{ZPD \& Scaffolding}
\item{Misconceptions}
\item{Everyday language \& Scientific language}
\item{Excerpts: 1, 2, 3, 4, 6, 7, 10}
\end{itemize}



\section{Multiple External Representations in Inquiry processes}
How does Monoplant, by visualizing/present photosynthesis differently from the text book, support the inquiry process? 
Here we can talk about: 
\begin{itemize}
\item{how do the textbook represent photosynthesis/external factors}
\item{how do Monoplant represent photosynthesis/external factors}
\item{\nameref{ex:excerpt10} (soilmoist)}
\item{\nameref{ex:excerpt6} (misconception from textbook representation)}
\item{Excerpts: 2, 3, 4, 6, 7, 8, 10}
\item{MER}
\item{Social practices}
\item{Everyday language \& Scientific language}
\end{itemize}




\section{Scaffolding}
How is scaffolding instantiated in the environment?
Here we can talk about: 
\begin{itemize}
\item{ZPD}
\item{Scaffolding}
\item{Inquiry learning}
\item{Misconceptions}
\item{Excerpts: (3), 4, 5, 6, 8, 9}
\end{itemize}



\section{Institutional setting}
How does the institutional setting frame the students inquiry process? (“doing school \& doing science”)
Here we can talk about: 
\begin{itemize}
\item{Institutional settings}
\item{Inquiry learning}
\item{Everyday language \& Scientific language}
\item{Excerpts: 1, 5, 8, 9}
\end{itemize}
