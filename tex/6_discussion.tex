%!TEX root = ../document.tex
\chapter{Discussion}
In this chapter we will discuss our research questions by contextualizing our findings to the theoretical concepts introduced earlier. As an overall theme we will look at the inquiry process of the students in interaction with Monoplant. This will be showed through 4 sections, the first being about the inquiry process. Next we will discuss how multiple external representations support the inquiry process of the students, then how scaffolding is instantiated in the environment, and finally how the institutional setting frame the students' inquiry process.


\section{Inquiry process}
In the previous chapter we presented excerpts from the session where the students interacted with Monoplant. We have seen that the students were generating hypotheses of what happened with the plant and why it grew as much as it did. We showed examples of explanations, discussion, misconceptions and surprises. In this section we will discuss some of these examples further and broadly address our first research question: \emph{"What characterizes the students’ inquiry in interaction with Monoplant?"}

\subsection{Tentative hypothesis}
We designed the experiments together with the teacher. The students were given a problem in form of the assignments they discussed. They had to figure out the answers with the help of Monoplant, which presented detailed data logging of the experiments. The experiments conducted combined with the problem solving-session with the students can be categorized as a hybrid of \emph{guided inquiry} and \emph{structured inquiry} \citetext{\citet{staver1987analysis}, referenced in \citealp{prince2006inductive}} as the students are given a problem and the means (Monoplant) to solve it. It is a structured inquiry in that Monoplant provides information that the students can use while solving the tasks. At the same time this information needs to be interpreted and evaluated, hence the students need to figure out how to interpret the information, making the inquiry process look more like guided inquiry.

As showed in \emph{excerpt 1}, Siri presented a hypothesis in which she stated that plant B would not grow much because it would not get as much light as plant A. In \emph{excerpt 2} data was presented to her that showed that plant B had indeed grown much. As she had already made a hypothesis, her interpretation of the data was directed by her preconceptions. Since the data disproved her first hypothesis, the next hypothesis is claiming there might be some sources of error in the experiment. Hence she is misinterpreting the data, which lays the basis for a denial of the fact that her first hypothesis was wrong, or in other words: she starts to explain why the first hypothesis did not hold even though she still thinks it should.

\citet{de1998scientific} addressed four problems that students encounter during inquiry learning. These were classified according to the main discovery learning processes: \textit{hypothesis generation}, \textit{design of experiments}, \textit{interpretation of data}, and \textit{regulation of discovery learning}. In our case we controlled two of these stages by designing and initiating the experiments for the students, as well as letting Monoplant do a systematic logging of data during the experiment, hence regulating the inquiry process. This meant that the students were facing two of the stages: interpreting the data and generating hypotheses based on their interpretation of the data. These two are closely linked and mutually dependent. \citeauthor*{klahr1993heuristics} \citetext{\citeyear{klahr1993heuristics}, referenced in \citealp{de1998scientific}} reported that misinterpretation of data often result in confirmation of the current hypothesis. In the case with Siri in \emph{excerpt 1} and \emph{2}, we can see that she is sticking to her first hypothesis when interpreting new data, but she tries to make the experiment invalid as the data compromise her understanding. \citeauthor*{dunbar1993concept} \citetext{\citeyear{dunbar1993concept}, referenced in \citealp{de1998scientific}} have also found evidence of the tendencies for students to keep initial hypothesis rather than stating a new. He mentions what calls the "unable-to-think-of-an-alternative-hypothesis" phenomenon, as a possible explanation. This means that the students keep their current hypothesis (despite conflicting evidence) simply because they have no alternative.

\subsection{Delayed inquiry}
The students were done with the textbook chapter of photosynthesis and were able to explain  phenomena such as growth theoretically. Their presumptions to the outcomes of the experiment colored their interpretation of data because it was connected to the students' prior conceptual knowledge. Siri knew that plants make food for themselves by doing photosynthesis. To do photosynthesis, a green plant such as the cress in the experiment, needs light of wavelengths other than green (e.g blue and red). This reasoning makes sense to Siri because she knows a lot about the scientific concepts concerning the theme at hand. We can say that the inquiry process became deductive as it was affected by the students preconceptions and their ability to explain the observations they made with Monoplant. 

However, this is a misconception in inquiry learning, and what \citet{gomez2008elementary} refers to as \emph{inconsistent understanding}, according to what the teacher intended. In this case Siri's conception of photosynthesis, which in the context of the textbook examples makes sense, becomes a misconception when she is confronted with a plant that germinates. Hence it leads her to an erroneous conclusion. \citet{smith1994misconceptions} makes the description of this kind of misconception as "faulty extensions of productive prior knowledge", i.e. a conception might help describe a phenomenon in one context, but falsely describe it in another context. \citeauthor*{klahr1993heuristics} put words to what seems to be the main issue: 

\begin{quote}"compared to the binary feedback provided to subjects in the typical psychology experiment, real-world evidence evaluation is not so straightforward" \citetext{\citet[p. 114]{klahr1993heuristics}, referenced in \citealp{de1998scientific}}
\end{quote}

Even though our field of study is different from \citeauthor{klahr1993heuristics}, this distinction helps us to illustrate what we can see in the students inquiry: the context of the plant in the experiment is new for the students, making it difficult for them to apply their prior knowledge to the phenomenon.  

We have now established that the inquiry process is influenced by the fact that the students have certain knowledge (preconceptions) about photosynthesis. Coming into the experiment, this can at one hand lead to misconceptions due to the students having a great freedom to pursue their ideas through the inquiry process. In that case, these misconceptions should be followed up and corrected by a more knowledgeable person. On the other hand, the system or an instructor can guide the students to pursue the most fruitful ideas from the start, staying one step ahead of possible misconceptions. We will discuss this further in the section about scaffolding.



\section{Multiple external representations in inquiry processes}
During the inquiry process the students were presented with different representations of the photosynthesis phenomenon. In this section we will give account for how those representations were used in the inquiry process and how they complemented each other. We will also look at differences in the students' language when engaging in talk with the different representations. To recap, our second research question is as follows: \emph{"How does Monoplant, by presenting photosynthesis differently from the text book, support the inquiry process?"}

\subsection{Spontaneous \& scientific concepts}
%How does textbook present photosynthesis
When reviewing the textbook used in the school class' science education, we found that the scientific concepts are mainly represented in a theoretical manner \citep{bios}. In the first paragraph of the chapter concerning photosynthesis, scientific words such as "pigments", "chloroplasts" and "glucose" appear. Later on, photosynthesis is explained by its chemical formula and the chapter rarely gives examples of how photosynthesis affects the life of plants at the concrete level. Therefore the textbook emphasizes how photosynthesis fits into a larger system of scientific concepts, and is more concerned with conveying the \emph{"big picture"} than the specific and concrete experiences encountered by the students. 

%How does Monoplant represent photosynthesis
On the other hand, Monoplant affords a more inductive or "bottom-up" approach. As a learning resource, Monoplant is a tool for exploring ideas related to photosynthesis. The variables relevant for the plant's photosynthesis are mediated through graphs and videos, but leaving the interpretation of those data to the students. The system is only concerned with one plant in one specific context, not trying to generalize from the specific results to a larger scientific concept. 

When looking at our data with this in mind, a pattern in the students' language emerge. During the inquiry process, students use \emph{everyday language} when engaging with Monoplant. An example comes from \emph{excerpt 10} where Siri says that the plant "use moisture from the earth". 

Another example is from \emph{excerpt 7} where students use concepts as "pop out", "capture" and "use sunlight". All of these concepts have their scientific counterpart in the textbook, but when discussing among themselves, the students choose to talk about the phenomenon in a "non-academic" way. 

However, the students' language seem to change when engaging with representations linked to the textbook. An example of this is from \emph{excerpt 8} where Siri use scientific concepts such as "chlorophyll molecule" and "excited" when looking at a textbook illustration of photosynthesis. 

An explanation of the change in language may be given by applying Vygotski{\u\i}'s (2012) theory of spontaneous and scientific concepts as presented in the theory chapter. When engaging with Monoplant, the students address the results of a concrete experiment obtained in a specific context. The concepts they use are therefore linked to what they observe. When Siri says that the plant "uses sunlight", it is because this is something she has experienced. She knows that the sun transfers energy that plants make use of, and she has perhaps seen plants die as a result of lack of light. This is an example of a spontaneous concept, "a nonconscious and nonsystematic" concept \citep{vygotsky2012thought}. Spontaneous concepts have their strength in explaining what concerns the situation, empirically and practically \citep{vygotsky2012thought}, and  therefore mediate the student's thoughts when discussing the plant on the screen in front of them. 

%The corresponding scientific concept in this case would be to "excite electrons", but this is an abstract concept that is difficult to link any concrete experiences to. As a result she feels more comfortable using the spontaneous concept: to "use sunlight" when explaining her thoughts to the other students. 

Yet we see from \emph{excerpt 8} that the same student also uses the scientific concept "excite electrons" when describing the same phenomenon, but now interacting with the textbook. This is a more abstract concept, but has its strength in its "conscious and deliberate character" \citep{vygotsky2012thought}. An explanation of the change in language may be that the student is not aware of the two concepts referring to the same phenomenon. She masters the scientific concept only in the realm of the textbook and the concept's relation to other scientific concepts. And she masters the spontaneous concept only when referring to the concrete situation from which they have observable results. 

Another more plausible explanation would be that in engaging with both Monoplant and the textbook, Siri has mastered both the scientific and spontaneous concepts of exciting electrons. The spontaneous concept has "in it's slow way upwards cleared the path for a scientific concept" \citep{vygotsky2012thought}. The student is therefore able to speak of "exciting electrons", both when talking about the concrete experiment and when discussing the experiment in more abstract terms. 

\citet{vygotsky2012thought} states that "as long as the curriculum supplies the necessary material, the development of scientific concepts runs ahead of the development of spontaneous concepts". We found this to be true in this setting as well. From \emph{excerpts 8-10} we can see that Siri, Nora and Fredrik are able to use the scientific concepts when discussing photosynthesis. The school has supplied the curriculum necessary for absorbing the scientific concepts in the weeks prior to the experiment, leading to the students "mastering" the scientific concepts. Whereas the students' inquiry process with Monoplant supplied a framework for enriching the scientific concepts with personal experiences. This is what has enabled Siri to conceptually and experimentally master the concept of "exciting electrons".  

On the other hand, we do not find any evidence of the other participants mastering the concept in the same way as Siri. Yet they are able to discuss the phenomenon with her using the scientific and spontaneous concepts, albeit not interchangeably. This would suggest that the other students are not far away from mastering both the scientific and spontaneous concept. The step from unconscious to controlled use of the spontaneous concept is therefore within their zone of proximal development \citep{vygotsky2012thought}. 

We believe our data warrants the assumption that different types of representations spurs complementary processes that can lead to stronger concept comprehension among the students. Inquiry-based environments have their strength in that they allow for personal experiences to accumulate, while more scientific representations (from the curriculum and the textbook) place the phenomenon in a broader scientific context. As scientific concepts and spontaneous concepts mutually enrich and depend on each other \citep{vygotsky2012thought}, it is important to take the development of both of them into account when designing learning environments. 
%As shown, using Monoplant in the inquiry process can provide concrete experiences which helps the concept "come to life" \citep{van1998concept}. 
%Can we write about inquiry learning in general as spontaneous concepts? Can this be elaborated?

\subsection{Moving between multiple representations}
During the inquiry process the students were faced with three representations of the same phenomenon: the textbook, the physical plant, and the Monoplant system. The textbook consists of textual representations, along with pictures, illustrations and graphs (see fig.\ref{fig:photosystem}, fig.\ref{fig:absorption}, and fig.\ref{fig:lightdependentdetail}). The physical plant is a real life representation of photosynthesis in action. And the Monoplant system mediates information through timelapse videos and graphs of data collected over time that would otherwise be unavailable for observation.  

As pointed out by \citet{van2006supporting} there are many benefits of representing the same phenomenon in multiple ways. First, each of the representations can show specific aspects of the domain to be learned. Second, one representation can constrain the interpretation of another representation. And third, learners can build abstractions by translating between the representations, which may lead to a deeper understanding of the domain. 

But while the benefits of using MER in education seem obvious, both \citet{ainsworth1999functions} and \citet{van2006supporting} point to problems students may face while undergoing extra tasks related to MER. To exemplify, let us take a look at the different representations involved in the experiment. First, the students must understand the syntax of the representations. E.g. one of the graphs represented in the Monoplant system is relative, meaning that the different units of measurement are discarded and replaced with percentage values. The students then have to understand what the different axes of the graph represent and how the variables relate to one another. Second, they have to understand which parts of the domain are represented. E.g that Monoplant mediates external factors' effect on photosynthesis. And finally, the students have to understand the relation between the different representations. E.g. when playing a video file, it is necessary to see it in relation with the graph to get both the quantitative and qualitative aspects of the phenomenon. 

In our data, we find evidence indicating that the students are able to use some of the different representations interchangeably. From \emph{excerpt 3} and \emph{excerpt 7} we see that the students are able to talk about the videos in the Monoplant system while pointing at and making references to the physical plant. They are also able to understand the syntax of the soil moisture graph and link it to the two experiments they conducted. This can be seen from \emph{excerpt 10} where Siri says: \emph{so that's the first plant and this is the second} while pointing at the graph. We can therefore assume that the students master the extra tasks related to linking between the video, graph and physical plant representations. 

On the other hand, we do not find any evidence of the students linking the representations contained in the textbook with Monoplant or the physical plant when discussing the assignments. At one point in the inquiry process an illustration of the light-dependent reaction (see fig. \ref{fig:lightdependentdetail}) was placed in front of the students and they were invited to bring in the representation to shed light on a theoretical problem they were discussing. But still we saw no evidence of this representation being used in relation with the others. 

One explanation might be the nature of the assignments given. Most of the questions were concerned with the experiments and could be answered, albeit poorly, without bringing in other representations than Monoplant. While answers to the "why" questions invited to higher abstraction levels and conceptual knowledge construction by linking the representations, the link between the representations were not made clear by the assignment. 

Another explanation is given by applying a concept described by \citet{van2006supporting} as "dynamic linking". Monoplant and the physical plant are related in such a way that actions on the plant are automatically reflected in the Monoplant system. E.g. when the students watered the plants in the experiment, they could instantly see the soil moisture level rise in the Monoplant web-interface. Similarly if lighting conditions changed during the day, it was reflected in the video compiled of that day. The relation between Monoplant and the physical plant is therefore made explicit by the nature of the Monoplant system, assisting the students by digitally scaffolding the task of understanding the relations between the representations. 

In contrast, the representations within the textbook are not dynamically linked in any way. This means that the illustrations and graphs work well for complementing the textual information, but leaving students with a greater cognitive load in order to make out the relation between the scientific representation in the textbook, Monoplant and the physical plant. 

A third explanation comes from how the different representations are grouped. The Monoplant system contains both video, graphs, images and live data. But since they are physically integrated within one system, it appears as one representation \citep{van2006supporting}. Similarly the link between the representations in the textbook are made explicit by their placement in relation to one another. The students then face problems when they are asked to relate two groups of representations where the link is not made explicit. 

While the extra tasks that comes with multiple representations may lead to deeper understanding of the domain, it also places a heavy cognitive load which “may leave less resources for actual learning” \citetext{Sweller, 1988, 1989, referenced in \citealp{van2006supporting}}. The task of linking can therefore be simplified, either by grouping or integrating representations, or dynamic linking. 

\subsection{Representation becomes Misconception}
As mentioned earlier, explanations can be accurate enough for one situation but lead to false conclusions in other situations \citep{smith1994misconceptions}. This becomes evident if we look at \emph{excerpt 6}. After looking at the textbook representation that uses the word "solar energy" to label photons, Nora asks \emph{"Can light cause excit.. that it excites. Or is it just the sun?"}. The textbook mostly frames examples of photosynthesis to the nature, where sunlight and solar energy is indeed valid simplifications of photons. But in the case of the experiments with Monoplant, this simplification is challenged but not addressed. Monoplant shows how much light the plant got, but does not distinguish between different types of light. The experiments were however designed in such a way that it differentiated light quality (wavelength of light), as plant A where given sunlight and fluorescent light from the ceiling whereas plant B only got green light. Nora might have interpreted the experiments to address differences to a plant that have access to solar energy and one which gets another type of light energy. In any case this is a good example of how an explanation can be plausible and have explanatory power in one setting, but in trying to link a simplified representation to another  setting, can lead to erroneous conclusions. This shows a good example of the need for scaffolding in an inquiry process, which will be discussed in the next section.

\section{Scaffolding}
During the inquiry process there were several occasions where the students would need extra guidance in order to keep on the path toward the goal. In this section we will discuss these occasions and delve into the question: \emph{"How is scaffolding operationalized in the environment?"}

\subsection{Opportunities for scaffold}
The environment provided for the students' inquiry was relatively open as we encouraged them to discuss and explore the questions among themselves. During the process we, as researchers, tried to stay on the sideline and not intervene unless they asked us direct questions. The teacher was also absent most of the time. The students were then left to their own devices for solving the tasks, leading to situations where scaffolding was needed to further the students' development. 

%Failure to scaffold because outside of ZPD?
One example comes from \emph{excerpt 6} where Nora asks the teacher if plants can absorb light in general or only sunlight, referring to plant B receiving artificial light. The teacher then responds \emph{"well, that's the question"}. This leads her to asking more questions without getting a satisfactory answer. 

When the teacher abstains from answering her question, although he knows the answer, it might be because he believes the answer to be within Nora's zone of proximal development. By not giving her the answer straight away, he tries to push her toward thinking if photosynthesis did happen in the experiment with plant B. But as we can see from the rest of the excerpt, Nora is left wondering. The answer to the question is outside of Nora's ZPD and the teacher's scaffold fails to reduce it to elements that are within Nora's range of competence \citep{wood1976role}. 

Siri however, seems to understand what the teacher is aiming at as she has an affirmative body language and tries to push the discussion forward. This proves that the ZPD is personal \citep{vygotskiui1978mind}, which also implies that the scaffold should be personally adjusted. The scaffold provided by the teacher is then sufficient for Siri, but not for Nora, which perhaps would need some extra rounds of scaffolding to reach Siri's level of development. 

%Success at scaffolding 
In \emph{excerpt 5} we find another example of a scaffold. The students are struggling to figure out why plant B grew more than plant A. Their knowledge about photosynthesis and light quality suggests that plant B should have no photosynthesis, but yet it has grown more than plant A. At this point, the teacher jumps in and asks if a \emph{"seed can not grow without photosynthesis.. do you all think that?"} This leads to a discussion where all the students agree that a seed can grow without photosynthesis, one argument being that we eat seeds and thus they must have energy which can be used for sprouting. This lays the basis for a new idea, that plant B has grown as much as it did without performing any photosynthesis at all, enabling the students to come closer to a possible solution. 

The rhetorical question asked by the teacher proved to be a good operationalization of scaffolding as all the students were able to reach the answer. By simply pointing to certain features of the experiment, the students are able to negotiate a new, and more plausible hypothesis. This implies that the solution was within the students' ZPD. By asking a question that was ahead of their development, but not too far ahead, the students reached a new level of actual development  \citep{vygotskiui1978mind}.

\subsection{Misconceptions}
The previous example from \emph{excerpt 5} can also be viewed as a strategy to fix the students' misconception about photosynthesis. As they have been left to their own devices for exploring the questions, they have had opportunities to generate preconceptions that do not coincide with the scientific concepts. If we look at \emph{excerpt 5} from this perspective, it is the open inquiry process and the preceding discussion from \emph{excerpt 4} that has lead them to believe that seeds need photosynthesis to grow. By intervening at a critical moment, the teacher is then able to steer the students toward more fruitful discoveries. 

On the other hand, the students have in \emph{excerpt 4} and \emph{5} used a lot of time walking down a path that did not lead to any fruitful discoveries. Another strategy that could have been employed would be to scaffold in such a way that misconceptions were not allowed to take root in the first place. I.e. steering the students toward the "right" discoveries \citep{kluge2010simulation}. The instructor would then know which path the students should take, and stay ahead of possible misconceptions. This could draw the students away from meaningless dead-ends, and create more opportunities for constructing appropriate understanding of the problem at hand \citep{kluge2010simulation}.

However, this might be to miss the point of the inquiry process. By stating which discoveries the students are allowed to make, the process becomes closed and more related to acquisition of knowledge than knowledge creation from discoveries. As stated by \citeauthor{de1998scientific}: \begin{quote}"this process should not be like walking down an existing path, rather, it should be an investigation of the environment in an attempt to discover and build knowledge from these discoveries" \citetext{\citet{de1998scientific}, referenced in \citealp{kluge2010simulation}}
\end{quote}

As shown, the inquiry process requires careful and complex orchestration of activities. The students need the freedom to explore, but at the same time be steered by an "invisible hand" toward the interesting discoveries. This is by no means an easy task, as the unpredictability of the situation requires improvisation and on-the-fly adjustments of scaffolds by the instructors.  

\subsection{Computer mediated scaffold}
At different points in the inquiry process a range of the different features of scaffolding, described by \citet{wood1976role}, were used by the teacher and us as more knowledgeable others \citep{vygotskiui1978mind}. The situation emerging in \emph{excerpt 5} is a good example of \emph{reduction in degrees of freedom}. The students' discussion is advancing slowly, so the teacher tries to break down the question into an easier one. The task is then narrowed down to a goal that is within reach, or within the students' ZPD. In \emph{excerpt 8} and \emph{9} we see evidence of \emph{direction maintenance} where the teacher and Sjur respectively intervene at slow points in the discussion in an effort to keep the students on track and to motivate them. In \emph{excerpt 9} Sjur is \emph{marking critical features} by trying to make the students reflect more on the question. And in \emph{excerpt 8} the teacher is doing \emph{frustration control} by asking if everything is OK and if they need any help. 

Similarly, Monoplant as a computer based system also mediated a digital scaffold. Especially the feature described by \citet{wood1976role} as \emph{recruitment}, to get the learners attention. By representing photosynthesis in the form of time lapse videos and interactive graphs we believe that the system created an interesting context for exploring the phenomenon. This can be seen from the excerpts where the students maintained their interest throughout the session.

Other computer mediated scaffolds tries to take over more of the steps involved than Monoplant. Some examples are the critiquing system described in \citet{fischer1991critics} where a computer presents a "reasoned opinion about a product or action generated by a human" and the HabiPro advising system described in \citet{soller2005mirroring}, which detects off-topic talk and intervenes to bring the students back on track. The result is that more resources are freed from the teacher.

On the other hand, we believe that it is difficult to tailor such systems to scaffold appropriately regarding the content in an open inquiry process.

 If Monoplant were to limit itself to a predefined path of discovery, the effort and success at making the discoveries would not have been the students' own. So while an open computer mediated scaffold demands more resources of human instructors, we believe that the task of tailoring the scaffold to the specific context is key to an successful inquiry process. 

%scaffold needs to be tailored to the specific context and learner. Especially in inquiry process with 

%Difficult to automate scaffold in inquiry process as it is open and unpredictable. Easier to scaffold when you have more predictable results. Although there may be specific feature of the students inquiry process that may repeat itself (i.e misconception of seed germination and photosynthesis). 

%In other words the students need to be steered towards the interesting discoveries, but at the same time have the freedom to explore and not be commanded in any way.


%We believe this proves that scaffolding is situated and therefore must be tailored to the specific context in which learning takes place.    




\section{Institutional setting}
Monoplant was tested in a biology class at the highest level offered at Norwegian high schools. It was tested during school hours in the classroom for biology. The teacher was present, walking around helping all the groups as much as possible when they were trying to solve the task. In this section we will address our final research question: \emph{"How does the institutional setting frame the students inquiry process?"} (“doing school \& doing science”).
By this we mean to focus on how the institutional setting affected the students' interaction with Monoplant and their inquiry process. The first thing to note is that since we were filming and recording audio of the interaction of one the groups, they were able to keep an oral discussion without actually writing any answers down on paper. We hoped that this would help us to avoid a test-like situation where the students became interested in finding a correct answer, but rather stimulate discussion and let them negotiate and explore possible answers.

\subsection{Embedded practices in design}
As mentioned in the first section of this chapter, we are controlling some parts of the inquiry process as we have designed the experiments and the assignments. 

\sout{Må kanskje ta med dette i teorien (har vel egentlig ikke skrevet om content- og process oriented)}
The assignments are designed to be content-oriented by asking the students discuss what happened to the plants. But the questions also have the quality of being process-oriented as they are giving the students hints of what to observe and what to discuss. In assignment 2, the students are asked to first watch the video from 29th of October, and then the video from 4th of November. This instruction is given to make sure the students have seen the two plants' movement, as those dates provide the best footage of each plant moving at roughly the same time of their respective life cycles. This instruction is followed up by asking two questions: \emph{"Do you see any difference in how the plant move?"} and \emph{"If yes, why?"}. Thus, the assignment presents the students with means to observe a phenomenon, an opening question to make them reflect on and evaluate their observation, and a follow-up question to make them generate explanations for their observation. In other words they are guided through the process of scientific inquiry.

\subsection{Doing science}
In \emph{excerpt} 7, Fredrik tries to explain why plant A grows and opens its leaves earlier than plant B, which remains as buds for a long time after sprouting. He explores the idea that since plant A has access to sunlight, it needs leaves as opposed to plant B, which have no sunlight. Siri asks a control question to check if she understands what he means. He then makes his explanation more concrete. It become apparent that they negotiate their way through generating an explanation for the observed data. Their language is characterized as \emph{spontaneous} with the use of words such as "pop out" and "capture". 

Once the students had found out that plant B grew more than expected in \emph{excerpt 2}, Siri asked if plant A also had white stems. As mentioned earlier this was probably to check if cress has white stems in general. If this was not the case, the white stems could be used as evidence to prove that plant B did no photosynthesis, as a plant with absolutely no photosynthesis would most likely be white. 

Both \emph{excerpt 7} and \emph{excerpt 2} contain examples of the students doing scientific activities such as constructing explanations and evaluating evidence. This clearly falls into what \citet{jimenez2000doing} calls "doing science".

\subsection{Doing school}
On several occasions during our observations of the class in the weeks prior to the session, students lost interest of a theme or a detail once it was established that it was not a part of the curriculum. For example during the teacher's review of the Calvin cycle, one of the students asked what happened to the $\text{H}_2\text{O}$, to which the teacher replied that to understand that one would have to know the chemical formulas, which was not part of the curriculum. This made the student reply that it was not important for her. During observations as these, it became evident that the students only were interested in learning the curriculum, not less and certainly not more. The students also showed interest in memorizing correct answers to specific questions they would get at an upcoming test and the final exam. 

We have showed some examples test-like situations in our data. Both \emph{excerpt 8} and \emph{9} are examples of situations where questions are asked to the group, and both times the language of the students changes dramatically. 

In \emph{excerpt 9}, Sjur has asked a question and created a test-like situation. The students have also noticed that the teacher is observing them, amplifying the test-like environment. At this point Nora brings in the scientific concepts \emph{light-independent reaction}, \emph{NADPH} and \emph{ATP}. These are words that have not been used earlier in the session, and are closely linked to the curriculum. 

An explanation to why the students introduce these scientific concepts when they do, can be seen if we look at expectations in the educational setting. The students have used the assignments as a guide to interact with Monoplant and discuss what they see, which has resulted in a concrete language referring to the observed data. When Sjur asked what it could mean in terms of photosynthesis that the soil moisture level drops less in the end of the experiment, he tried to guide the students to reflect on a more abstract and scientific level. The students however, are interested in displaying their knowledge in front of their teacher, as he will be the one giving them their final grade in biology based on class participation and test results. They are used to the classroom setting where the teacher asks questions to check if they know what they are supposed to know. In other words \emph{"they are attuned towards what they think is expected of them, and adjust their responses accordingly"} \citep{furberg2009socio}. 

This shows how the educational and institutional setting frames the inquiry process and affects the student talk to become more closely linked to the curriculum. In other words it instantiates rituals, routines and expectations in the educational setting, placing Nora's activity in the category \citet{jimenez2000doing} calls "doing school". 


\subsection{Doing science and school}
By engaging in scientific inquiry while interacting with Monoplant, the students tend to use an everyday language to explain what is happening. As they are discussing what they see, answering the questions and presenting plausible explanations they are doing so without using any scientific language. The only times we see the students' language change toward more scientific concepts is when the interaction is affected by the educational practices. We will therefore argue that a scientific inquiry can be taken up an abstraction level by exposing the students of educational practices. This can be seen in \emph{excerpt 8} where Siri presents her hypothesis for the teacher after engaging with both Monoplant and the textbook representation of photosynthesis, mastering both the scientific and spontaneous concepts of exciting electrons. However, we see no evidence that explicit scientific concepts are applied to the observable data in Monoplant when the students are left to discuss for themselves. 
In order for scientific inquiry to be successful, some institutional practices might be necessary for the students to make the right connections between empirical data from the real world and the scientific concepts from the schools curriculum. When this connection is achieved, the inquiry process with Monoplant leaves the student with personal experiences attached to scientific concepts. 

 


