\chapter{Background}

\section{Photosynthesis}
The variables which we monitor in our application are directly linked to the precondition of all life - photosynthesis. During this process, plants and other organisms transform energy from light to chemical energy in the form e.g glucose and starch. As most organisms are not able to make use of light energy directly, plants are a necessity for producing energy that other organisms can consume. In addition, oxygen is generated as a waste product of the process, enabling us to breathe. The equation for photosynthesis is written as: \ce{(CO2)n + (H2O)n + photons -> (CH2O)n + (O2)n}

Photosynthesis consists of two main parts; the light-dependent reactions, and the light-independent reactions. The light-dependent reaction, as the name implies, only occur in the light. The light-independent reaction occur both in the light and dark, but does not rely on solar energy.  

\subsection{Light-dependent reaction}
The light-dependent reaction consists of two different photosystems (photosystem 1 \& photosystem 2) creating ATP and NADPH molecules for the light-independent reactions. Both systems are located in the thylakoid membrane inside the chloroplast organelles.  In the process, photosystem 2 precedes photosystem 1 due to photosystem 1 being discovered first. 

\subsubsection{Photosystem 2}
In photosystem 2, antenna-complexes consisting of pigments, proteins and enzymes absorb light of different wavelengths and transfer the energy to chlorophyll molecules. The energy leads to electrons transferring to an orbit lying further from the nucleus, making the atom excited. This makes the atom unstable, and a perfect candidate for giving away it’s electrons to electron-acceptors in an electron-transport chain

Since the chlorophyll loses two of its electrons in the process, it gets positively charged and need to find new electrons somehow to be able to absorb photons again. This happens by taking two electrons from a water molecule absorbed by the plant’s roots which then gets split into \ce{2H+} and \ce{1/2O2}. The oxygen dissolves in the air, while the hydrogen protons are “trapped” on the inside of the thylakoid membrane (lumen). This makes the lumen positively charged relative to the stroma, which enables generation of ATP-molecules from ADP- and P-molecules. 

\subsubsection{Photosystem 1}
Photosystem 1 consists of the same parts as photosystem 2, but instead of splitting water molecules, it receives two electrons from the electron transport chain in photosystem 2. These electrons gets transferred out in the stroma, and are then tied together with an h+-proton and NADP+ to produce NADPH.

\subsection{Light-independent reaction}
This reaction is named after its discoverers, Calvin, Benson and Bassam, and is therefore often referred to as the Calvin-Benson-Bassam-cycle. The reaction works as a “sugar-factory”, collecting carbon dioxie and hydrocarbone in many cycles to make glucose. The process is located in the stroma, and requires the NADPH and ATP generated in the light-dependent reaction. 

The glucose produced can be used to generate other organic compounds such as other carbohydrates (ribose, sucrose, starch, glycogen, cellulose), proteins and lipids, depending on what the plant needs.

\subsection{External factors}
Many external factors affect the photosynthesis in plants either they are living outside, in the living room or in a greenhouse. As photosynthesis is a relatively inefficient process, using only 8-10\% of the energy in sunlight (Long et. al, 2006; Zhu et. al, 2010, referenced in Kirschbaum, 2010), much research has gone into increasing photosynthesis to achieve greater conversion rates (Reynolds et al., 2000;Sinclair et al., 2004; Long et al., 2006; Zhu et al., 2010, referenced in Kirschbaum, 2010). The factors of significance are:
\begin{itemize}
\item \ce{CO2} levels
\item Temperature
\item Light intensity and wavelength
\item Water
\end{itemize}
Each of these factors may be a limiting factor, or stressfactor, not enabling photosynthesis to reach its full potential. 

\subsubsection{\ce{CO2} levels}
\ce{CO2} is used in the calvin-cycle for making glucose. The atmosphere contains approximately 0.038\% \ce{CO2}, while the air in e.g a classroom would most likely contain slightly higher values due to a high concentration of students exhaling \ce{CO2}. In a greenhouse \ce{CO2} levels can get too low, due to a high concentration of plants consuming \ce{CO2} and outputting \ce{O2}. The optimal concentration for most plants is between 0.015\% and 0.05\%. 

\subsubsection{Temperature}
All enzymes have an optimal temperature where they function best \citep{bios}. This temperature may vary from species to species, as plants grow in different climates, altitudes and seasons. If the temperature is too low or too high, the enzymes may become denatured.  
Light intensity and wavelength
The different pigments in the light dependent reaction absorbs light of wavelengths from (mainly) 400nm to 700nm. Chlorophyll b for instance absorbs blue light (450nm). If a plant with a high concentration of chlorophyll b is not given light of this wavelength, the electrons would not be excited and the reaction in PS2 would not start. 

Light intensity also plays a role in this reaction. In low light conditions, there is not enough energy available to excite the chlorophyll molecules, moving electrons needed in PS2. In optimal light conditions all the chlorophyll molecules are exciting electrons, the production is light-saturated. In too strong light conditions, the chloroplasts may be burned from the heat and die.  

\subsubsection{Water}
Water is used in both the light-dependent and light-independent reaction, but is seldom a limiting factor by itself. If water-levels are low and the evaporation-rate is high, most plants will close the leaves to minimize water-loss. This makes the plant unable to absorb \ce{CO2} and photons, leading to photorespiration which again leads to plant reduction. Water shortage is only a problem in itself when the plant’s cells dries out leading to the stem and tissue collapsing. 
