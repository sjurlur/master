%!TEX root = ../document.tex
\chapter{Data \& Analysis}
In this chapter we will present the findings from our case study ...

Something worth noting?:
\begin{table}[H]
\begin{center}
	\begin{tabular}{l r r } \toprule
	Who &  Interactions  & Percentage\\ \midrule  
	Linda &	 14  & 3.67\% \\
	Nora&	118 & 30,97\% \\ 
	Siri& 	182 & 47.77\% \\
	Fredrik& 67 & 17.59\% \\ \midrule
	Everybody &	381 & 100\%\\
	\bottomrule
	\end{tabular}
\end{center}
\end{table}
381 beeing the total number of interactions (verbal that is)



\section{Hypothesis generation and testing}
Some intro to this category
\subsection{First hypothesis}
\subsubsection*{Context}
\label{firsthypothesis}
We enter the situation in the beginning of class. The students have been divided into groups, and we are approximately two minutes into filming. Preceding this discussion, the students have tried for about one minute to figure out what the task is, and what the experiment involved. "Siri" has read out loud the first question given to them: "what did you expect to happen" (in the experiment), and they have started to formulate some generic thoughts about what would happen (e.g soil moisture going down over time). Prior to the excerpt, the students have appeared a bit insecure about the task. But as we enter the setting they are more focused and interested, and the discussion has changed from general observations to generating hypotheses.

\subsubsection*{Raw data}
\def\arraystretch{1.5}
\begin{table}[H]
	\begin{adjustwidth}{-4em}{-4em}
		\begin{center}
			\begin{tabular}{r l p{7cm} p{3cm} } \toprule
					Time &  Who &  Speech  & Action \\ \midrule 
					2:04 %time
					&Nora %name
					&\parbox[t]{7cm}{\raggedright hehe.. mm.. hmhm .. når den stod i skapet så.. jeg visste ... %speech 
					}&\parbox[t]{3cm}{\raggedright  %action 
					}
					\\

					2:13 %time
					&Siri %name
					&\parbox[t]{7cm}{\raggedright ... neddi skapet ... %speech 
					}&\parbox[t]{3cm}{\raggedright  %action 
					}
					\\

					2:13 %time
					&Nora %name
					&\parbox[t]{7cm}{\raggedright eller jeg visste ikke helt hva den skull.. hva som skulle skje da egentlig .. %speech 
					}&\parbox[t]{3cm}{\raggedright  %action 
					}
					\\
				2:16 %time
				&Siri %name
				&\parbox[t]{7cm}{\raggedright .. det var det planten stod i skapet også skulle det være bare grønt lys på den ... men det kan jo hende for eksempel at det kom litt annet lys inn i skapet også .. så da er det ikke sikkert at det bare var grønt lys ..  %speech 
				}&\parbox[t]{3cm}{\raggedright peker på skapet %action 
				}
				\\

				2:31 %time
				&Nora %name
				&\parbox[t]{7cm}{\raggedright  %speech 
				}&\parbox[t]{3cm}{\raggedright nikker %action 
				} 
				\\

				2:31 %time
				&Siri %name
				&\parbox[t]{7cm}{\raggedright og planten tar jo opp littegrann grønt lys også, men ikke så mye .. så derfor kunne det hende atte den ikke vokste like my.. eller jeg trodde at den ikke ville vokse like mye i skapet .. siden da fikk den bare grønt lys ...  %speech 
				}&\parbox[t]{3cm}{\raggedright  %action 
				}
				\\
				2:46 %time
				&Nora %name
				&\parbox[t]{7cm}{\raggedright ... mmm ... %speech 
				}&\parbox[t]{3cm}{\raggedright  nikker%action 
				}
				\\
			\end{tabular}
		\end{center}
	\end{adjustwidth}
\end{table}

\subsubsection*{Explanation}
At first, Nora is not sure about what would happen to the plant given green light in the cupboard. Siri, as one who thinks out loud, promptly starts reflecting on what could have happened. First she proposes that the plant was not only given green light, indicating that there could be error sources to the experiment. This is acknowledged by a slight nod by Nora. Then she goes on to reflect about the wavelengths plants absorb, and given that they only absorb a small amount of green light, the plant in the cupboard would not grow as much as the plant in the window sill. Nora agrees to this hypothesis by nodding and saying "mmm". 



\subsection{Testing the hypothesis}

\subsubsection*{Context}
We enter the setting immediately after the excerpt explained in chapter \ref{firsthypothesis}. Siri has generated a hypothesis which she wants to test out. The mood in the group has now gone from laughter and insecurity about the task to concentration. The overall noise level in the class room has also fallen significantly. 

\subsubsection*{Raw data}
\begin{table}[H]
	
		\begin{center}
			\begin{tabular}{r l p{7cm} p{3cm} } \toprule
					Time &  Who &  Speech  & Action \\ \midrule 
2:47 %time
&Siri %name
&\parbox[t]{7cm}{\raggedright ...eller nesten bare grønt lys ihvertfall ... men hvor mye vokste den egentlig? er det den ((refererer til planten på bordet)) som stod i skapet? %speech 
}&\parbox[t]{3cm}{\raggedright peker på planten som står på pulten %action 
}\\

2:52 %time
&Sjur %name
&\parbox[t]{7cm}{\raggedright ja %speech 
}&\parbox[t]{3cm}{\raggedright  %action 
}\\

2:53 %time
&Nora %name
&\parbox[t]{7cm}{\raggedright OJ(!) %speech 
}&\parbox[t]{3cm}{\raggedright  %action 
}\\

2:53 %time
&Siri %name
&\parbox[t]{7cm}{\raggedright Den har jo vokst ganske mye %speech 
}&\parbox[t]{3cm}{\raggedright smiler %action 
}\\
2:59 %time
&Siri %name
&\parbox[t]{7cm}{\raggedright men var stilkene på den som stod i vinduet var de også hvite? %speech 
}&\parbox[t]{3cm}{\raggedright Peker mot vinduet %action 
}\\
		\end{tabular}
		\end{center}
	
\end{table}
\subsubsection*{Explanation}
After Siri has proposed the hypothesis that the plant in the cupboard would not grow as much as the plant in the window sill, she wants to find out if it holds true. Suddenly she notices the plant which is placed on the table in front of them, and asks "is it that one(!)?". When Sjur confirms, the whole group and especially Siri seems surprised. It seems like their knowledge of photosynthesis would point to the plant not growing as much due to none of the leaves' pigments absorbing light in the green spectrum. Thus, the first hypothesis generated by the group has been falsified. 

As a reaction to this Siri stops to think for a few seconds before she poses a question: "were the stems on the one in the window also white?". This is in fact a very valid scientific question, as a plant with absolutely no photosynthesis would most likely be white, as a result of having no pigments. The reason for her asking this may be related to a comment made by another student in a previous lecture. He had observed that when they put plants in the basement for winter storage, the leaves would turn white. 

The reason for Siri asking about the color of the stem might be that different pigments reflect different colors, the most common being chlorophyll reflecting green. If a plant does not receive light of the required wavelengths, photosynthesis does not happen and the pigments deteriorates. Thus, the plant turns white. 
\subsection{A new hypothesis}
\subsubsection*{Context}
This excerpt is from a situation occurring only a few seconds after the first hypothesis has been falsified. The group has then been instructed to interact with the system on the computer in front of them to find the answer to the question posed at 2:59 "were the stems on the one in the window also white?". When we enter the situation they have a video of the plant in the window sill on the screen, ready to play. 

\subsubsection*{Raw data}
\begin{table}[H]
		\begin{center}
			\begin{tabular}{r l p{7cm} p{3cm} } \toprule
					Time &  Who &  Speech  & Action \\ \midrule 
3:21 %time
&Nora %name
&\parbox[t]{7cm}{\raggedright Ja for karse har jo hvit stilk %speech 
}&\parbox[t]{3cm}{\raggedright  %action 
}\\

3:23 %time
&Siri %name
&\parbox[t]{7cm}{\raggedright Ja det de har hvit stilk de også %speech 
}&\parbox[t]{3cm}{\raggedright  %action 
}\\

3:24 %time
&Fredrik %name
&\parbox[t]{7cm}{\raggedright mhm ... mmja så da er det jo egentlig ganske ... ja ikke så stor forskjell da på de som stod ...  i skapet ((peker på planten på border)) og de som stod i vinduskarmen hvis man bare ser på ...  utseende %speech 
}&\parbox[t]{3cm}{\raggedright Dette sies mens Siri starter videoen, hun stopper også videoen før de har sett den halvferdig. %action 
}\\

3:37 %time
&Siri %name
&\parbox[t]{7cm}{\raggedright ja .. men da ville jeg kanskje tenke at det kan hende at det kom inn annet lys enn det grønne lyset også. siden de har vokst så bra, og at de vokser bedre hvis de får flere.. lys i flere bølgelengder enn bare grønt lys %speech 
}&\parbox[t]{3cm}{\raggedright Stemmeleiet går opp mot slutten av setningen, og løfter blikket fra arket for å få bekreftelse
 %action 
}\\
		\end{tabular}
		\end{center}
	
\end{table}
\subsubsection*{Explanation}
When this is falsified, Siri is quick to formulate a new hypothesis: light of other wavelengths has entered the cupboard. Her reasoning for this might be that if the plant were only given green light, it would become more pale than the plant in the window, thus some photosynthesis has happened.
This hypothesis also explains why her first hypothesis, that the plant would not grow as much as the other, failed.  

\section{Some headline about scaffolding}

\subsection{Hypothesis generation based on misconception}


\subsubsection*{Context}
In table~\ref{excerpt:disconfirmation1} the students have been looking at the movements of the two plants, and have observed that the plant in the window are moving towards the sun, a so called heliotropism. They are now observing the movements of the plant in the closet and Fredrik have just pointed out that it is growing straight up without any large movement like the other plant. All the students in the group are watching the video closely. Suddenly Nora observes that the plant is growing a lot faster and higher than the window plant.


\subsubsection*{Raw data}

\def\arraystretch{1.5}
\begin{table}[H]
\begin{adjustwidth}{-4em}{-4em}
\begin{center}
\begin{tabular}{r l p{7cm} p{3cm} } \toprule
	Time &  Who &  Speech  & Action\\ \midrule  

	7:46 %time
	&Nora %name
	&\parbox[t]{7cm}{\raggedright Jeg føler at de vokser veldig mye inni ... skapet eller er det? ... %speech 
	}&\parbox[t]{3cm}{\raggedright  %action 
	}\\

	7:51 %time
	&Siri %name
	&\parbox[t]{7cm}{\raggedright Ja det virka som om de vokste ... %speech 
	}&\parbox[t]{3cm}{\raggedright  %action 
	}\\

	7:53 %time
	&Nora %name
	&\parbox[t]{7cm}{\raggedright ... ser ut som de ble lenger lissom ... %speech 
	}&\parbox[t]{3cm}{\raggedright  %action 
	}\\

	7:53 %time
	&Siri %name
	&\parbox[t]{7cm}{\raggedright ... enda mer der. %speech 
	}&\parbox[t]{3cm}{\raggedright  %action 
	}\\

	7:54 %time
	&Fredrik %name
	&\parbox[t]{7cm}{\raggedright ja %speech 
	}&\parbox[t]{3cm}{\raggedright  %action 
	}\\

	7:56 %time
	&Siri %name
	&\parbox[t]{7cm}{\raggedright ... enn ute, at de ble mye lengre. %speech 
	}&\parbox[t]{3cm}{\raggedright  %action 
	}\\

	7:59 %time
	&Fredrik %name
	&\parbox[t]{7cm}{\raggedright mhm. %speech 
	}&\parbox[t]{3cm}{\raggedright  %action 
	}\\

	8:01 %time
	&Siri %name
	&\parbox[t]{7cm}{\raggedright Kanskje de fokuserer veldig på å vokse oppover når lyset er rett over dem.. at de vokser rett oppover ((fører hånden oppover)) i stedet for å følge lyset og gå lissom sånn sakte oppover ((snurrer hånden sakte oppover)) %speech 
	}&\parbox[t]{3cm}{\raggedright  %action 
	}\\
	
	
	\bottomrule
\end{tabular}
\end{center}
\end{adjustwidth}
\caption{Excerpt from disconfirmation of growth}
\label{excerpt:disconfirmation1}
\end{table}

\subsubsection*{Explanation}
When Nora states that the plants are growing more in the closet, she is very cautious when introducing the idea, it seems like an unlikely observation. Siri approves and states that it is indeed growing more than the plant outside of the closet, Fredrik agrees and they all seem a bit puzzled by this observation.
After they have agreed on this, Siri starts to generate a hypothesis for why the plant in the closet are growing more than the one in the window. It might look like it is hard for her to realize that her expectations where wrong and so she is trying to cope with this by finding explanations why the plants grew so different from what she expected. It seems like she is mixing germination and photosynthesis, which is not so weird, because the textbook does not say anything about germination. 
Basically what she is saying is that heliotropism makes the plant in the window grow slower because it has to move after the sun, and since the plant in the closet can grow straight up without following the sun, it can grow faster. 

\subsection{Scaffolding to fix misconception}

\subsubsection*{Context}
In table~\ref{excerpt:scaffold1} the teacher has been talking with the group for a couple of minutes and Siri has talked about her hypothesis that the plant in the closet might have got more than just green light, and if it only got green light it would probably not grow that much.
 
\subsubsection*{Raw data}

\def\arraystretch{1.5}
\begin{table}[H]
\begin{adjustwidth}{-4em}{-4em}
\begin{center}
\begin{tabular}{r l p{7cm} p{3cm} } \toprule
	Time &  Who &  Speech  & Action\\ \midrule  

	13:44 %time
	&Lærer %name
	&\parbox[t]{7cm}{\raggedright ja.. så altså dere tenker at .. sammenhengen mellom \underline{vekst} og fotosyntese den er helt klar ... du kan ikke du tenker at du kan ik et frø kan ikke spire og vokse og bli en plante uten at drives fotosyntese.. tenker dere alle det? %speech 
	}&\parbox[t]{3cm}{\raggedright  %action 
	}\\

	14:00 %time
	&Fredrik %name
	&\parbox[t]{7cm}{\raggedright Det er jo noen planter som ikke har fotosyntese ... og de spirer jo og fordet ikkesant.. det er vel en liten energipakke på en måte i  frøet da? er det ikke det da? %speech 
	}&\parbox[t]{3cm}{\raggedright  %action 
	}\\

	14:14 %time
	&Lærer %name
	&\parbox[t]{7cm}{\raggedright okei, er det? %speech 
	}&\parbox[t]{3cm}{\raggedright  %action 
	}\\

	14:14 %time
	&Nora %name
	&\parbox[t]{7cm}{\raggedright Ja %speech 
	}&\parbox[t]{3cm}{\raggedright nikker annerkjennende %action 
	}\\
	
	\bottomrule
\end{tabular}
\end{center}
\end{adjustwidth}
\caption{Excerpt from teacher talk}
\label{excerpt:scaffold1}
\end{table}

\subsubsection*{Explanation}
In the excerpt the teacher formulates a question which lead the group to think outside of the model of photosynthesis and hints to the germination process. Fredrik starts to answer the question at once, and introduces the notion that there are plant that do not have photosynthesis, and that the seeds have an energy pack. This notion lays the basis for a discussion in the group where the teacher leads the students to find out that seeds have starch as a food reserve, which makes germination possible. 

\subsection{Misconception not followed}

\subsubsection*{Context}
The following excerpt is from a situation occurring almost immediately after the excerpt in (reference to teacher intervention). The teacher has come up to the group to check on them, and has asked why the plant in the closet grew more than the one in the window. Siri then comes with a theory, leading the teacher to asking follow-up questions to get the group to generate more hypothesis. The conversation follows a pattern where the teacher asks a question, and the students answer. As we enter the setting, Siri has just presented an hypothesis. As the teacher asks for other explanations, all of the students are looking down on the textbook illustration of the light-dependent reaction placed on the table in front of them. 
\subsubsection*{Raw data}
\begin{table}[H]
	\begin{center}
		\begin{tabular}{r l p{7cm} p{3cm} } \toprule
			Time &  Who &  Speech  & Action \\ \midrule 
			12:34 %time
			&Lærer %name
			&\parbox[t]{7cm}{\raggedright ja det er et alternativ en alterna har dere noen andre eventuelle forklaringer? det kunne være andre forklaringer? %speech 
			}&\parbox[t]{3cm}{\raggedright  %action 
			}\\

			12:42 %time
			&Nora %name
			&\parbox[t]{7cm}{\raggedright kan jeg bar sp.. solener.. ehh kan det bare være lys også? %speech 
			}&\parbox[t]{3cm}{\raggedright Peker på ordet "solenergi" på modellen på arket %action 
			}\\

			12:45 %time
			&Lærer %name
			&\parbox[t]{7cm}{\raggedright Hva sier du %speech 
			}&\parbox[t]{3cm}{\raggedright bøyer seg frem for å høre bedre %action 
			}\\

			12:46 %time
			&Nora %name
			&\parbox[t]{7cm}{\raggedright Kan lys forårsake eksit.... at det eksiterer? eller bare sol? %speech 
			}&\parbox[t]{3cm}{\raggedright Tar fingeren langs pilen i modellen hvor det står "solenergi", og illustrerer at solenergi kommer inn til klorofyllmolekylene %action 
			}\\

			12:50 %time
			&Lærer %name
			&\parbox[t]{7cm}{\raggedright vanlig lys.. åja du mener lampe altså sånn grønt lys? %speech 
			}&\parbox[t]{3cm}{\raggedright  %action 
			}\\

			12:54 %time
			&Nora %name
			&\parbox[t]{7cm}{\raggedright mhm %speech 
			}&\parbox[t]{3cm}{\raggedright  %action 
			}\\

			12:55 %time
			&Lærer %name
			&\parbox[t]{7cm}{\raggedright Altså det er jo spørsmålet...  %speech 
			}&\parbox[t]{3cm}{\raggedright  %action 
			}\\

			12:57 %time
			&Nora %name
			&\parbox[t]{7cm}{\raggedright eller jeg mente ehh.. lys  %speech 
			}&\parbox[t]{3cm}{\raggedright peker opp mot lampene i taket %action 
			}\\
			12:57 %time
			&Siri %name
			&\parbox[t]{7cm}{\raggedright ... det var jo det de gjorde i skapet %speech 
			}&\parbox[t]{3cm}{\raggedright peker mot skapet %action 
			}\\

			12:58 %time
			&Lærer %name
			&\parbox[t]{7cm}{\raggedright Åja her inne? jammen få.. fikk de det inne i skapet? %speech 
			}&\parbox[t]{3cm}{\raggedright  %action 
			}\\

			13:00 %time
			&Nora %name
			&\parbox[t]{7cm}{\raggedright Nei jeg bare lurer jeg mm. %speech 
			}&\parbox[t]{3cm}{\raggedright  %action 
			}\\
		\end{tabular}
	\end{center}
\end{table}

\subsubsection*{Explanation}
After the teacher has asked if there can be any other explanations, Nora takes the opportunity to ask the question: "...ehh can it be light as well?". As she asks the question, she points at the word "solar energy" in the illustration of the light-dependent reaction (see figure in photosynthesis chapter). The teacher does not quite understand what she is asking, and therefore leans in and ask her to repeat the question. She then reformulates the question in a scientific language, asking if only sunlight can excite chlorophylls, and not artificial light. As she says the word "excite", she is pointing at the illustration of the chlorophyll molecule, and as she says "sun", she is pointing at the word "solar energy".

When Nora is asking these questions, she is clearly referring to the illustration in front of her. The reason for Nora asking this is that in the illustration, photons are labeled as "solar energy". This is probably done to simplify the model, but in this case it leads to a big misconception. As we can see from her questions, she is unsure if artificial light can cause photosynthesis (which it can). If this were the case, Nora could rule out photosynthesis as the cause of the cupboard plant growing more than the window plant.

The teacher then proceeds to ask her if she means a lamp with green light, whereupon she confirms by saying "mmm". When the teacher replies that it is the question they are supposed to answer, she quickly replies that she meant artificial light, by pointing at the fluorescent lighting in the class room. The teacher then misinterprets, and think she is speaking of the specific lighting in the classroom, and not artificial lighting in general.

After Noras question regarding the "erroneous" representation in the model, and the teachers failure to understand the motivation behind the question, the discussion quickly takes another turn. The question is left hanging, and is not followed up later in the session.  


\section{Language}

\subsection{Teacher intervention}

\subsubsection*{Context}
The discussions preceding this excerpt has been a bit slow, leading us to intervene more in the situation, and asking them more questions. The students still seem interested and concentrated, with Siri taking the lead. The language used by the participants has been informal, and most utterances has been related to observations. A few seconds prior to the excerpt Sjur has instructed them to flip the paper sheet with the tasks, revealing an illustration from their text-book of the light-dependent reaction. 
\subsubsection*{Raw data}
\begin{table}[H]
	\begin{center}
		\begin{tabular}{r l p{7cm} p{3cm} } \toprule
			Time &  Who &  Speech  & Action \\ \midrule 
			11:20 %time
			&Lærer %name
			&\parbox[t]{7cm}{\raggedright Går det bra eller %speech 
			}&\parbox[t]{3cm}{\raggedright kommer bort til bordet og lener seg på det.%action 
			}\\

			11:23 %time
			&Siri %name
			&\parbox[t]{7cm}{\raggedright mmm, ja %speech 
			}&\parbox[t]{3cm}{\raggedright  alle nikker%action 
			}\\

			11:24 %time
			&Lærer %name
			&\parbox[t]{7cm}{\raggedright skjønner dere ... har dere funnet forklaring på alle spørsmålene? %speech 
			}&\parbox[t]{3cm}{\raggedright  %action 
			}\\

			11:26 %time
			&Alle jentene %name
			&\parbox[t]{7cm}{\raggedright *** vi prøver ... %speech 
			}&\parbox[t]{3cm}{\raggedright snakker i munnen på hverandre %action 
			}\\

			11:27 %time
			&Siri %name
			&\parbox[t]{7cm}{\raggedright Jeg tror kanskje jeg har en ide om det med at den her ute ((peker mot vinduet, refererer til planten i vinduet)) ikke vokser like høyt, eller så fort ihvertfall.. fordi atte når det kommer veldig mye sol så blir jo \textbf{klorofyllmolekylene eksitert}, men når alle ... alle \textbf{klorofyllene} blir \textbf{eksitert} i planten, sånn atte det ikke er flere som kan bli \textbf{eksitert} så hjelper det ikke om det er mere lys. %speech 
			}&\parbox[t]{3cm}{\raggedright  %action 
			}\\
		\end{tabular}
	\end{center}
\end{table}
\subsubsection*{Explanation}
When the teacher approaches the group, Siri's language quickly change from explaining things in everyday terms, to a more scientific language. After roughly 11 minutes of discussion, this is the first occurrence of the words "excited", "clorophyl", and "molecules" (bold text in excerpt). So even though they may have been using the material in the textbook before, this is the first time the link between the different artifacts is clear. 

One reason for the sudden change in language may be that only seconds before the excerpt, the students looked at the figure from the textbook, representing the light-dependent part of photosynthesis. This may have led Siri onto a more theoretical path, causing her to try and explain the phenomena, using the scientific language. 

Another explanation may be that when the teacher asks a question, the students think he's assessing the answer. Thereby creating a test-like situation, where Siri is eager to express her knowledge about the photosynthesis model explained in the textbook. 



\subsection{Everyday language}


\subsubsection*{Context}
Teacher has left, Morten asked the students to look at the plant videos and see if there is any difference in their appearance. The students have looked at the plant i the cupboard and found that it is mostly the stem that grows, not the leafs. Fredrik has requested that they should check the window plant to compare the two, and Siri has started the video from 29th of October.


\subsubsection*{Raw data}

\def\arraystretch{1.5}
\begin{table}[H]
\begin{adjustwidth}{-4em}{-4em}
\begin{center}
\begin{tabular}{r l p{7cm} p{3cm} } \toprule
	Time &  Who &  Speech  & Action\\ \midrule  

	17:12 %time
	&Siri %name
	&\parbox[t]{7cm}{\raggedright Der åpner jo bladene seg med en gang nesten %speech 
	}&\parbox[t]{3cm}{\raggedright  %action 
	}\\

	17:15 %time
	&Fredrik %name
	&\parbox[t]{7cm}{\raggedright ja ... ((stillhet, venter til video er ferdig)) det kan jo ha noe med at her trenger den jo bladene for fange lyset da, mens den trenger jo ikke det så mye inni skapet.. eh kanskje %speech 
	}&\parbox[t]{3cm}{\raggedright Planten trenger ikke bladene i skapet fordi det ikke er så mye lys? %action 
	}\\

	17:34 %time
	&Siri %name
	&\parbox[t]{7cm}{\raggedright at den bruker næringen fra jorda og frøet mer i skapet? %speech 
	}&\parbox[t]{3cm}{\raggedright  %action 
	}\\

	17:37 %time
	&Fredrik %name
	&\parbox[t]{7cm}{\raggedright ehhhh.. ja. eller at den ikke utnytter den sol.. det sollyset inne i skapet så det den trenger jo ikke da også at bladene spretter ut så tidlig eller at... eh ja. %speech 
	}&\parbox[t]{3cm}{\raggedright Fredrik er ikke helt enig med Siri. Mener at planten i skapet ikke har noe lys å utnytte, derfor ingen blader %action 
	}\\

	\bottomrule
\end{tabular}
\end{center}
\end{adjustwidth}
\caption{Excerpt from some usage of everyday language}
\label{excerpt:everydaylanguage}
\end{table}

\subsubsection*{Explanation}
In the excerpt in table~\ref{excerpt:everydaylanguage} the students are using everyday words like \emph{leaf, sunlight, pops out, makes use of} etc. to explain what is happening with the plant. 

\subsection{Scientific language}
\subsubsection*{Context}
The students are working with task 3 regarding soil moisture, and differences in absorption/evaporation rate in the experiments. Most of the discussion has been regarding general observations, and they are struggling to form hypotheses. The main observation is that there are major differences in the absorbption rate in the two expreiments. In an effort to push the discussion further, Sjur has started to intervene, asking what it could mean in terms of photosynthesis that the soil moisture level drops less in the end of the experiment. Approximately one minute before the excerpt, the teacher has tried to position himself discretely behind the group, but all the students except Linda has noticed him. As we enter the setting, Nora has taken control over the discussion, as Siri has tried and failed to generate an answer. 

\subsubsection*{Raw data}
\begin{table}[H]
	\begin{center}
		\begin{tabular}{r l p{7cm} p{3cm} } \toprule
			Time &  Who &  Speech  & Action \\ \midrule 
			29:16:00 %time
			&Nora %name
			&\parbox[t]{7cm}{\raggedright men det er sånn...fordi vi har jo...det er jo den \textbf{lysuavhengige} delen av \textbf{fotosyntesen} også...jeg vet ikke om den har...\textbf{atp} og \textbf{nadph} fra f... %speech 
			}&\parbox[t]{3cm}{\raggedright ser mot Sjur mens hun snakker, vender seg mot Fredrik når han avbryter henne %action 
			}\\

			29:26:00 %time
			&Fredrik %name
			&\parbox[t]{7cm}{\raggedright ...den må jo ha den...først drive den lys... eller den må jo drive den \textbf{lysavhengige} også for å drive den \textbf{lysuavhengige} %speech 
			}&\parbox[t]{3cm}{\raggedright bruker hendene til å vise at den lysuavhengige reaksjonen er avhengig av den lysavhengige reaksjonen %action 
			}\\

			29:35:00 %time
			&Siri %name
			&\parbox[t]{7cm}{\raggedright mhm %speech 
			}&\parbox[t]{3cm}{\raggedright  %action 
			}\\

			29:36:00 %time
			&Fredrik %name
			&\parbox[t]{7cm}{\raggedright ...den har vel ikke \underline{atp} eller \underline{nadph} fra før av? %speech 
			}&\parbox[t]{3cm}{\raggedright alle ler %action 
			}\\

			29:44:00 %time
			&Nora %name
			&\parbox[t]{7cm}{\raggedright ja det var det jeg lurte på også %speech 
			}&\parbox[t]{3cm}{\raggedright  %action 
			}\\

			29:46:00 %time
			&Siri %name
			&\parbox[t]{7cm}{\raggedright nei det er vel den \textbf{lysavhengige} reaksjonen bruker til å danne det? %speech 
			}&\parbox[t]{3cm}{\raggedright  %action 
			}\\
		\end{tabular}
	\end{center}
\end{table}

\subsubsection*{Explanation}
After not having managed to explain the observation of difference in absorption rate in the two experiments, Nora suddenly switches to a more scientific language than in the minutes preceding this discussion. The words emphasized in bold can be found again both in the textbook representation of photosynthesis, and in language used by the teacher in earlier presentations of the material. There may be several reasons to this sudden change in language. First: Sjur has asked a question, and while she is saying this, she is looking at him as if he knows the answer, leading to a test-like situation. Second: the teacher is standing behind her listening to the whole situation. Or, she is simply trying to bring in another representation as the students have not been able to explain the phenomena with the use the physical plant and the system. 


\section{Linking between representations}
\subsection{Soil moisture representation}


\subsubsection*{Context}
The students have read the introduction to assignment 3, Siri has navigated to the soil moisture graph in the system, and got some help from Sjur to expand the graph so they could watch the entire graph. After expanding the graph to include the lifespan of both plants, Siri leaves the system to read assignment 3a: \emph{Is there any difference in the absorption rate?}



\subsubsection*{Raw data}

\def\arraystretch{1.5}
\begin{table}[H]
\begin{adjustwidth}{-4em}{-4em}
\begin{center}
\begin{tabular}{r l p{7cm} p{3cm} } \toprule
	Time &  Who &  Speech  & Action\\ \midrule  

	21:34 %time
	&Nora %name
	&\parbox[t]{7cm}{\raggedright Åja, fra de... den og den ((peker på høyre og venstre side av grafen)) %speech 
	}&\parbox[t]{3cm}{\raggedright Lager v-tegn med fingrene og viser hvilken periode i grafen planten var i vinduet, og hvilken periode den var i skapet %action 
	}\\

	21:36 %time
	&Sjur %name
	&\parbox[t]{7cm}{\raggedright ja. %speech 
	}&\parbox[t]{3cm}{\raggedright  %action 
	}\\

	21:37 %time
	&Siri %name
	&\parbox[t]{7cm}{\raggedright Åja, så det der er den ene planten og det der er den andre.. %speech 
	}&\parbox[t]{3cm}{\raggedright Peker først på venstre side av grafen, så på høyre %action 
	}\\

	21:41 %time
	&Nora %name
	&\parbox[t]{7cm}{\raggedright mhm, den der går litt brattere ned på ... %speech 
	}&\parbox[t]{3cm}{\raggedright Peker på området i grafen hvor planten sto i skapet %action 
	}\\

	21:44 %time
	&Fredrik %name
	&\parbox[t]{7cm}{\raggedright Ja, den går mye brattere ned. %speech 
	}&\parbox[t]{3cm}{\raggedright  %action 
	}\\

	21:46 %time
	&Siri %name
	&\parbox[t]{7cm}{\raggedright Kanskje det betyr at den der andre planten bruker mye mer fuktighet fra jorden %speech 
	}&\parbox[t]{3cm}{\raggedright Peker på området i grafen hvor planten sto i skapet %action 
	}\\

	\bottomrule
\end{tabular}
\end{center}
\end{adjustwidth}
\caption{Excerpt from some important stuff}
\label{excerpt:hypothesis3.1}
\end{table}

\subsubsection*{Explanation}
Here the students are looking at Monoplants representation of the soil moisture over time. They interpret it as how wet the soil is, and when curves in the graph go downwards, it represents when the plant is using water. Since the curves from the plant in the cupboard is much steeper than from the one in the window, Siri interpret this to mean that the plant in the cupboard uses a lot more water.