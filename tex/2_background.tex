%!TEX root = ../document.tex
\chapter{Scientific background}
\label{cha:scientificbg}
In this chapter we will give a rudimentary introduction to photosynthesis as it is described in the curriculum for Biology 2 \citep{bios}. This is provided as a tool to give the reader some background for understanding the domain specific discussions of the students.

\section{Photosynthesis}
The variables that we monitor in our application are directly linked to one of the preconditions of all life on Earth, photosynthesis. During this process plants transform energy from light to chemical energy in the form of e.g., glucose and starch. As most organisms are not able to utilize the energy of light directly, plants are a necessity for producing energy that other organisms can transform. The equation for photosynthesis is written as: \ce{(CO2)n + (H2O)n + photons -> (CH2O)n + (O2)n}, which means that carbon dioxide, water and light transform to glucose and oxygen.  

Photosynthesis consists of two main parts: the light-dependent reaction, and the light-independent reaction. The light-dependent reaction, as the name implies, occur only in the light. The light-independent reaction occurs both in the light and dark, but does not rely on energy from photons.  

\begin{figure}
\centering
\includegraphics[width=0.8\textwidth]{img/photosynthesis/Chloroplast_diagram.png}
\caption{Illustration of a chloroplast molecule \citep{wiki:chloroplast}}
\label{fig:chloroplast}
\end{figure}

\subsection{Light-dependent reaction}
The light-dependent reaction consists of two different photosystems (photosystem 1 and photosystem 2) creating adenosine triphosphate (ATP) and nicotinamide adenine dinucleotide phosphate (NADPH) molecules for the light-independent reactions. Both systems are located in the thylakoid membrane inside the chloroplast organelles (see fig.~\ref{fig:chloroplast}). In the process, photosystem 2 precedes photosystem 1 as photosystem 1 was discovered first. 

\subsubsection*{Photosystem 2}
In photosystem 2, antenna-complexes consisting of pigments, proteins and enzymes absorb light of different wavelengths and transfer the energy to chlorophyll molecules \citep{bios}. The energy leads to electrons jumping to an orbit lying further from the nucleus, making the atom excited. This makes the atom unstable, and a perfect candidate for giving away its electrons to electron-acceptors in an electron-transport chain.

Since the chlorophyll loses two of its electrons in the process, it gets positively charged and need to find new electrons to be able to absorb photons again. This happens by taking two electrons from a water molecule absorbed by the plant's roots, which then gets split into \ce{2H+} and \ce{1/2O2} \citep{bios}. The oxygen dissolves in the air, while the hydrogen protons are “trapped” on the inside of the thylakoid membrane (lumen). This makes the lumen positively charged relative to the stroma, which enables generation of ATP-molecules from ADP- and P-molecules. 

\subsubsection*{Photosystem 1}
Photosystem 1 consists of the same parts as photosystem 2, but instead of splitting water molecules, it receives two electrons from the electron transport chain in photosystem 2. These electrons get transferred out in the stroma, and are then tied together with an h+-proton and NADP+ to produce NADPH.

\begin{figure}
\centering
\includegraphics[width=\textwidth]{img/photosynthesis/light_dependent.png}
\caption{Illustration of photosystem 1 and photosystem 2 \citep{bios}}
\label{fig:photosystem}
\end{figure}

\subsection{Light-independent reaction (Calvin-cycle)}
This reaction works as a “sugar-factory”, collecting carbon dioxide and hydrocarbon in many cycles to make glucose. The process takes place in the stroma (see fig.~\ref{fig:chloroplast}), and requires the NADPH and ATP generated in the light-dependent reaction \citep{bi2}. 

The glucose produced can be used to generate other organic compounds such as other carbohydrates (e.g., starch and cellulose), proteins and lipids, depending on what the plant needs.

\subsection{External factors}
Many external factors affect the photosynthesis in plants. As photosynthesis is a relatively inefficient process, using only 8-10\% of the energy in sunlight, much research has gone into increasing photosynthesis to achieve greater conversion rates \citep{kirschbaum2011does}. The factors of significance are \citep{bios}:
\begin{itemize}
\item \ce{CO2} levels
\item Temperature
\item Light intensity and wavelength
\item Water
\end{itemize}
Each of these factors may be a limiting factor, or stress factor, not enabling photosynthesis to reach its full potential. 

\subsubsection{\ce{CO2} levels}
\ce{CO2} is used in the light-independent reaction for making glucose. The atmosphere contains approximately 0.038\% \ce{CO2}, while the air in e.g., a classroom would most likely contain slightly higher values due to a high concentration of students exhaling \ce{CO2}. In a greenhouse \ce{CO2} levels can get too low, due to a high concentration of plants consuming \ce{CO2} and outputting \ce{O2}. The optimal concentration for most plants is between 0.015\% and 0.05\% \citep{bios}. 


\begin{figure}
        \centering
        \begin{subfigure}[b]{0.45\textwidth}
                \includegraphics[width=\textwidth]{img/photosynthesis/co2.png}
                \caption{Effect of \ce{CO2} levels on photosynthesis}
                \label{fig:co2levels}
        \end{subfigure}
        ~~
        \begin{subfigure}[b]{0.45\textwidth}
                \includegraphics[width=\textwidth]{img/photosynthesis/light_intensity.png}
                \caption{Effect of light intensity (lux) on photosynthesis}
                \label{fig:lightintensity}
        \end{subfigure}
       
\end{figure}

\begin{figure}
\centering
\includegraphics[width=0.75\textwidth]{img/photosynthesis/temperature_new.png}
\caption{Effect of temperature on photosynthesis. Species:
 \textit{\ensuremath{\blacktriangle} Pinus Taeda},  
\textit{\ensuremath{\bigcirc} Pinus Strobus}, 
\textit{\ensuremath{+} Pinus Sylvestris}, 
\textit{\ensuremath{\blacksquare} Picea Engelmanii}, 
\textit{\ensuremath{\times} Pinus Ponderosa}
\citep{hollinger1995external}
}
\label{fig:temperature}
\end{figure}

\subsubsection{Temperature}
All enzymes have an optimal temperature during which they function best \citep{bios}. This temperature may vary from species to species as plants grow in different climates, altitudes and seasons. If the temperature is too low or too high, the molecular structure of the enzymes may be destroyed.

\subsubsection{Light intensity and wavelength}
The different pigments in the light dependent reaction absorb light of wavelengths from mainly 400nm to 700nm. Chlorophyll b for instance absorbs blue light (450nm). If a plant with a high concentration of chlorophyll b is not given light of this wavelength, the electrons would not be excited and the reaction in photosystem 2 would not start.

Light intensity also plays a role in this reaction. In low light conditions, there is not enough energy available to excite the chlorophyll molecules, in order to move electrons as needed in photosystem 2. In optimal light conditions, the production is light saturated meaning that all the chlorophyll molecules are exciting electrons. In too strong light conditions, the chloroplasts may burn out from the heat and die.  

\subsubsection{Water}
Water is used in both the light-dependent and light-independent reactions, but is seldom a limiting factor. If water levels are low and the evaporation-rate is high, most plants will close their leaves to minimize water loss. This makes the plant unable to absorb \ce{CO2} and photons, which leads to plant reduction \citep{bi2}. Water shortage is only a problem in itself when the plant's cells dries out, leading to the stem and tissue collapsing. 



