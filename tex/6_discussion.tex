%!TEX root = ../document.tex
\chapter{Discussion}
In this chapter we will discuss our research question by contextualizing our findings to the analytical concepts introduced earlier . As an overall theme we will look at the inquiry process of the students in interaction with Monoplant. This will manifest through 4 sections, the first being about the inquiry process. Further we will discuss how multiple External representations support the inquiry process of the students, how Scaffolding is instatiated in the environment and how the institutional setting frame the students' inquiry process.

\section{Inquiry process}
In this section we will broadly address our first research question: What characterizes the students’ inquiry in interaction with monoplant? 
In the previous chapter we presented excerpts from the session where the students interacted with monoplant. We have seen that the students were generating hypotheses of what happened with the plant and why it grew as much as it did. We showed examples of explanations, discussion, misconceptions and surprises. 
The experiments conducted and the session with the students can be categorized as \emph{guided inquiry} \citeauthor*{staver1987analysis} \citetext{\citeyear{staver1987analysis}, referenced in \citealp{prince2006inductive}}. The students are given a problem (the questions) but need to figure out answers by themselves (with the help of Monoplant). 

The four problems that \citep{de1998scientific} addressed for inquiry learning where: \textit{hypothesis generation}, \textit{design of experiments}, \textit{interpretation of data}, and \textit{regulation of discovery learning}. In our case it can be said that we controlled two of these problems by designing and initiating the experiments for the students, as well letting Monoplant do a systematic logging of data during the experiment, hence regulating the inquiry process. This meant that the students faced two problems, that of interpreting the data as well as generating hypotheses based on the data. \citeauthor*{klahr1993heuristics} \citetext{\citeyear{klahr1993heuristics}, referenced in \citealp{de1998scientific}} reported that misinterpretation of data often result in confirmation of current hypothesis. 

As we showed in excerpt 1, Siri had a hypothesis in which she stated that plant B would not grow much, because it would not get as much light as plant A. In excerpt 2 data was presented to her that showed that plant B had indeed grown much. It can be argued that her interpretation of this data and her new hypothesis in excerpt 3 was directed by her first hypothesis, since the data collided with her first hypothesis the next hypothesis is claiming there might be some error source in the experiment. Hence the misinterpretation did not result in a direct confirmation of the previous hypothesis, instead it lay the basis for a denial of a disconfirmation, or an explanation for why the first hypothesis did not hold.

As the students were done with the textbook chapter of photosynthesis and had connected theoretical explanations to phenomenas such as growth, it can be said that the presumptions of the outcomes to the experiment colored the interpretation of data. Siri knew that plants makes food for themselves by doing photosynthesis. To do photosynthesis, a green plant such as the cress in the experiment need light of wavelengths other than green (blue and red). This reasoning is enabled in Siri because she know a lot of the scientific concepts concerning the theme at hand. We can almost say that the inquiry process became deductive as it was affected by the students preconceptions. This relates to misconceptions in inquiry learning, and what \citep{gomez2008elementary} refers to as inconsistent understanding according to what the teacher intended. In this case Siris conception of photosynthesis, which in the context of the textbook examples makes sense, becomes a misconception when she is confronted with a plant that germinates, hence leads her to erroneous conclusions. As mentioned earlier \citet{smith1994misconceptions} makes the description of this kind of misconceptions as "faulty extentions of productive prior knowledge". A conception might describe a phenomena in one context, and falsely describe it in another context. A similar statement to that of \citeauthor{klahr1993heuristics}: 

\begin{quote}"compared to the binary feedback provided to subjects in the typical psychology experiment, real-world evidence evaluation is not so straightforward" \citetext{\citet[p. 114]{klahr1993heuristics}, referenced in \citealp{de1998scientific}}
\end{quote}

We have now established that the inquiry process is influenced by the fact that the students have a certain knowledge about the phenomena at hand. This can at one side lead to misconceptions, since the students have a greater freedom to pursue ideas that can lead to different conclusions than intended by the instructor. Therefore it becomes important to guide the students on the right tracks as we will discuss later in the section about scaffolding.

balbalbalbalblblalbalblablalblablalblablablalbalblaalbalb

The students tend to talk with an everyday language, explaining what they see when looking at videos and images in the system. Even though the inquiry process is affected by their presumptions and knowledge about photosynthesis, they merely speak in scientific terms when explaining what they see. 



\section{Multiple External Representations in Inquiry processes}
During the inquiry process the students were presented with different representations of the photosynthesis phenomenon. In this section we will give account for how they were used in the inquiry process, and how they complimented and constrained each other. We will also look at differences in the students' language when engaging with the different representations. To recap, our second research question is as follows: "How does Monoplant, by presenting photosynthesis differently from the text book, support the inquiry process?"

\subsection{heading}
%How does textbook present photosynthesis
When we reviewed the literature used in the science education, we found that the scientific concepts are represented in a purely theoretical manner. In the first paragraph of the chapter concerning photosynthesis, scientific words such as "pigments", "chloroplasts" and "glucose" appears. Later on, the concept is explained by it's chemical formula and the chapter rarely gives any examples of how photosynthesis affect the life of plants concretely. The textbook therefore emphasizes how photosynthesis fits into a larger system of scientific concepts, and is more concerned with giving the big picture than the specific and concrete experiences. 

%How does monoplant represent photosynthesis
Monoplant on the other hand affords a more inductive or "bottom-up" approach. As a learning resource, Monoplant is a tool for exploring ideas related to photosynthesis. The variables relevant for the plant's photosynthesis are mediated through graphs and videos, but leaving the interpretation of these data in the hands of the students. The system is only concerned with one plant in one specific context, and not trying to generalize from the specific results to a larger scientific concept. 

When looking at our data with this in mind, a pattern in the students' language emerge. During the inquiry process, students use everyday language when engaging with Monoplant. An example of this is from excerpt 10 where one of the students says that the plant "use moisture from the earth". Another example is from excerpt 7 where students use concepts as "pop out", and "capture" and "use sunlight". All of these concepts have their scientific counterpart which is represented in the textbook, but when discussing among themselves, the students choose to talk about the phenomenon in a "non-academic" way. 

However, the students language seem to change when engaging with representations linked to the textbook. An example of this is from excerpt 8 where one of the students use scientific concepts such as "chlorophyll molecule" and "excited" when looking at a textbook illustration of photosynthesis. 

An explanation of this may be given by applying \citet{vygotsky2012thought} theory of spontaneous and scientific concepts as presented in chapter \ref{cha:spontaneous_scientific} on page \pageref{cha:spontaneous_scientific}. In the students' utterances they are addressing concrete results in a concrete experiment in a specific context. The concepts they use are therefore linked to what they are observing. When one of the students says that the leaves "pop out", it is because that is what she is observing in the video. The corresponding scientific concept in this case would be "heliotropism", but the student may not have saturated the concept with enough experiences, and as a result she uses the spontaneous concept: to "pop out". 


When the students are 

%Monoplant as spontaneous concept

%Inquiry-based learning in general as spontaneous concept

\subsection{} 


How does Monoplant, by visualizing/present photosynthesis differently from the text book, support the inquiry process? 
Here we can talk about: 
\begin{itemize}
\item{how do the textbook represent photosynthesis/external factors}
\item{how do Monoplant represent photosynthesis/external factors}
\item{\nameref{ex:excerpt10} (soilmoist)}
\item{\nameref{ex:excerpt6} (misconception from textbook representation)}
\item{Excerpts: 2, 3, 4, 6, 7, 8, 10}
\item{MER}
\item{Social practices}
\item{Everyday language \& Scientific language}
\end{itemize}




\section{Scaffolding}
How is scaffolding instantiated in the environment?
Here we can talk about: 
\begin{itemize}
\item{ZPD}
\item{Scaffolding}
\item{Inquiry learning}
\item{Misconceptions}
\item{Excerpts: (3), 4, 5, 6, 8, 9}
\end{itemize}



\section{Institutional setting}
How does the institutional setting frame the students inquiry process? (“doing school \& doing science”)
Here we can talk about: 
\begin{itemize}
\item{Institutional settings}
\item{Inquiry learning}
\item{Everyday language \& Scientific language}
\item{Excerpts: 1, 5, 8, 9}
\end{itemize}
