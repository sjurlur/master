\section{Theoretical perspective}

\section{Sociocultural}
The sociocultural perspective may be thought of as the synthesis of the behavioristic thesis and cognitive antithesis. The behavioristic model focuses on the role of the individual and the notion that knowledge arises through individual drill and practice. In contrast, the cognitive model focuses on the environment and the notion that the environment provides raw material for testing innately conceived hypotheses, thus focusing on instruction methods and acquisitional models. The sociocultural perspective considers the individual in the context of their environment, with a focus on means of mediation between the two. Thus, action is the primary unit in sociocultural analysis, which will be elaborated further in the following paragraphs. 

In a biological sense the human species has not evolved significantly the last ten thousand years or so. The biological changes are only minor, and can’t explain the difference intellectually from the world view of people of the stone age and ourselves. Still we are able to achieve tasks that would have been incomprehensible for our ancestors \citep{saljo2001laering}.