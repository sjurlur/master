tid %time
&Hvem %name
&\parbox[t]{5cm}{\raggedright Verbalt %speech 
}&\parbox[t]{4cm}{\raggedright Non-verbalt %action 
}&\parbox[t]{4cm}{\raggedright kommentar %comment 
}\\

 %time
& %name
&\parbox[t]{5cm}{\raggedright  %speech 
}&\parbox[t]{4cm}{\raggedright  %action 
}&\parbox[t]{4cm}{\raggedright  %comment 
}\\

Hypotesetesting: Utelukker at temperaturen er merkverdig forskjellig i skapet og vinduet %time
& %name
&\parbox[t]{5cm}{\raggedright  %speech 
}&\parbox[t]{4cm}{\raggedright  %action 
}&\parbox[t]{4cm}{\raggedright  %comment 
}\\

24:56:00 %time
&Nora %name
&\parbox[t]{5cm}{\raggedright ..så vi det på?.. %speech 
}&\parbox[t]{4cm}{\raggedright  %action 
}&\parbox[t]{4cm}{\raggedright  %comment 
}\\

24:57:00 %time
&Siri %name
&\parbox[t]{5cm}{\raggedright Temperaturen? %speech 
}&\parbox[t]{4cm}{\raggedright  %action 
}&\parbox[t]{4cm}{\raggedright  %comment 
}\\

24:58:00 %time
&Nora %name
&\parbox[t]{5cm}{\raggedright ...mhm.. %speech 
}&\parbox[t]{4cm}{\raggedright  %action 
}&\parbox[t]{4cm}{\raggedright  %comment 
}\\

24:59:00 %time
&Siri %name
&\parbox[t]{5cm}{\raggedright Jeg tror det %speech 
}&\parbox[t]{4cm}{\raggedright Går inn på siden med oversikt over alle videoene %action 
}&\parbox[t]{4cm}{\raggedright  %comment 
}\\

25:02:00 %time
&Siri %name
&\parbox[t]{5cm}{\raggedright Ihvertfall ganske lik %speech 
}&\parbox[t]{4cm}{\raggedright Alles oppmerksomhet rettet mot skjerm %action 
}&\parbox[t]{4cm}{\raggedright  %comment 
}\\

25:03:00 %time
&Nora %name
&\parbox[t]{5cm}{\raggedright gå på grafen ((refererer til temperaturgraf på skjermen)) %speech 
}&\parbox[t]{4cm}{\raggedright  %action 
}&\parbox[t]{4cm}{\raggedright  %comment 
}\\

25:04:00 %time
&Siri %name
&\parbox[t]{5cm}{\raggedright ..det står...men det står her også %speech 
}&\parbox[t]{4cm}{\raggedright  %action 
}&\parbox[t]{4cm}{\raggedright Står her også refererer til maks og minimumsverdiene for dagene videoene er fra %comment 
}\\

25:09:00 %time
&Siri %name
&\parbox[t]{5cm}{\raggedright ...hvor er det? ((snakker til seg selv)) %speech 
}&\parbox[t]{4cm}{\raggedright Klikker %action 
}&\parbox[t]{4cm}{\raggedright  %comment 
}\\

25:12:00 %time
&Nora %name
&\parbox[t]{5cm}{\raggedright fjerde november %speech 
}&\parbox[t]{4cm}{\raggedright Peker mot skjermen %action 
}&\parbox[t]{4cm}{\raggedright  %comment 
}\\

25:18:00 %time
&Siri %name
&\parbox[t]{5cm}{\raggedright mellom 21 og 22...litt rundt det? %speech 
}&\parbox[t]{4cm}{\raggedright  %action 
}&\parbox[t]{4cm}{\raggedright Refererer til maks og minimumsverdiene for temperaturen i videoen mandag 4. november %comment 
}\\

25:24:00 %time
&Nora %name
&\parbox[t]{5cm}{\raggedright mhm %speech 
}&\parbox[t]{4cm}{\raggedright  %action 
}&\parbox[t]{4cm}{\raggedright  %comment 
}\\

25:25:00 %time
&Siri %name
&\parbox[t]{5cm}{\raggedright Også...den andre %speech 
}&\parbox[t]{4cm}{\raggedright  %action 
}&\parbox[t]{4cm}{\raggedright  %comment 
}\\

25:26:00 %time
&Nora %name
&\parbox[t]{5cm}{\raggedright ja %speech 
}&\parbox[t]{4cm}{\raggedright  %action 
}&\parbox[t]{4cm}{\raggedright  %comment 
}\\

25:27:00 %time
&Fredrik %name
&\parbox[t]{5cm}{\raggedright ja %speech 
}&\parbox[t]{4cm}{\raggedright  %action 
}&\parbox[t]{4cm}{\raggedright  %comment 
}\\

25:28:00 %time
&Siri %name
&\parbox[t]{5cm}{\raggedright ja %speech 
}&\parbox[t]{4cm}{\raggedright  %action 
}&\parbox[t]{4cm}{\raggedright  %comment 
}\\

25:29:00 %time
&Siri %name
&\parbox[t]{5cm}{\raggedright Det var cirka likt! %speech 
}&\parbox[t]{4cm}{\raggedright smiler %action 
}&\parbox[t]{4cm}{\raggedright Refererer til maks og minimumsverdiene for temperaturen i videoen tirsdag 29. oktober i forhold til mandag 4. november %comment 
}\\

25:40:00 %time
&Nora %name
&\parbox[t]{5cm}{\raggedright det var circa en grad forskjell %speech 
}&\parbox[t]{4cm}{\raggedright  %action 
}&\parbox[t]{4cm}{\raggedright  %comment 
}\\

25:41:00 %time
&Siri %name
&\parbox[t]{5cm}{\raggedright Ja, men jeg tror ikke det har noe å si, %speech 
}&\parbox[t]{4cm}{\raggedright  %action 
}&\parbox[t]{4cm}{\raggedright  %comment 
}\\

25:41:00 %time
& %name
&\parbox[t]{5cm}{\raggedright  %speech 
}&\parbox[t]{4cm}{\raggedright Siri går inn på grafen over jordfuktighet og zoomer ut slilk at grafen viser hele perioden.  %action 
}&\parbox[t]{4cm}{\raggedright  %comment 
}\\

Går videre til oppgave 3c ettersom de allerede har snakket om 3b %time
& %name
&\parbox[t]{5cm}{\raggedright  %speech 
}&\parbox[t]{4cm}{\raggedright  %action 
}&\parbox[t]{4cm}{\raggedright  %comment 
}\\

25:42:00 %time
&Alle %name
&\parbox[t]{5cm}{\raggedright  %speech 
}&\parbox[t]{4cm}{\raggedright Leser på oppgavearket %action 
}&\parbox[t]{4cm}{\raggedright  %comment 
}\\

25:52:00 %time
&Siri %name
&\parbox[t]{5cm}{\raggedright Ja...vi har jo også snakket om hva som kan være...årsakene til at...til at...det er så...eller til den forskjellen ((refererer til oppgave 3b)) %speech 
}&\parbox[t]{4cm}{\raggedright  %action 
}&\parbox[t]{4cm}{\raggedright  %comment 
}\\

26:03:00 %time
&Nora %name
&\parbox[t]{5cm}{\raggedright mhm %speech 
}&\parbox[t]{4cm}{\raggedright  %action 
}&\parbox[t]{4cm}{\raggedright  %comment 
}\\

26:15:00 %time
&Fredrik %name
&\parbox[t]{5cm}{\raggedright Hvorfor den flater ut ((refererer til oppgave 3c))...det er jo... %speech 
}&\parbox[t]{4cm}{\raggedright Siri beveger musepekeren over punktene i slutten av grafen, der den flater ut, og ser på bildene som dukekr opp %action 
}&\parbox[t]{4cm}{\raggedright  %comment 
}\\

26:17:00 %time
&Siri %name
&\parbox[t]{5cm}{\raggedright ...kanskje... %speech 
}&\parbox[t]{4cm}{\raggedright  %action 
}&\parbox[t]{4cm}{\raggedright  %comment 
}\\

26:18:00 %time
&Fredrik %name
&\parbox[t]{5cm}{\raggedright ...ja kanskje den begynner på en måte å bli ferdig vokst da? %speech 
}&\parbox[t]{4cm}{\raggedright  %action 
}&\parbox[t]{4cm}{\raggedright  %comment 
}\\

26:22:00 %time
&Siri %name
&\parbox[t]{5cm}{\raggedright ..mhm siden her er den jo veldig høy %speech 
}&\parbox[t]{4cm}{\raggedright  %action 
}&\parbox[t]{4cm}{\raggedright Refererer til bildet på skjermen fra den 11. nov kl 22:24 %comment 
}\\

26:24:00 %time
&Fredrik %name
&\parbox[t]{5cm}{\raggedright mhm %speech 
}&\parbox[t]{4cm}{\raggedright  %action 
}&\parbox[t]{4cm}{\raggedright  %comment 
}\\

26:28:00 %time
&Siri %name
&\parbox[t]{5cm}{\raggedright ...at den ikke kan vokse så veldig mye mer! %speech 
}&\parbox[t]{4cm}{\raggedright  %action 
}&\parbox[t]{4cm}{\raggedright  %comment 
}\\

26:29:00 %time
&Fredrik %name
&\parbox[t]{5cm}{\raggedright mhm %speech 
}&\parbox[t]{4cm}{\raggedright Siri blar seg gjennom punktene i slutten av grafen for å se på bildene %action 
}&\parbox[t]{4cm}{\raggedright  %comment 
}\\

26:33:00 %time
&Siri %name
&\parbox[t]{5cm}{\raggedright Fordi hvis den vokser veldig mye så blir den kanskje så tung at den bøyer seg nedover? %speech 
}&\parbox[t]{4cm}{\raggedright  %action 
}&\parbox[t]{4cm}{\raggedright  %comment 
}\\

26:40:00 %time
&Nora %name
&\parbox[t]{5cm}{\raggedright Den ((kurven)) flater seg ut der også? ((refererer til graf)) %speech 
}&\parbox[t]{4cm}{\raggedright Nora holder musepekeren over ett tidligere punkt i grafen, hvor de fleste punktene også ligger rundt samme y-verdi %action 
}&\parbox[t]{4cm}{\raggedright  %comment 
}\\

26:42:00 %time
&Siri %name
&\parbox[t]{5cm}{\raggedright ...ja men det er fordi at det er så tørt tror jeg? %speech 
}&\parbox[t]{4cm}{\raggedright  %action 
}&\parbox[t]{4cm}{\raggedright Kan mene at y-verdiene er like fordi vannet fordamper senere når det er lite %comment 
}\\

26:45:00 %time
&Siri %name
&\parbox[t]{5cm}{\raggedright Men her så flater den seg liksom...her går den mye jevnere ned %speech 
}&\parbox[t]{4cm}{\raggedright Viser med mus på skjerm %action 
}&\parbox[t]{4cm}{\raggedright Refererer til forskjellen mellom absorbasjonsraten i de to siste vanningene %comment 
}\\

26:49:00 %time
&Siri %name
&\parbox[t]{5cm}{\raggedright her går den ((kurven)) veldig brått ned sånn at den ((planten)) bruker veldig mye vann veldig fort  %speech 
}&\parbox[t]{4cm}{\raggedright Holder musepeker over nest siste vanning i grafen %action 
}&\parbox[t]{4cm}{\raggedright  %comment 
}\\

26:55:00 %time
&Siri %name
&\parbox[t]{5cm}{\raggedright her vokser den sikkert bare litt kanskje?  %speech 
}&\parbox[t]{4cm}{\raggedright Holder musepeker over siste vanning i grafen %action 
}&\parbox[t]{4cm}{\raggedright Kan mene at planten vokser mindre når mindre vann absorberes %comment 
}\\

Går videre til oppgave 3d %time
& %name
&\parbox[t]{5cm}{\raggedright  %speech 
}&\parbox[t]{4cm}{\raggedright  %action 
}&\parbox[t]{4cm}{\raggedright  %comment 
}\\

27:00:00 %time
&Alle %name
&\parbox[t]{5cm}{\raggedright  %speech 
}&\parbox[t]{4cm}{\raggedright leser på oppgavearket %action 
}&\parbox[t]{4cm}{\raggedright  %comment 
}\\

27:12:00 %time
&Siri %name
&\parbox[t]{5cm}{\raggedright Det var jo det med fotosyntese...det var jo det vi snakket om at hvis den ikke bruker fotosyntese så kan det hende at den bruker veldig mye vann veldig fort ((peker på nest siste vanning i grafen med mus på skjerm))...mens her ((peker på aller første vanning på planten i vinduet i graf)) så bruker den bare...eller bruker ganske jevnt fordi den hele tiden har fotosyntese? %speech 
}&\parbox[t]{4cm}{\raggedright Beveger musepeker frem og tilbake i jordfuktighetsgrafen %action 
}&\parbox[t]{4cm}{\raggedright Svarer på oppgave 3d, men svaret passer bedre på 3b %comment 
}\\

27:36:00 %time
&Siri %name
&\parbox[t]{5cm}{\raggedright Ja og at vi kanskje tenkte at den andre planten ((planten i skapet)) må bruke mer vann? %speech 
}&\parbox[t]{4cm}{\raggedright  %action 
}&\parbox[t]{4cm}{\raggedright  %comment 
}\\

27:40:00 %time
&Alle %name
&\parbox[t]{5cm}{\raggedright  %speech 
}&\parbox[t]{4cm}{\raggedright Nikker %action 
}&\parbox[t]{4cm}{\raggedright  %comment 
}\\

27:46:00 %time
&Nora %name
&\parbox[t]{5cm}{\raggedright jepp %speech 
}&\parbox[t]{4cm}{\raggedright  %action 
}&\parbox[t]{4cm}{\raggedright  %comment 
}\\

27:47:00 %time
&Siri %name
&\parbox[t]{5cm}{\raggedright Er det noe annet vi kan si på det der? ((refererer til oppgave 3d)) %speech 
}&\parbox[t]{4cm}{\raggedright  %action 
}&\parbox[t]{4cm}{\raggedright  %comment 
}\\

Intervensjon for å få dem til å snakke om fotosyntese %time
& %name
&\parbox[t]{5cm}{\raggedright  %speech 
}&\parbox[t]{4cm}{\raggedright  %action 
}&\parbox[t]{4cm}{\raggedright  %comment 
}\\

27:51:00 %time
&Sjur %name
&\parbox[t]{5cm}{\raggedright Men hva er det en trenger for å ha fotosyntese...hvilke faktorer? %speech 
}&\parbox[t]{4cm}{\raggedright alle ser mot Sjur %action 
}&\parbox[t]{4cm}{\raggedright  %comment 
}\\

27:56:00 %time
&Nora %name
&\parbox[t]{5cm}{\raggedright vann %speech 
}&\parbox[t]{4cm}{\raggedright  %action 
}&\parbox[t]{4cm}{\raggedright  %comment 
}\\

27:57:00 %time
&Siri %name
&\parbox[t]{5cm}{\raggedright lys %speech 
}&\parbox[t]{4cm}{\raggedright tar i oppgavearket og ser på det. %action 
}&\parbox[t]{4cm}{\raggedright  %comment 
}\\

28:00:00 %time
&Linda og Nora %name
&\parbox[t]{5cm}{\raggedright og co2 %speech 
}&\parbox[t]{4cm}{\raggedright  %action 
}&\parbox[t]{4cm}{\raggedright  %comment 
}\\

28:04:00 %time
&Siri %name
&\parbox[t]{5cm}{\raggedright det er jo co2 både i skapet og i vinduet %speech 
}&\parbox[t]{4cm}{\raggedright  %action 
}&\parbox[t]{4cm}{\raggedright  %comment 
}\\

Intervensjon for å få elevene til å stille spørsmål med at de har sagt om at planten i skapet trenger mer vann fordi den har mindre lys. %time
& %name
&\parbox[t]{5cm}{\raggedright  %speech 
}&\parbox[t]{4cm}{\raggedright  %action 
}&\parbox[t]{4cm}{\raggedright  %comment 
}\\

28:09:00 %time
&Sjur %name
&\parbox[t]{5cm}{\raggedright ...men er det slik at hvis en mangler lys så kan en ta og kompensere med mer vann? %speech 
}&\parbox[t]{4cm}{\raggedright  %action 
}&\parbox[t]{4cm}{\raggedright  %comment 
}\\

28:14:00 %time
&alle %name
&\parbox[t]{5cm}{\raggedright nei %speech 
}&\parbox[t]{4cm}{\raggedright  %action 
}&\parbox[t]{4cm}{\raggedright  %comment 
}\\

28:16:00 %time
&Siri %name
&\parbox[t]{5cm}{\raggedright ...ikke for å bruke fotosyntesen hvertfall %speech 
}&\parbox[t]{4cm}{\raggedright  %action 
}&\parbox[t]{4cm}{\raggedright  %comment 
}\\

28:22:00 %time
&Siri %name
&\parbox[t]{5cm}{\raggedright ...men det virker jo som om den planten i skapet har brukt mye mer vann...siden når den har blitt vannet så har det gått veldig bratt ned ((refererer til jordfuktighetsgraf))...sånn at veldig mye av vannfuktigheten i jorda har blitt borte veldig fort! %speech 
}&\parbox[t]{4cm}{\raggedright  %action 
}&\parbox[t]{4cm}{\raggedright  %comment 
}\\

28:45:00 %time
&Sjur %name
&\parbox[t]{5cm}{\raggedright men når han ((planten)) bruker mindre vann mot slutten av perioden...enn tidligere...hvilken innvirkning tror dere det hadde på fotosyntesen? eller hvordan henger det sammen med fotosyntesen? %speech 
}&\parbox[t]{4cm}{\raggedright  %action 
}&\parbox[t]{4cm}{\raggedright  %comment 
}\\

Intervensjon for å få elevene til å se en relasjon mellom lite vannbruk og fotosyntese %time
& %name
&\parbox[t]{5cm}{\raggedright  %speech 
}&\parbox[t]{4cm}{\raggedright  %action 
}&\parbox[t]{4cm}{\raggedright  %comment 
}\\

29:00:00 %time
&Siri %name
&\parbox[t]{5cm}{\raggedright hmm...kanskje den begynner å...jeg vet ikke...kanskje den bruker fotosyntesen mer da på en måte...eller... %speech 
}&\parbox[t]{4cm}{\raggedright virker veldig usikker, fikler med hendene og har et skeptisk ansiktsutrykk %action 
}&\parbox[t]{4cm}{\raggedright  %comment 
}\\

29:10:00 %time
&Nora %name
&\parbox[t]{5cm}{\raggedright ...ja men da...da...slutter...hvis den ikke har behov for vann så driver den jo ikke fotosyntese %speech 
}&\parbox[t]{4cm}{\raggedright  %action 
}&\parbox[t]{4cm}{\raggedright  %comment 
}\\

Begynner å diskutere lysavhengig og lysuavhengig reaksjon %time
& %name
&\parbox[t]{5cm}{\raggedright  %speech 
}&\parbox[t]{4cm}{\raggedright  %action 
}&\parbox[t]{4cm}{\raggedright  %comment 
}\\

29:16:00 %time
&Nora %name
&\parbox[t]{5cm}{\raggedright men det er sånn...fordi vi har jo...det er jo den lysuavhengige delen av fotosyntesen også...jeg vet ikke om den har...atp og nadph fra f... %speech 
}&\parbox[t]{4cm}{\raggedright ser mot Sjur mens hun snakker, vender seg mot Fredrik når han avbryter henne %action 
}&\parbox[t]{4cm}{\raggedright  %comment 
}\\

29:26:00 %time
&Fredrik %name
&\parbox[t]{5cm}{\raggedright ...den må jo ha den...først drive den lys... eller den må jo drive den lysavhengige også for å drive den lysuavhengige %speech 
}&\parbox[t]{4cm}{\raggedright bruker hendene til å vise at den lysuavhengige reaksjonen er avhengig av den lysavhengige reaksjonen %action 
}&\parbox[t]{4cm}{\raggedright  %comment 
}\\

29:35:00 %time
&Siri %name
&\parbox[t]{5cm}{\raggedright mhm %speech 
}&\parbox[t]{4cm}{\raggedright  %action 
}&\parbox[t]{4cm}{\raggedright  %comment 
}\\

29:36:00 %time
&Fredrik %name
&\parbox[t]{5cm}{\raggedright ...den har vel ikke atp eller nadph fra før av? %speech 
}&\parbox[t]{4cm}{\raggedright alle ler %action 
}&\parbox[t]{4cm}{\raggedright her er det interessant, Det kan hende at alle ler fordi de har kommet til et ytterpunkt av forståelsen sin og blir usikker. Evt kan det være at de ler siden de alle vet at de burde kunne dette siden det er pensum. %comment 
}\\

29:44:00 %time
&Nora %name
&\parbox[t]{5cm}{\raggedright ja det var det jeg lurte på også %speech 
}&\parbox[t]{4cm}{\raggedright  %action 
}&\parbox[t]{4cm}{\raggedright . %comment 
}\\

29:46:00 %time
&Siri %name
&\parbox[t]{5cm}{\raggedright nei det er vel den lysavhengige reaksjonen bruker til å danne det? %speech 
}&\parbox[t]{4cm}{\raggedright  %action 
}&\parbox[t]{4cm}{\raggedright  %comment 
}\\

Hopper tilbake til å diskutere hvorfor grafen flater ut (3c) %time
& %name
&\parbox[t]{5cm}{\raggedright  %speech 
}&\parbox[t]{4cm}{\raggedright  %action 
}&\parbox[t]{4cm}{\raggedright  %comment 
}\\

29:56:00 %time
&Siri %name
&\parbox[t]{5cm}{\raggedright men mot slutten... %speech 
}&\parbox[t]{4cm}{\raggedright  %action 
}&\parbox[t]{4cm}{\raggedright  %comment 
}\\

29:57:00 %time
&Nora %name
&\parbox[t]{5cm}{\raggedright ...men den hadde jo faktisk litt lys, det ble vi jo... %speech 
}&\parbox[t]{4cm}{\raggedright peker mot spektrometer-bildet %action 
}&\parbox[t]{4cm}{\raggedright  %comment 
}\\

29:59:00 %time
&Fredrik %name
&\parbox[t]{5cm}{\raggedright ja %speech 
}&\parbox[t]{4cm}{\raggedright  %action 
}&\parbox[t]{4cm}{\raggedright  %comment 
}\\

30:01:00 %time
&Siri %name
&\parbox[t]{5cm}{\raggedright så den kan ihvertfall utnytte litt mer av det enn hvis det bare hadde vært det grønne lyset...som den utnytter veldig lite av...men mot slutten da vokste den jo mindre, og da kan det hende at den etterhvert...jeg vet ikke hvor lenge en karse varer holdt jeg på å si, men det kan jo hende at den etterhvert visner %speech 
}&\parbox[t]{4cm}{\raggedright alle ler når hun sier at hun ikke vet hvor lenge en karse varer. %action 
}&\parbox[t]{4cm}{\raggedright  %comment 
}\\

30:29:00 %time
&Siri %name
&\parbox[t]{5cm}{\raggedright ...og da vil, bruker den jo ikke så mye vann lenger! %speech 
}&\parbox[t]{4cm}{\raggedright  %action 
}&\parbox[t]{4cm}{\raggedright  %comment 
}\\

30:33:00 %time
&Fredrik %name
&\parbox[t]{5cm}{\raggedright men det er vel den her %speech 
}&\parbox[t]{4cm}{\raggedright Peker på fysisk plante som står på pulten %action 
}&\parbox[t]{4cm}{\raggedright  %comment 
}\\

30:36:00 %time
&Fredrik %name
&\parbox[t]{5cm}{\raggedright og det ser jo ikke ut som om den har visna helt... %speech 
}&\parbox[t]{4cm}{\raggedright Alle bortsett fra Siri ler. %action 
}&\parbox[t]{4cm}{\raggedright  %comment 
}\\

30:39:00 %time
&Siri %name
&\parbox[t]{5cm}{\raggedright ...men den har bøyd seg litt nedover her da %speech 
}&\parbox[t]{4cm}{\raggedright Tar på stilkene til planten, og snur potten rundt for å se bedre %action 
}&\parbox[t]{4cm}{\raggedright  %comment 
}\\

30:41:00 %time
&Fredrik %name
&\parbox[t]{5cm}{\raggedright ja...det er sant %speech 
}&\parbox[t]{4cm}{\raggedright Nora tar på stilkene %action 
}&\parbox[t]{4cm}{\raggedright  %comment 
}\\

30:42:00 %time
&Siri %name
&\parbox[t]{5cm}{\raggedright den var jo helt...også er den veldig myk i bladene på en måte, den pleier, de pleier å være litt fastere i bladene, den er litt sånn slapp (( rister på planten for å vise)) ... %speech 
}&\parbox[t]{4cm}{\raggedright Tar på bladene og stilkene  %action 
}&\parbox[t]{4cm}{\raggedright  %comment 
}\\

30:51:00 %time
&Fredrik %name
&\parbox[t]{5cm}{\raggedright mhm %speech 
}&\parbox[t]{4cm}{\raggedright  %action 
}&\parbox[t]{4cm}{\raggedright  %comment 
}\\

Intervensjon for å få elevene til å komme med teorier om hvorfor planten er slapp %time
& %name
&\parbox[t]{5cm}{\raggedright  %speech 
}&\parbox[t]{4cm}{\raggedright  %action 
}&\parbox[t]{4cm}{\raggedright  %comment 
}\\

30:55:00 %time
&Morten %name
&\parbox[t]{5cm}{\raggedright hvorfor tror du den er sånn? %speech 
}&\parbox[t]{4cm}{\raggedright alle ser mot Morten %action 
}&\parbox[t]{4cm}{\raggedright  %comment 
}\\

30:59:00 %time
&Nora %name
&\parbox[t]{5cm}{\raggedright den mangler ett eller annet %speech 
}&\parbox[t]{4cm}{\raggedright Fredrik stikker en finger i jorden %action 
}&\parbox[t]{4cm}{\raggedright  %comment 
}\\

31:02:00 %time
&Siri %name
&\parbox[t]{5cm}{\raggedright er den tørr? %speech 
}&\parbox[t]{4cm}{\raggedright  %action 
}&\parbox[t]{4cm}{\raggedright  %comment 
}\\

31:04:00 %time
&Fredrik %name
&\parbox[t]{5cm}{\raggedright den er ikke særlig tørr...littegrann men ikke... %speech 
}&\parbox[t]{4cm}{\raggedright ler av å ha fått jord på handa, børster det av på gulvet %action 
}&\parbox[t]{4cm}{\raggedright  %comment 
}\\

31:08:00 %time
&Siri %name
&\parbox[t]{5cm}{\raggedright ...kanskje det har med atte den begynner å visne etterhvert eller at kanskje når den har så lange stilker så blir det vanskelig for den å holde det oppe %speech 
}&\parbox[t]{4cm}{\raggedright  %action 
}&\parbox[t]{4cm}{\raggedright  %comment 
}\\

31:23:00 %time
&Nora %name
&\parbox[t]{5cm}{\raggedright ***...sto den denne veien her?  %speech 
}&\parbox[t]{4cm}{\raggedright holder på planten %action 
}&\parbox[t]{4cm}{\raggedright  %comment 
}\\

31:26:00 %time
&Nora %name
&\parbox[t]{5cm}{\raggedright så den har strukket seg etter lyset som kom fra vinduet %speech 
}&\parbox[t]{4cm}{\raggedright peker mot vinduet %action 
}&\parbox[t]{4cm}{\raggedright  %comment 
}\\

31:30:00 %time
&Sjur %name
&\parbox[t]{5cm}{\raggedright den har stått i skapet %speech 
}&\parbox[t]{4cm}{\raggedright  %action 
}&\parbox[t]{4cm}{\raggedright  %comment 
}\\

31:31:00 %time
&Morten %name
&\parbox[t]{5cm}{\raggedright den har stått i skapet helt fram til nå %speech 
}&\parbox[t]{4cm}{\raggedright  %action 
}&\parbox[t]{4cm}{\raggedright  %comment 
}\\

31:33:00 %time
&Sjur %name
&\parbox[t]{5cm}{\raggedright men lyset sto kanskje sånn %speech 
}&\parbox[t]{4cm}{\raggedright peker ned mot siden av planten for å illustrere at lyskilden var litt til høyre for planten %action 
}&\parbox[t]{4cm}{\raggedright  %comment 
}\\

31:39:00 %time
&Siri %name
&\parbox[t]{5cm}{\raggedright var den sånn når dere tok den ut av skapet? %speech 
}&\parbox[t]{4cm}{\raggedright peker på plante, alle ser mot Sjur %action 
}&\parbox[t]{4cm}{\raggedright  %comment 
}\\

31:40:00 %time
&Sjur %name
&\parbox[t]{5cm}{\raggedright  %speech 
}&\parbox[t]{4cm}{\raggedright nikker %action 
}&\parbox[t]{4cm}{\raggedright  %comment 
}\\

31:42:00 %time
&Siri %name
&\parbox[t]{5cm}{\raggedright okei, da virker det som om den...begynner å visne eller noe sånt.  %speech 
}&\parbox[t]{4cm}{\raggedright  %action 
}&\parbox[t]{4cm}{\raggedright  %comment 
}\\

31:48:00 %time
&Siri %name
&\parbox[t]{5cm}{\raggedright noe av bladene er jo litt brune! %speech 
}&\parbox[t]{4cm}{\raggedright tar på plante.  %action 
}&\parbox[t]{4cm}{\raggedright  %comment 
}\\

31:51:00 %time
&Nora %name
&\parbox[t]{5cm}{\raggedright jeg tror det er frøet, er det ikke? skallet på frøet? %speech 
}&\parbox[t]{4cm}{\raggedright tar på bladene %action 
}&\parbox[t]{4cm}{\raggedright  %comment 
}\\

31:56:00 %time
&Siri %name
&\parbox[t]{5cm}{\raggedright åja! kanskje det %speech 
}&\parbox[t]{4cm}{\raggedright  %action 
}&\parbox[t]{4cm}{\raggedright  %comment 
}\\

31:57:00 %time
&Morten %name
&\parbox[t]{5cm}{\raggedright det der er jo siste bilde, det er litt dårlig kvalitet men... %speech 
}&\parbox[t]{4cm}{\raggedright klikker seg ut av grafen til hjem-siden og viser siste bilde i systemet %action 
}&\parbox[t]{4cm}{\raggedright  %comment 
}\\

Hypotesetesting, blir planten for lang for å holde seg oppe? %time
& %name
&\parbox[t]{5cm}{\raggedright  %speech 
}&\parbox[t]{4cm}{\raggedright  %action 
}&\parbox[t]{4cm}{\raggedright  %comment 
}\\

32:08:00 %time
&Nora %name
&\parbox[t]{5cm}{\raggedright den har jo tydeligvis bare blitt for lang da, så har den plutselig bare faller ned %speech 
}&\parbox[t]{4cm}{\raggedright tar på plante. Siri starter siste video av planten på skjermen %action 
}&\parbox[t]{4cm}{\raggedright  %comment 
}\\

32:13:00 %time
&Alle %name
&\parbox[t]{5cm}{\raggedright  %speech 
}&\parbox[t]{4cm}{\raggedright ler %action 
}&\parbox[t]{4cm}{\raggedright  %comment 
}\\

32:14:00 %time
&Nora %name
&\parbox[t]{5cm}{\raggedright den rister %speech 
}&\parbox[t]{4cm}{\raggedright rister på hodet %action 
}&\parbox[t]{4cm}{\raggedright Refererer til plantespirene som "rister" i videoen %comment 
}\\

32:15:00 %time
&Siri %name
&\parbox[t]{5cm}{\raggedright ja det ser jo sånn ut ((refererer til video på skjermen))...men vi kan se om...ja der var det en som plutselig falt ned %speech 
}&\parbox[t]{4cm}{\raggedright  %action 
}&\parbox[t]{4cm}{\raggedright Refererer til en plantespire som faller over ende i videoen %comment 
}\\

32:20:00 %time
&Nora %name
&\parbox[t]{5cm}{\raggedright den bare gav etter %speech 
}&\parbox[t]{4cm}{\raggedright Demonstrerer med handa at noe velter, mens hun lager tegneserielyd type PTSSJ! %action 
}&\parbox[t]{4cm}{\raggedright  %comment 
}\\

32:23:00 %time
&Siri %name
&\parbox[t]{5cm}{\raggedright men de holder seg jo ganske... %speech 
}&\parbox[t]{4cm}{\raggedright  %action 
}&\parbox[t]{4cm}{\raggedright  %comment 
}\\

32:23:00 %time
&linda %name
&\parbox[t]{5cm}{\raggedright ...de holder seg sikkert fordi lyset er der enda da? %speech 
}&\parbox[t]{4cm}{\raggedright  %action 
}&\parbox[t]{4cm}{\raggedright  %comment 
}\\

32:27:00 %time
&Nora %name
&\parbox[t]{5cm}{\raggedright Men det lyset er kanskje plassert litt skjevt %speech 
}&\parbox[t]{4cm}{\raggedright peker mot videoen på skjermen, demonstrer me hendene en skjev vinkel. %action 
}&\parbox[t]{4cm}{\raggedright Man kan til dels se ut fra videoen at lyset er plassert over og litt til høyre for potten.  %comment 
}\\

32:29:00 %time
&linda %name
&\parbox[t]{5cm}{\raggedright ja det var plassert litt her %speech 
}&\parbox[t]{4cm}{\raggedright beveger hånden mot planten fra siden. %action 
}&\parbox[t]{4cm}{\raggedright  %comment 
}\\

32:30:00 %time
&Fredrik %name
&\parbox[t]{5cm}{\raggedright ...men det er jo fortsatt lys nå også ikke sant %speech 
}&\parbox[t]{4cm}{\raggedright holder hånden over planten %action 
}&\parbox[t]{4cm}{\raggedright  %comment 
}\\

32:37:00 %time
&Siri %name
&\parbox[t]{5cm}{\raggedright nå var det en annen som falt litt nedover og %speech 
}&\parbox[t]{4cm}{\raggedright  %action 
}&\parbox[t]{4cm}{\raggedright Refererer til en annen plantespire som knekker sammen i videoen %comment 
}\\

32:42:00 %time
&Fredrik %name
&\parbox[t]{5cm}{\raggedright ja nå begynner de å %speech 
}&\parbox[t]{4cm}{\raggedright gestikulerer at spirene lener seg i en retning %action 
}&\parbox[t]{4cm}{\raggedright  %comment 
}\\

32:44:00 %time
&Nora %name
&\parbox[t]{5cm}{\raggedright de blir jo så lange at de faller da %speech 
}&\parbox[t]{4cm}{\raggedright  %action 
}&\parbox[t]{4cm}{\raggedright  %comment 
}\\

32:47:00 %time
&Siri %name
&\parbox[t]{5cm}{\raggedright men de vokste jo ikke så mye fra begynnelsen ((begynnelsen av videoen)) her til da de falt ((refererer til video på skjermen)) %speech 
}&\parbox[t]{4cm}{\raggedright peker på skjermen og viser med musepeker begynnelsen og punktet der spirene falt i videoen.  %action 
}&\parbox[t]{4cm}{\raggedright  %comment 
}\\

32:50:00 %time
&Fredrik %name
&\parbox[t]{5cm}{\raggedright nei %speech 
}&\parbox[t]{4cm}{\raggedright  %action 
}&\parbox[t]{4cm}{\raggedright  %comment 
}\\

32:53:00 %time
&Siri %name
&\parbox[t]{5cm}{\raggedright ...så det virker jo...og disse her grafene er jo nesten helt flate %speech 
}&\parbox[t]{4cm}{\raggedright holder musepeker over grafen som viser de korrensponderende variablene i videoen.  %action 
}&\parbox[t]{4cm}{\raggedright Det at grafene er flate betyr at variablene er konstante. Refererer til at det ikke kan være eksterne faktorer som gjør at spirene faller %comment 
}\\

32:59:00 %time
&Siri %name
&\parbox[t]{5cm}{\raggedright ...men visner den liksom bare av seg selv plutselig? %speech 
}&\parbox[t]{4cm}{\raggedright Virker veldig tvilende til at dette kan være tilfelle, skeptisk ansiktsutrykk og plutselig sies med litt sjelvende og rar stemme. %action 
}&\parbox[t]{4cm}{\raggedright  %comment 
}\\

Vitenskapelig snakk om hvorfor planten ikke klarer å holde seg oppe %time
& %name
&\parbox[t]{5cm}{\raggedright  %speech 
}&\parbox[t]{4cm}{\raggedright  %action 
}&\parbox[t]{4cm}{\raggedright  %comment 
}\\

33:04:00 %time
&Nora %name
&\parbox[t]{5cm}{\raggedright ...nå har den kanskje ikke nok sånn cellulose for å holde seg oppe ... hmm? %speech 
}&\parbox[t]{4cm}{\raggedright holder på de plantene som henger ned fra potten, Sier setningen med veldig overbevisning, stopper opp, nikker spent og bekreftende mot Sjur mens hun lager en "ikke sant?"-lyd %action 
}&\parbox[t]{4cm}{\raggedright  %comment 
}\\

33:11:00 %time
&Alle %name
&\parbox[t]{5cm}{\raggedright  %speech 
}&\parbox[t]{4cm}{\raggedright ler %action 
}&\parbox[t]{4cm}{\raggedright  %comment 
}\\

33:13:00 %time
&Nora %name
&\parbox[t]{5cm}{\raggedright og det lager den jo av glukose...som kommer av fotosyntesen(!) %speech 
}&\parbox[t]{4cm}{\raggedright gestikulerer med hendene for å vise at noe avhenger av noe. %action 
}&\parbox[t]{4cm}{\raggedright  %comment 
}\\

33:22:00 %time
&Siri %name
&\parbox[t]{5cm}{\raggedright og kanskje... %speech 
}&\parbox[t]{4cm}{\raggedright  %action 
}&\parbox[t]{4cm}{\raggedright  %comment 
}\\

33:23:00 %time
&Nora %name
&\parbox[t]{5cm}{\raggedright ...så har den blitt sånn lang også har den ikke nok cellulose! ... Til å stives opp ... %speech 
}&\parbox[t]{4cm}{\raggedright Viser med hendene at noe rettes opp. %action 
}&\parbox[t]{4cm}{\raggedright  %comment 
}\\

33:24:00 %time
&Siri %name
&\parbox[t]{5cm}{\raggedright dere snakket jo om i sted at den kunne få litt cellulose...eller glukose eller noe sånt fra frøet...men da kan det hende at de har brukt opp det! også at de får så lite utbytte av det grønne lyset at den ikke klarer å lage nok glukose til å holde seg oppe %speech 
}&\parbox[t]{4cm}{\raggedright  %action 
}&\parbox[t]{4cm}{\raggedright  %comment 
}\\

33:45:00 %time
&Sjur %name
&\parbox[t]{5cm}{\raggedright hva er det som skjer når den ikke får nok av det grønne lyset? eller at det ((planten)) ikke får nok lys? %speech 
}&\parbox[t]{4cm}{\raggedright Nora blåser luft sakte ut av munnen og virker litt oppgitt over spørsmålet %action 
}&\parbox[t]{4cm}{\raggedright  %comment 
}\\

33:50:00 %time
&Siri %name
&\parbox[t]{5cm}{\raggedright da klarer den ikke å gjennomføre fotosyntesen? %speech 
}&\parbox[t]{4cm}{\raggedright  %action 
}&\parbox[t]{4cm}{\raggedright  %comment 
}\\

33:51:00 %time
&Sjur %name
&\parbox[t]{5cm}{\raggedright fordi %speech 
}&\parbox[t]{4cm}{\raggedright  %action 
}&\parbox[t]{4cm}{\raggedright  %comment 
}\\

33:52:00 %time
&Siri %name
&\parbox[t]{5cm}{\raggedright fordi da blir for få klorofyllmolekyler...atomer...eksitert %speech 
}&\parbox[t]{4cm}{\raggedright  %action 
}&\parbox[t]{4cm}{\raggedright  %comment 
}\\

33:53:00 %time
&Sjur %name
&\parbox[t]{5cm}{\raggedright ja %speech 
}&\parbox[t]{4cm}{\raggedright  %action 
}&\parbox[t]{4cm}{\raggedright  %comment 
}\\

34:05:00 %time
&Nora %name
&\parbox[t]{5cm}{\raggedright ja ... var det det som var svaret hele tiden?  %speech 
}&\parbox[t]{4cm}{\raggedright Rister på skuldrene og ler %action 
}&\parbox[t]{4cm}{\raggedright  %comment 
}\\

Intervensjon for å få elevene til å svare på spm. 3d %time
& %name
&\parbox[t]{5cm}{\raggedright  %speech 
}&\parbox[t]{4cm}{\raggedright  %action 
}&\parbox[t]{4cm}{\raggedright  %comment 
}\\

34:11:00 %time
&Sjur %name
&\parbox[t]{5cm}{\raggedright men kan en bruke vann som ett mål for fotosyntese? ((refererer til spørsmål 3d)) %speech 
}&\parbox[t]{4cm}{\raggedright  %action 
}&\parbox[t]{4cm}{\raggedright  %comment 
}\\

34:13:00 %time
&Siri %name
&\parbox[t]{5cm}{\raggedright istedetfor lyset mener du? %speech 
}&\parbox[t]{4cm}{\raggedright  %action 
}&\parbox[t]{4cm}{\raggedright  %comment 
}\\

34:17:00 %time
&Nora %name
&\parbox[t]{5cm}{\raggedright ...ja for hvis vannet nå...ehm..ble sånn rett igjen, grafen? %speech 
}&\parbox[t]{4cm}{\raggedright blir engasjert, gestikulerer graf som flater ut, ser på spørsmålsarket for å finne det rette ordet (flater ut) %action 
}&\parbox[t]{4cm}{\raggedright  %comment 
}\\

34:23:00 %time
&Siri %name
&\parbox[t]{5cm}{\raggedright ...ja her på grafen %speech 
}&\parbox[t]{4cm}{\raggedright beveger musepekeren langs grafen til den siste videoen %action 
}&\parbox[t]{4cm}{\raggedright  %comment 
}\\

34:26:00 %time
&Siri %name
&\parbox[t]{5cm}{\raggedright lysfuktighet, nei, luftfuktighet står det... jordtemperatur %speech 
}&\parbox[t]{4cm}{\raggedright  %action 
}&\parbox[t]{4cm}{\raggedright Leser opp variabelnavnene fra grafen i systemet %comment 
}\\

34:31:00 %time
&Nora %name
&\parbox[t]{5cm}{\raggedright ...ja, siden grafen flatet ut...og igjen siden behovet for vann ikke...var der...så betyr det at...fotosyntesen går saktere! %speech 
}&\parbox[t]{4cm}{\raggedright ser på Sjur mens hun svarer. %action 
}&\parbox[t]{4cm}{\raggedright  %comment 
}\\

34:42:00 %time
&Siri %name
&\parbox[t]{5cm}{\raggedright ja! %speech 
}&\parbox[t]{4cm}{\raggedright alle nikker %action 
}&\parbox[t]{4cm}{\raggedright  %comment 
}\\

34:43:00 %time
&Nora %name
&\parbox[t]{5cm}{\raggedright at det ikke er like mye fotosyntese %speech 
}&\parbox[t]{4cm}{\raggedright  %action 
}&\parbox[t]{4cm}{\raggedright  %comment 
}\\

34:45:00 %time
&Sjur %name
&\parbox[t]{5cm}{\raggedright det er en god teori %speech 
}&\parbox[t]{4cm}{\raggedright  %action 
}&\parbox[t]{4cm}{\raggedright  %comment 
}\\

34:49:00 %time
&Nora %name
&\parbox[t]{5cm}{\raggedright takk %speech 
}&\parbox[t]{4cm}{\raggedright alle ler %action 
}&\parbox[t]{4cm}{\raggedright  %comment 
}\\

34:53:00 %time
&Nora %name
&\parbox[t]{5cm}{\raggedright så ja, det kan nok brukes som mål på raten av fotosyntese! %speech 
}&\parbox[t]{4cm}{\raggedright alle ser på oppgavearket. %action 
}&\parbox[t]{4cm}{\raggedright  %comment 
}\\

34:56:00 %time
&Siri %name
&\parbox[t]{5cm}{\raggedright ja %speech 
}&\parbox[t]{4cm}{\raggedright  %action 
}&\parbox[t]{4cm}{\raggedright  %comment 
}\\

34:57:00 %time
&Nora %name
&\parbox[t]{5cm}{\raggedright er vi enige? %speech 
}&\parbox[t]{4cm}{\raggedright  %action 
}&\parbox[t]{4cm}{\raggedright  %comment 
}\\

34:58:00 %time
&Siri %name
&\parbox[t]{5cm}{\raggedright ja %speech 
}&\parbox[t]{4cm}{\raggedright  %action 
}&\parbox[t]{4cm}{\raggedright  %comment 
}\\

34:58:00 %time
&Nora %name
&\parbox[t]{5cm}{\raggedright ja %speech 
}&\parbox[t]{4cm}{\raggedright  %action 
}&\parbox[t]{4cm}{\raggedright  %comment 
}\\

35:00:00 %time
&Fredrik %name
&\parbox[t]{5cm}{\raggedright helt enig %speech 
}&\parbox[t]{4cm}{\raggedright alle humrer %action 
}&\parbox[t]{4cm}{\raggedright  %comment 
}\\

35:02:00 %time
&Siri %name
&\parbox[t]{5cm}{\raggedright vekstraten ((refererer til oppgave 4)) %speech 
}&\parbox[t]{4cm}{\raggedright  %action 
}&\parbox[t]{4cm}{\raggedright  %comment 
}\\

35:03:00 %time
&Nora %name
&\parbox[t]{5cm}{\raggedright har den vekstrate også ((refererer til system)) %speech 
}&\parbox[t]{4cm}{\raggedright  %action 
}&\parbox[t]{4cm}{\raggedright  %comment 
}\\

35:06:00 %time
&Siri %name
&\parbox[t]{5cm}{\raggedright hvor er den henne da? %speech 
}&\parbox[t]{4cm}{\raggedright  %action 
}&\parbox[t]{4cm}{\raggedright  %comment 
}\\

35:09:00 %time
&Sjur %name
&\parbox[t]{5cm}{\raggedright jeg tror vi kan slutte av %speech 
}&\parbox[t]{4cm}{\raggedright  %action 
}&\parbox[t]{4cm}{\raggedright tidsbegrensning. Lærer vil ha tid til oppsummering %comment 
}\\

De diskuterer at de har kommet seg gjennom hele arket uten å ha sett på alle spørsmålene %time
& %name
&\parbox[t]{5cm}{\raggedright  %speech 
}&\parbox[t]{4cm}{\raggedright  %action 
}&\parbox[t]{4cm}{\raggedright  %comment 
}\\

35:18:00 %time
&Nora %name
&\parbox[t]{5cm}{\raggedright oi(!) der er jo bladenes utseende og sånt ((refererer til oppgave 4 c)) %speech 
}&\parbox[t]{4cm}{\raggedright peker på oppgavearket. %action 
}&\parbox[t]{4cm}{\raggedright  %comment 
}\\

35:21:00 %time
&Siri %name
&\parbox[t]{5cm}{\raggedright hæ! hva da? "er det noen ulikeheter i bladenes utseende" ((leser opp oppgave 4c)). Ja, også har vi jo snakket om hvor mye de har vokst og sånn ((refererer til oppgave 5c)) %speech 
}&\parbox[t]{4cm}{\raggedright  %action 
}&\parbox[t]{4cm}{\raggedright  %comment 
}\\

35:27:00 %time
&Fredrik %name
&\parbox[t]{5cm}{\raggedright ja... uten å se på spørsmålet har vi jo på en måte %speech 
}&\parbox[t]{4cm}{\raggedright  %action 
}&\parbox[t]{4cm}{\raggedright  %comment 
}\\

35:28:00 %time
&Siri %name
&\parbox[t]{5cm}{\raggedright ...har vi jo egentlig svart på alt %speech 
}&\parbox[t]{4cm}{\raggedright  %action 
}&\parbox[t]{4cm}{\raggedright  %comment 
}\\

35:35:00 %time
&Nora %name
&\parbox[t]{5cm}{\raggedright ...filosofert litt %speech 
}&\parbox[t]{4cm}{\raggedright  %action 
}&\parbox[t]{4cm}{\raggedright  %comment 
}\\

35:45:00 %time
&Sjur %name
&\parbox[t]{5cm}{\raggedright Da er timen nesten ferdig %speech 
}&\parbox[t]{4cm}{\raggedright snakker mot klasse %action 
}&\parbox[t]{4cm}{\raggedright  %comment 
}\\

35:50:00 %time
&Siri %name
&\parbox[t]{5cm}{\raggedright oi(!) det gikk fort...tiden går fort når man snakker om fotosyntese %speech 
}&\parbox[t]{4cm}{\raggedright Siri og Nora ler %action 
}&\parbox[t]{4cm}{\raggedright  %comment 
}\\

24:55:00 %time
& %name
&\parbox[t]{5cm}{\raggedright  %speech 
}&\parbox[t]{4cm}{\raggedright  %action 
}&\parbox[t]{4cm}{\raggedright  %comment 
}\\

går over i klassediskusjon. kun lyd %time
& %name
&\parbox[t]{5cm}{\raggedright  %speech 
}&\parbox[t]{4cm}{\raggedright  %action 
}&\parbox[t]{4cm}{\raggedright  %comment 
}\\

36:06:00 %time
&Lærer %name
&\parbox[t]{5cm}{\raggedright vi kan vel si at dette forsøket var et...jeg synes det ble ett mye mere interessant forsøk enn det jeg hadde tenkt meg på forhånd. Fordi det er mange ting som det ikke er noe fasitsvar på her. Vi har rett og slett ikke kunnet måle tilstrekkelig mange...altså sammenligne tilstrekkelig mange ting. Sånn at for eksempel den planten som sto i skapet...har den fotosyntese i det hele tatt? vi vet ikke en gang det. Flere av dere tok utgangspunkt i det at den hadde det. Men det kunne vært artig å høre forskjellige...hva dere svarte på noen av disse spørsmålene før vi tar pause...for det kanskje dere har forskjellige svar...skal vi gjøre det? bare høre litt på noe av de...jeg vet ikke. er det noen av disse som er ekstra interessante å...for eksempel 1c ((oppgave))... oppgave 1c der, hva svarte dere "hvis det var noen forskjeller i resultatene, hva kan årsaken ha vært?" ((leser fra oppgaveark)) %speech 
}&\parbox[t]{4cm}{\raggedright  %action 
}&\parbox[t]{4cm}{\raggedright  %comment 
}\\

37:09:00 %time
&Lærer %name
&\parbox[t]{5cm}{\raggedright hva var hovedforskjellen forresten? på de som sto der ((vindu)) og de som sto der ((skap))? Det kan vi høre med dere for eksempel ((henvender seg til gruppe 1)) %speech 
}&\parbox[t]{4cm}{\raggedright  %action 
}&\parbox[t]{4cm}{\raggedright  %comment 
}\\

37:14:00 %time
&Gruppe 1 %name
&\parbox[t]{5cm}{\raggedright at de som står der ((vindu)) har mye mer heliotropisme...sånn at de beveger seg med lyset i løpet av dagen %speech 
}&\parbox[t]{4cm}{\raggedright  %action 
}&\parbox[t]{4cm}{\raggedright  %comment 
}\\

37:19:00 %time
&Lærer %name
&\parbox[t]{5cm}{\raggedright ja %speech 
}&\parbox[t]{4cm}{\raggedright  %action 
}&\parbox[t]{4cm}{\raggedright  %comment 
}\\

37:20:00 %time
&Gruppe 1 %name
&\parbox[t]{5cm}{\raggedright mens de inni der ((skapet)) *** der er jo lyset fastmontert så det er jo på en måte ikke noe...heliotropisme...for det er jo på en måte... %speech 
}&\parbox[t]{4cm}{\raggedright  %action 
}&\parbox[t]{4cm}{\raggedright  %comment 
}\\

37:26:00 %time
&Gruppe 1 %name
&\parbox[t]{5cm}{\raggedright ...Og de blir høyere %speech 
}&\parbox[t]{4cm}{\raggedright  %action 
}&\parbox[t]{4cm}{\raggedright  %comment 
}\\

37:31:00 %time
&Lærer %name
&\parbox[t]{5cm}{\raggedright De blir høyere! og hvordan kan man tolke det? Dere tolket det på en måte ((henvender seg til gruppe 2)) og dere tolket det på en annen måte ((henvender seg til gruppe 4)). Få høre dere først ((henvender seg til gruppe 4)) %speech 
}&\parbox[t]{4cm}{\raggedright  %action 
}&\parbox[t]{4cm}{\raggedright  %comment 
}\\

37:39:00 %time
&Siri %name
&\parbox[t]{5cm}{\raggedright ...ehm...vi sa så mye...eeh %speech 
}&\parbox[t]{4cm}{\raggedright  %action 
}&\parbox[t]{4cm}{\raggedright  %comment 
}\\

37:43:00 %time
&Nora %name
&\parbox[t]{5cm}{\raggedright Hva var spørmålet? om hvorfor den... %speech 
}&\parbox[t]{4cm}{\raggedright  %action 
}&\parbox[t]{4cm}{\raggedright  %comment 
}\\

37:46:00 %time
&Lærer %name
&\parbox[t]{5cm}{\raggedright hvorfor blir de plantene høyere enn de der? ((refererer til plante i skap i forhold til plante i vindu))...For det var det jo helt åpenbart at de ble %speech 
}&\parbox[t]{4cm}{\raggedright  %action 
}&\parbox[t]{4cm}{\raggedright  %comment 
}\\

37:53:00 %time
&Nora %name
&\parbox[t]{5cm}{\raggedright vi hadde ikke sånn klart svar på det, men %speech 
}&\parbox[t]{4cm}{\raggedright  %action 
}&\parbox[t]{4cm}{\raggedright  %comment 
}\\

37:55:00 %time
&Siri %name
&\parbox[t]{5cm}{\raggedright ...dere sa det om frøet %speech 
}&\parbox[t]{4cm}{\raggedright  %action 
}&\parbox[t]{4cm}{\raggedright  %comment 
}\\

37:56:00 %time
&Nora %name
&\parbox[t]{5cm}{\raggedright ja! det fikk energi fra frøet %speech 
}&\parbox[t]{4cm}{\raggedright  %action 
}&\parbox[t]{4cm}{\raggedright  %comment 
}\\

37:57:00 %time
&Lærer %name
&\parbox[t]{5cm}{\raggedright de fikk? %speech 
}&\parbox[t]{4cm}{\raggedright  %action 
}&\parbox[t]{4cm}{\raggedright  %comment 
}\\

37:58:00 %time
&Nora %name
&\parbox[t]{5cm}{\raggedright energi fra frøet %speech 
}&\parbox[t]{4cm}{\raggedright  %action 
}&\parbox[t]{4cm}{\raggedright  %comment 
}\\

38:01:00 %time
&Lærer  %name
&\parbox[t]{5cm}{\raggedright okei!  %speech 
}&\parbox[t]{4cm}{\raggedright  %action 
}&\parbox[t]{4cm}{\raggedright  %comment 
}\\

38:03:00 %time
&Fredrik %name
&\parbox[t]{5cm}{\raggedright men de andre hadde jo også frø med energi...så...men det var jo ikke %speech 
}&\parbox[t]{4cm}{\raggedright  %action 
}&\parbox[t]{4cm}{\raggedright  %comment 
}\\

38:10:00 %time
&Siri %name
&\parbox[t]{5cm}{\raggedright ...men det vi tenkte på var atte siden det var så lite lys inni skapet... åja(!) nei...ja(!) siden det var så lite lys inni skapet så kunne ikke de... enten så kunne de kanskje ikke utføre fotosyntesen, eller utføre den veldig dårlig. Så da måtte kanskje planten finne en annen måte å vokse på. Og da kan det hende at den klarte å utnytte...vannet og næringssaltene i jorda og næringen i frøet? %speech 
}&\parbox[t]{4cm}{\raggedright  %action 
}&\parbox[t]{4cm}{\raggedright  %comment 
}\\

38:48:00 %time
&Lærer %name
&\parbox[t]{5cm}{\raggedright okei. Hva sa dere om dette? ((henvender seg til gruppe 2)) %speech 
}&\parbox[t]{4cm}{\raggedright  %action 
}&\parbox[t]{4cm}{\raggedright  %comment 
}\\

38:50:00 %time
&Gruppe 2 %name
&\parbox[t]{5cm}{\raggedright vi sa at den kanskje strakk seg for å...finne mere lys %speech 
}&\parbox[t]{4cm}{\raggedright  %action 
}&\parbox[t]{4cm}{\raggedright  %comment 
}\\

39:02:00 %time
&Lærer %name
&\parbox[t]{5cm}{\raggedright at den der inne ((skapet)) strakk seg fordi den ikke var fornøyd med lyset... så den strakk seg videre og videre og videre? Kan det være en god forklaring? det hadde kanskje ikke dere tenkt på heller? ((henvender seg til Morten og Sjur)) %speech 
}&\parbox[t]{4cm}{\raggedright  %action 
}&\parbox[t]{4cm}{\raggedright  %comment 
}\\

39:03:00 %time
&Lærer %name
&\parbox[t]{5cm}{\raggedright Fordi det er en god forklaring... De av dere som hadde biologi i fjor...hvordan noen planter...altså...husk at...frøene som dere sier har opplagsnæring, de har stivelse. Det kan de komme ganske langt med. Og når frøene spirer så er det om å gjøre å få bladene ut i lyset. Og ett alternativ kan være at disse fant jo ikke noe lys! ((planten i skapet)), så de fortsatte med å strekke seg enda videre for å komme til ett lys. Men en annen forklaring - en helt annen forklaring - kan være at disse ((planten i skapet)) fikk jo ly hele døgnet og derfor skjedde det mer fotosyntese og derfor ble de større. Det går også an å svare det, men da burde man testet det videre. Det som er fint med disse forsøkene her er at man må rett og slett gruble på forskjellige ting og se hva som mangler i forsøket også må man lage nye forsøk videre. Og det er sånne typer oppgaver man får til eksamen nå faktisk: tolk dette! og det kan være mange forskjellige tolkninger.  %speech 
}&\parbox[t]{4cm}{\raggedright  %action 
}&\parbox[t]{4cm}{\raggedright  %comment 
}\\

40:05:00 %time
&Lærer %name
&\parbox[t]{5cm}{\raggedright Også det med jordfuktighet har jeg lyst til...hva sa dere om det...om at den flata ut ((refererer til oppgave 3c))...altså alt det som hadde med jordfuktighet å gjøre. Hva svarte dere på det? ((henvender seg til gruppe 1)) %speech 
}&\parbox[t]{4cm}{\raggedright  %action 
}&\parbox[t]{4cm}{\raggedright  %comment 
}\\

40:13:00 %time
&Gruppe 1  %name
&\parbox[t]{5cm}{\raggedright vi kom ikke til det %speech 
}&\parbox[t]{4cm}{\raggedright  %action 
}&\parbox[t]{4cm}{\raggedright  %comment 
}\\

40:16:00 %time
&Lærer %name
&\parbox[t]{5cm}{\raggedright dere ((henvender seg til gruppe 3)) %speech 
}&\parbox[t]{4cm}{\raggedright  %action 
}&\parbox[t]{4cm}{\raggedright  %comment 
}\\

40:26:00 %time
&Lærer %name
&\parbox[t]{5cm}{\raggedright Det var det med jordfuktighet som er spørsmål tre...etter at man har vannet den starter den med høy jordfuktighet, så gikk den ganske bratt ned, så flata den ut på slutten. Og hvordan kan man tolke det? %speech 
}&\parbox[t]{4cm}{\raggedright  %action 
}&\parbox[t]{4cm}{\raggedright  %comment 
}\\

40:40:00 %time
&Gruppe 3 %name
&\parbox[t]{5cm}{\raggedright ehm...at det var forskjell på absorbasjonen fordi plantene vokste forskjellig. Og...fordi...den flatet ut fordi det var på en måte slutt på veksten. at den vokste mindre %speech 
}&\parbox[t]{4cm}{\raggedright  %action 
}&\parbox[t]{4cm}{\raggedright  %comment 
}\\

41:03:00 %time
&Lærer %name
&\parbox[t]{5cm}{\raggedright Vokste mindre og da tok den opp mindre vann? %speech 
}&\parbox[t]{4cm}{\raggedright  %action 
}&\parbox[t]{4cm}{\raggedright  %comment 
}\\

41:05:00 %time
&Gruppe 3  %name
&\parbox[t]{5cm}{\raggedright ja %speech 
}&\parbox[t]{4cm}{\raggedright  %action 
}&\parbox[t]{4cm}{\raggedright  %comment 
}\\

41:05:00 %time
&Lærer %name
&\parbox[t]{5cm}{\raggedright okei. Svarte dere noe annet ((henvender seg til gruppe 4)) %speech 
}&\parbox[t]{4cm}{\raggedright  %action 
}&\parbox[t]{4cm}{\raggedright  %comment 
}\\

41:09:00 %time
&Siri %name
&\parbox[t]{5cm}{\raggedright vi svarte at når...eller nå tenkte jeg det at...ja, plantene kunne jo utnytte veldig lite av fotosyntesen, og hvis de kunne utnytte litt av fotosyntesen, så gjorde de det veldig mye når de først fikk vann. Fordi da fikk de...da kunne de liksom...utvikle %speech 
}&\parbox[t]{4cm}{\raggedright  %action 
}&\parbox[t]{4cm}{\raggedright  %comment 
}\\

41:35:00 %time
&Lærer %name
&\parbox[t]{5cm}{\raggedright Så du tenkte at fotosyntesen gikk saktere jo mindre vann det var... og dermed ble det også tatt opp mindre vann %speech 
}&\parbox[t]{4cm}{\raggedright  %action 
}&\parbox[t]{4cm}{\raggedright  %comment 
}\\

41:37:00 %time
&Siri %name
&\parbox[t]{5cm}{\raggedright nei.. kanskje ikke %speech 
}&\parbox[t]{4cm}{\raggedright  %action 
}&\parbox[t]{4cm}{\raggedright  %comment 
}\\

41:38:00 %time
&Nora %name
&\parbox[t]{5cm}{\raggedright Når fotosyntesen går saktere er det ikke behov %speech 
}&\parbox[t]{4cm}{\raggedright  %action 
}&\parbox[t]{4cm}{\raggedright  %comment 
}\\

41:44:00 %time
&Lærer %name
&\parbox[t]{5cm}{\raggedright åja(!) sånn ja %speech 
}&\parbox[t]{4cm}{\raggedright  %action 
}&\parbox[t]{4cm}{\raggedright  %comment 
}\\

41:45:00 %time
&Siri %name
&\parbox[t]{5cm}{\raggedright nei, det var helt på slutten når den flata ut...når fotosyntesen gikk saktere på slutten så var det ikke behov for vann fordi da gikk ikke fotosyntesen uansett! %speech 
}&\parbox[t]{4cm}{\raggedright  %action 
}&\parbox[t]{4cm}{\raggedright  %comment 
}\\

41:55:00 %time
&Lærer %name
&\parbox[t]{5cm}{\raggedright spennende, hadde dere noen andre forklaringer ((henvender seg til gruppe 2)) %speech 
}&\parbox[t]{4cm}{\raggedright  %action 
}&\parbox[t]{4cm}{\raggedright  %comment 
}\\

42:00:00 %time
&Gruppe 2 %name
&\parbox[t]{5cm}{\raggedright ja, eller vi var jo ikke sikre da, men vi hadde to forsjellige som vi tenkte på. Og da var det også det at det med at når den først fikk vann. At når den fikk mye vann så brukte den mer også, så kanskje det at den klarte å utnytte det bedre når den fikk mye. Men også det med at det ble eller var ubalanse mellom jorden og luften.  %speech 
}&\parbox[t]{4cm}{\raggedright  %action 
}&\parbox[t]{4cm}{\raggedright  %comment 
}\\

42:25:00 %time
&Lærer %name
&\parbox[t]{5cm}{\raggedright jorden og luften? Hva tenker du på da at? %speech 
}&\parbox[t]{4cm}{\raggedright  %action 
}&\parbox[t]{4cm}{\raggedright  %comment 
}\\

42:31:00 %time
&Gruppe 2 %name
&\parbox[t]{5cm}{\raggedright nei at det fordamper mer vann fra jorden %speech 
}&\parbox[t]{4cm}{\raggedright  %action 
}&\parbox[t]{4cm}{\raggedright  %comment 
}\\

42:36:00 %time
&Lærer %name
&\parbox[t]{5cm}{\raggedright når det er fuktig i jorda? %speech 
}&\parbox[t]{4cm}{\raggedright  %action 
}&\parbox[t]{4cm}{\raggedright  %comment 
}\\

42:37:00 %time
&Gruppe 2 %name
&\parbox[t]{5cm}{\raggedright når det er fuktig i jorda %speech 
}&\parbox[t]{4cm}{\raggedright  %action 
}&\parbox[t]{4cm}{\raggedright  %comment 
}\\

42:39:00 %time
&Lærer %name
&\parbox[t]{5cm}{\raggedright hm! hadde dere tenkt på den ((henvender seg til Sjur)) %speech 
}&\parbox[t]{4cm}{\raggedright  %action 
}&\parbox[t]{4cm}{\raggedright  %comment 
}\\

42:40:00 %time
&Sjur %name
&\parbox[t]{5cm}{\raggedright nei %speech 
}&\parbox[t]{4cm}{\raggedright  %action 
}&\parbox[t]{4cm}{\raggedright  %comment 
}\\

42:45:00 %time
&Lærer %name
&\parbox[t]{5cm}{\raggedright altså rett og slett hvis man har to kar uten noen plantefrø i, i det hele tatt. det burde man jo sjekka isåfall. Kanskje rett og slett den som har høyest luftfuktighet også, nei jordfuktighet også fordamper raskest sånn at kurven...det...det høres jo veldig riktig ut ut fra det lille vi kan om fysikk. Jo mer fuktighet det er ett sted jo fortere fordamper det, jo fortere går fordampningen. Inntil det er kommet lengre ned. Jeg vil si at - det er masse flere ting å diskutere her. Og er det noe flere. Hadde dere noen betraktninger på spørsmålene som ikke er kommet frem her? ((henvender seg til gruppe 1)) %speech 
}&\parbox[t]{4cm}{\raggedright  %action 
}&\parbox[t]{4cm}{\raggedright  %comment 
}\\

43:25:00 %time
&Gruppe 1 %name
&\parbox[t]{5cm}{\raggedright nei, det var bare det at vi var litt usikre på, vi skjønte ikke hvorfor de plantene i skapet vokste så utrolig høyt.  %speech 
}&\parbox[t]{4cm}{\raggedright  %action 
}&\parbox[t]{4cm}{\raggedright  %comment 
}\\

43:33:00 %time
&Lærer %name
&\parbox[t]{5cm}{\raggedright nei. Fikk dere ett svar nå, ett mulig svar? %speech 
}&\parbox[t]{4cm}{\raggedright  %action 
}&\parbox[t]{4cm}{\raggedright  %comment 
}\\

43:35:00 %time
&Gruppe 1 %name
&\parbox[t]{5cm}{\raggedright ja, men vi er fremdeles usikre på hvorfor %speech 
}&\parbox[t]{4cm}{\raggedright  %action 
}&\parbox[t]{4cm}{\raggedright hvordan det relaterer til modellen? %comment 
}\\

43:36:00 %time
&Lærer %name
&\parbox[t]{5cm}{\raggedright at cellene rett og slett strekker seg lengre, hver eneste celle strekker seg lengre fordi det utskilles hormoner for eksempel på grunn av at det er så lite lys? Så den leter etter lys %speech 
}&\parbox[t]{4cm}{\raggedright  %action 
}&\parbox[t]{4cm}{\raggedright  %comment 
}\\

43:49:00 %time
&Gruppe 1 %name
&\parbox[t]{5cm}{\raggedright ja, det kan godt hende %speech 
}&\parbox[t]{4cm}{\raggedright  %action 
}&\parbox[t]{4cm}{\raggedright  %comment 
}\\

43:50:00 %time
&Lærer %name
&\parbox[t]{5cm}{\raggedright det er ihvertfall en teori (emph) %speech 
}&\parbox[t]{4cm}{\raggedright  %action 
}&\parbox[t]{4cm}{\raggedright  %comment 
}\\

43:54:00 %time
&Lærer %name
&\parbox[t]{5cm}{\raggedright Det var det dere kom med tror jeg ((henvender seg til gruppe 3)) eller dere ((gruppe 4)) %speech 
}&\parbox[t]{4cm}{\raggedright  %action 
}&\parbox[t]{4cm}{\raggedright  %comment 
}\\

43:55:00 %time
&Siri %name
&\parbox[t]{5cm}{\raggedright begge deler %speech 
}&\parbox[t]{4cm}{\raggedright  %action 
}&\parbox[t]{4cm}{\raggedright  %comment 
}\\

44:00:00 %time
&Lærer %name
&\parbox[t]{5cm}{\raggedright jeg synes at konklusjonen på dette her er at det er ett typisk åpent forsøk hvor heller ikke de som setter det opp egentlig vet alt. Jeg visste ihvertfall ikke alt, jeg tror ikke dere visste alt ((Sjur og Morten)) hva som kom til å skje og hvorfor. Vi følger noen kurver, også stimulerer vi diskusjonen og det er jo det egentlig forsøk i naturfag skal være bygget opp på den måten. Sette opp hypotese, hva tror du skjer. Hvordan tolker du det som har skjedd. Det er de beste forsøkene egentlig for da må vi tenke og finne ut at jøss dette skjønner vi jo ikke helt, vi må gjøre flere forsøk. Mens alt for mange forsøk i skolen er jo sånn elektrolyse, stikk ned to stenger. hvorfor blir det rødt på den ene og klorgass på den andre. Altså det er liksom så forutsigbart alt sammen. Så sånn sett synes jeg dette var veldig vellykket.  %speech 
}&\parbox[t]{4cm}{\raggedright  %action 
}&\parbox[t]{4cm}{\raggedright  %comment 
}\\

