%!TEX root = ../document.tex
\section{Oppgaver til forsøk - Foss VGS 13.11.2013}

\subsection{Gruppe 1}

\begin{enumerate}
	\item Oppgave 1
	\begin{enumerate}
		\item Fuktigheten i jorden går ned. Kom til å spire - men ikke like avhengig av lyset
		\item I grønt lys ble karsen mye lenger
		\item Strakk seg antakelig høyere for å "se" etter lys
	\end{enumerate}

	\item Oppgave 2
	\begin{enumerate}
		\item Ja, plante 1 beveger seg frem og tilbake, mens plante 2 vokser rett opp.
		\item Plantene beveger seg etter lyset
	\end{enumerate}	

	\item Oppgave 3
	\begin{enumerate}
		\item Ja, planten i grønt lys absorberte vannet mye fortere enn den i sollys
		\item Dette er antakelig fordi det var mange fler planter i plante 2, og at de vokste mye fortere og høyere. 
		\item Kanskje klarer ikke planten å utnytte vannet like godt når det er lite. Eller ubalanse mellom jorden og luften. Fordamp.
		\item Man skulle tro at jo mer vann en plante tar opp, desto mer fotosyntese driver den. Dette vet vi imidlertid ikke.
	\end{enumerate}

	\item Oppgave 4
	\begin{enumerate}
		\item Den i grønt lys vokser raskest
		\item Antakelig fordi den strekker seg etter lys. 
		\item Bladene på planten som står i sollys krummer seg mer en bladene på planten i grønt lys. 
	\end{enumerate}

	\item Oppgave 5
	\begin{enumerate}
		\item Planten vokser mye fortere i starten enn på slutten
		\item Planten står i grønt lys og vil ikke drive fotosyntese. Derfor vil den naturligvis slutte å vokse når næringen i frøene er brukt opp. 
	\end{enumerate}
\end{enumerate}

\subsection{Gruppe 2}
\begin{enumerate}
	\item Oppgave 1
	\begin{enumerate}
		\item I vinduskarmen: vi forventet at plantene skulle vokse godt. I lukket skap: vi trodde at plantene ikke ville vokse eller veldig sakte, og få gule og brune flekker
		\item I vinduskarmen: plantene vokste mot lyset og de bevegde seg etter solas bevegelse i løpet av dagen. I lukket skap: planten vokste godt, de ble lengre enn de i vinduskarmen, de bevegde seg litt, men vi vet ikke helt etter hva.
		\item Vi tror at plantene i skapet vokser høyere fordi de vil strekke seg/er på jakt etter (sol)lys. Vi tror at plantene i skapet beveger seg fordi de er på jakt etter mer (sol)lys. \sout{Vi tror at selv om plantene i skapet bruker lengre tid på å vokse seg høye enn de i vinduskarmen.} Vi tror at plantene i skapet har klart seg så godt fordi de har fått tilgang på lys hele døgnet, mens plantene i skapet (her mener de nok vinduskarmen) hadde ikke tilgang på lys om natta.
	\end{enumerate}

	\item Oppgave 2
	\begin{enumerate}
		\item ja.
		\item I vinduskarmen beveger plantene seg etter solas bevegelse på himmelen, men i skapet bever de seg med slange-bevegelser. Vi tror at plantene i skapet er på jakt etter (sol)lys og derfor bever seg på denne måten.
	\end{enumerate}	

	\item Oppgave 3
	\begin{enumerate}
		\item ja, plantene i skapet absorberer mer vann enn de i vinduskarmen
		\item plantene i skapet vokser fortere derfor trenger de mer vann. I tillegg er det (tror det skal stå flere her) planter i skapet enn i vinduskarmen og derfor absorberer plantene i skapet mer.
		\item -
		\item -
	\end{enumerate}

	\item Oppgave 4
	\begin{enumerate}
		\item -
		\item -
		\item -
	\end{enumerate}

	\item Oppgave 5
	\begin{enumerate}
		\item -
		\item -
	\end{enumerate}
\end{enumerate}

 
 


\subsection{Gruppe 3}
\begin{enumerate}
	\item Oppgave 1
	\begin{enumerate}
		\item 1) planten skulle vokse 2) ingen fotosyntese
		\item 1) Det samme 2) Planten ble grønn (heliotropisme) 
		\item Plante 2 ble litt pjuskete
	\end{enumerate}

	\item Oppgave 2
	\begin{enumerate}
		\item 1) heliotropisme (pga. sollys) 2) pga. konstant lys? 2)vekst, blir lengre stilker
		\item - 
	\end{enumerate}	

	\item Oppgave 3
	\begin{enumerate}
		\item Plante 2 absorberer mye raskere
		\item større vekst
		\item slutt på vekstperiode?
		\item Ja, hvis ikke veldig mye fordampet
	\end{enumerate}

	\item Oppgave 4
	\begin{enumerate}
		\item 2
		\item konstant lys
		\item -
	\end{enumerate}

	\item Oppgave 5
	\begin{enumerate}
		\item flatet ut
		\item nådd høyest mulig høyde?
	\end{enumerate}
\end{enumerate}
\newpage
