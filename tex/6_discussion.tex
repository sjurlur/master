%!TEX root = ../document.tex
\chapter{Discussion}
In this chapter we will discuss our research question by contextualizing our findings to the theoretical concepts introduced earlier . As an overall theme we will look at the inquiry process of the students in interaction with Monoplant. This will be showed through 4 sections, the first being about the inquiry process. Next we will discuss how multiple external representations support the inquiry process of the students, then how scaffolding is instantiated in the environment, and finally how the institutional setting frame the students' inquiry process.


\section{Inquiry process}
In this section we will broadly address our first research question: \emph{"What characterizes the students’ inquiry in interaction with Monoplant?"}
In the previous chapter we presented excerpts from the session where the students interacted with Monoplant. We have seen that the students were generating hypotheses of what happened with the plant and why it grew as much as it did. We showed examples of explanations, discussion, misconceptions and surprises.

\subsection{Tentative hypothesis}
We designed the experiments together with the teacher. The students were given a problem in form of the assignments they discussed but needed to figure out the answers by themselves with the help of Monoplant, which presented detailed data logging of the experiments. The experiments conducted combined with the problem solving-session with the students can be categorized as \emph{guided inquiry}  \citetext{\citet{staver1987analysis}, referenced in \citealp{prince2006inductive}}. 

As showed in excerpt 1, Siri presented a hypothesis in which she stated that plant B would not grow much because it would not get as much light as plant A. In excerpt 2 data was presented to her that showed that plant B had indeed grown much. Her interpretation of this data was directed by her first hypothesis and because the data disconfirmed it, the next hypothesis is claiming there might be some error source in the experiment. Hence the misinterpretation did not result in a direct confirmation of the previous hypothesis, instead it laid the basis for a denial of a disconfirmation, or an explanation for why the first hypothesis did not hold even though she still thinks it should.

The four problems that \citet{de1998scientific} addressed for inquiry learning where: \textit{hypothesis generation}, \textit{design of experiments}, \textit{interpretation of data}, and \textit{regulation of discovery learning}. In our case we controlled two of these problems by designing and initiating the experiments for the students, as well as letting Monoplant do a systematic logging of data during the experiment, hence regulating the inquiry process. This meant that the students faced two problems, interpreting the data and generating hypotheses based on the data. \citeauthor*{klahr1993heuristics} \citetext{\citeyear{klahr1993heuristics}, referenced in \citealp{de1998scientific}} reported that misinterpretation of data often result in confirmation of current hypothesis. 



\subsubsection*{Delayed inquiry}
As the students were done with the textbook chapter of photosynthesis and were able to explain  phenomena such as growth theoretically, the presumptions of the outcomes to the experiment colored the interpretation of data because it was connected to the student's prior conceptual knowledge. Siri knew that plants make food for themselves by doing photosynthesis. To do photosynthesis, a green plant such as the cress in the experiment needs light of wavelengths other than green (blue and red). This reasoning makes sense to Siri because she knows a lot of the scientific concepts concerning the theme at hand. We can say that the inquiry process became deductive as it was affected by the students preconceptions and their ability to explain the observations they made with Monoplant. 

However, this is a misconception in inquiry learning, and what \citet{gomez2008elementary} refers to as inconsistent understanding, according to what the teacher intended. In this case Siri's conception of photosynthesis, which in the context of the textbook examples makes sense, becomes a misconception when she is confronted with a plant that germinates. Hence it leads her to an erroneous conclusion. \citet{smith1994misconceptions} makes the description of this kind of misconceptions as "faulty extensions of productive prior knowledge". So that a conception might help describe a phenomenon in one context, but falsely describe it in another context. \citeauthor{klahr1993heuristics} sets words to what seems to be the main issue: 

\begin{quote}"compared to the binary feedback provided to subjects in the typical psychology experiment, real-world evidence evaluation is not so straightforward" \citetext{\citet[p. 114]{klahr1993heuristics}, referenced in \citealp{de1998scientific}}
\end{quote}

Even though our field of study is different from \citeauthor{klahr1993heuristics}, this distinction helps us to illustrate what we can see in the students inquiry: the context of the plant in the experiment is new for the students, making it difficult for them to apply their prior knowledge to the phenomenon. 

We have now established that the inquiry process is influenced by the fact that the students have certain knowledge about photosynthesis. Coming into the experiment, this can at one hand lead to misconceptions due to the students having a great freedom to pursue their ideas through the inquiry process. In that case, these misconceptions should be followed up and corrected. On the other hand, the system or an instructor or can guide the students to pursue the fruitful ideas from the start, staying one step ahead of possible misconceptions. We will discuss this further in the section about scaffolding.



\section{Multiple external representations in inquiry processes}
During the inquiry process the students were presented with different representations of the photosynthesis phenomenon. In this section we will give account for how those representations were used in the inquiry process and how they complimented each other. We will also look at differences in the students' language when engaging in talk with the different representations. To recap, our second research question is as follows: \emph{"How does Monoplant, by presenting photosynthesis differently from the text book, support the inquiry process?"}

\subsection{Complementary processes}
%How does textbook present photosynthesis
When reviewing the textbook used in the school class' science education, we found that the scientific concepts are largely represented in a theoretical manner. In the first paragraph of the chapter concerning photosynthesis, scientific words such as "pigments", "chloroplasts" and "glucose" appear. Later on, photosynthesis is explained by its chemical formula and the chapter rarely gives any examples of how photosynthesis affects the life of plants at the concrete level. The textbook therefore emphasizes how photosynthesis fits into a larger system of scientific concepts, and is more concerned with conveying the big picture than the specific and concrete experiences. 

%How does Monoplant represent photosynthesis
On the other hand Monoplant affords a more inductive or "bottom-up" approach. As a learning resource, Monoplant is a tool for exploring ideas related to photosynthesis. The variables relevant for the plant's photosynthesis are mediated through graphs and videos, but leaving the interpretation of those data to the students. The system is only concerned with one plant in one specific context, not trying to generalize from the specific results to a larger scientific concept. 

When looking at our data with this in mind, a pattern in the students' language emerge. During the inquiry process, students use everyday language when engaging with Monoplant. An example comes from \emph{excerpt 10} where Siri says that the plant "use moisture from the earth". 

Another example is from \emph{excerpt 7} where students use concepts as "pop out", "capture" and "use sunlight". All of these concepts have their scientific counterpart in the textbook, but when discussing among themselves, the students choose to talk about the phenomenon in a "non-academic" way. 

However, the students' language seem to change when engaging with representations linked to the textbook. An example of this is from \emph{excerpt 8} where Siri use scientific concepts such as "chlorophyll molecule" and "excited" when looking at a textbook illustration of photosynthesis. 

An explanation of the change in language may be given by applying Vygotski{\u\i}'s (2012) theory of spontaneous and scientific concepts as presented in chapter \ref{cha:spontaneous_scientific}. When engaging with Monoplant, the students are addressing the results of a concrete experiment in a specific context. The concepts they use are therefore linked to what they observe. When Siri says that the plant "uses sunlight", it is because this is something she has experienced. She knows that the sun transfers energy that plants make use of, and she has perhaps seen plants die as a result of lack of light. This is an example of a spontaneous concept, "a nonconscious and nonsystematic" concept \citep{vygotsky2012thought}. Spontaneous concepts have their strength in explaining what concerns the situation, empirically and practically \citep{vygotsky2012thought}, and  therefore mediate the student's thoughts when discussing the plant on the screen in front of them. 

%The corresponding scientific concept in this case would be to "excite electrons", but this is an abstract concept that is difficult to link any concrete experiences to. As a result she feels more comfortable using the spontaneous concept: to "use sunlight" when explaining her thoughts to the other students. 

Yet we see from \emph{excerpt 8} that the same student also uses the scientific concept "excite electrons" when describing the same phenomenon, but now interacting with the textbook. This is a more abstract concept, but has its strength in its "conscious and deliberate character" \citep{vygotsky2012thought}. An explanation of the change in language may be that the student is not aware of the two concepts referring to the same phenomenon. She masters the scientific concept only in the realm of the textbook and the concept's relation to other scientific concepts. And she masters the spontaneous concept only when referring to the concrete experiments where they have observable results. 

Another more plausible explanation would be that in engaging with both Monoplant and the textbook, she has mastered both the scientific and spontaneous concept of exciting electrons. The spontaneous concept has "in it's slow way upwards cleared the path for a scientific concept" \citep{vygotsky2012thought}. The student is therefore able to speak of "exciting electrons", both when talking about the concrete experiment, and when discussing the experiment in more abstract terms. 

\citet{vygotsky2012thought} states that "as long as the curriculum supplies the necessary material, the development of scientific concepts runs ahead of the development of spontaneous concepts". We found this to be true in this setting as well. From \emph{excerpts 8-10} we can see that Siri, Nora and Fredrik are able to use the scientific concepts when discussing photosynthesis in general. The school has supplied the curriculum necessary for the scientific concept generation in the weeks before conducting the experiment, leading to the students mastering the scientific concepts. Whereas the students' inquiry process with Monoplant supplied a framework for enriching the scientific concept with personal experiences. This is what has enabled Siri to fully master the concept of "exciting electrons".  

On the other hand, we do not find any evidence of the other participants mastering the concept in the same way as Siri. Yet they are able to discuss the phenomenon with her using the scientific and spontaneous concepts, albeit not interchangeably. This would suggest that the other students are not far away from mastering both the scientific and spontaneous concept. The step from unconscious to controlled use of the spontaneous concept is therefore within their zone of proximal development. 

We believe our data warrants the assumption that different types of representations spurs complementary processes that can lead to stronger concept comprehension among the students. Inquiry-based environments have their strength in that they provide personal experiences, while more scientific representations (from the curriculum and the textbook) are able to place the phenomenon in a broader scientific context. As scientific concepts and spontaneous concepts mutually enrich and are dependent on each other \citep{vygotsky2012thought}, it is important to take the development of both into account when designing learning environments. 
%As shown, using Monoplant in the inquiry process can provide concrete experiences which helps the concept "come to life" \citep{van1998concept}. 
%Can we write about inquiry learning in general as spontaneous concepts? Can this be elaborated?

\subsection{Complementary roles}
During the inquiry process the students were faced with three fundamentally different representations of the same phenomenon: the textbook, the physical plant, and the digital Monoplant system. The textbook consists of textual representations, along with pictures, illustrations and graphs. The physical plant is a real-life representation of photosynthesis in action. And the Monoplant system mediates information such as timelapse-videos that would otherwise be unavailable in the real world. 

In our data we find data supporting that the students are able to use three of the different representations of the phenomenon interchangeably. First, they point at and discuss the physical plant while watching videos of the same plant in the Monoplant system. This can be seen from both \emph{excerpt 3} and \emph{excerpt 7}. Second,  

As pointed out by \citet{van2006supporting} there are many benefits of representing the same phenomenon in multiple ways. First, each of the representations can show specific aspects of the domain to be learned. Second, one representation can constrain the interpretation of another representation. And third, learners can build abstractions by translating between the representations, which may lead to a deeper understanding of the domain. 



But while the benefits of using MER in education seem obvious, both \citet{ainsworth1999functions} and \citet{van2006supporting} point to problems students face while undergoing extra tasks related to MER. To exemplify, let us take a look at the different representations involved in the students' inquiry process. The first task the students have to face is to understand the syntax of the representations. E.g. one of the graphs represented in the Monoplant system is relative, meaning that the different units of measurement are discarded and replaced with percentage-values. The students then have to understand what the different axes of the graph represent and how the variables relate to one another. Second, the students have to understand which parts of the domain are represented. Monoplant is designed in such a way that it does not constrain to one specific form of interaction, and the relation between Monoplant and external factors' effect on photosynthesis can therefore be somewhat obscure. And finally, the students have to understand the relation between the different representations. E.g. when playing a video file, it is necessary to see it in relation with the graph to get both the quantitative and qualitative aspects of the phenomenon. 

When discussing the physical plant the students often consulted the Monoplant system to look at videos of the plant's life at different points. Examples of this can be seen in \emph{excerpt 3} and \emph{excerpt 7}. This leads us to believe that the students had no problems in understanding the relation between the plant placed on the desk in front of them, and the videos from the different experiments. Another example of linking representations 

But when the students are encouraged to look at the relation between the graph and the physical plant, they experience larger difficulties and need more scaffolding to overcome the obstacles. This can be seen in \emph{excerpt 10} where the students are discussing the relation between the experiment and the graph. 

One explanation of this may be 


%In our data we see evidence of the students having problems with the latter of these tasks: comprehending the relation between the different representations. 




\subsection{Representation becomes Misconception}
As mentioned earlier, explanations can be accurate enough for one situation but lead to false conclusions in other situations. \citep{smith1994misconceptions} This is true for representations and models as well and becomes evident if we look at Excerpt 6. After looking at the textbook representation that uses the word "solar energy" to label photons, Nora asks "Can light cause excit.. that it excites. Or is it just the sun?". As the book provides the context to photosynthesis it mostly frames examples to the nature where sunlight and solar energy is indeed valid simplifications of photons. But in the case of the experiments with Monoplant, this simplification is challenged but not addressed. Monoplant show how much light the plant got, but does not distinguish what sort of light. The experiments were however designed in such a way that it differentiated light quality (wavelength of light). Nora might have interpreted the experiments to address differences to a plant that have access to solar energy and one which gets another type of light energy. In any case this is a good example of how an explanation can be plausible and have explanatory power in one setting, but lead to erroneous conclusions in another setting. This is also a great example of the need for scaffolding in an inquiry process, which leads us to the next section........ . . . 









How does Monoplant, by visualizing/present photosynthesis differently from the text book, support the inquiry process? 
Here we can talk about: 
\begin{itemize}
\item{how do the textbook represent photosynthesis/external factors}
\item{how do Monoplant represent photosynthesis/external factors}
\item{\nameref{ex:excerpt10} (soilmoist)}
\item{\nameref{ex:excerpt6} (misconception from textbook representation)}
\item{Excerpts: 2, 3, 4, 6, 7, 8, 10}
\item{MER}
\item{Social practices}
\item{Everyday language \& Scientific language}
\end{itemize}




\section{Scaffolding}
How is scaffolding instantiated in the environment?
Here we can talk about: 
\begin{itemize}
\item{ZPD}
\item{Scaffolding}
\item{Inquiry learning}
\item{Misconceptions}
\item{Excerpts: (3), 4, 5, 6, 8, 9}
\end{itemize}



\section{Institutional setting}
Monoplant was tested in a biology class at the highest level offered at Norwegian high schools. It was tested during school hours in the classroom for biology. The teacher was present, walking around helping all the groups as they were trying to solve the task at hand. In this section we will address our final research question: \emph{"How does the institutional setting frame the students inquiry process?"} (“doing school \& doing science”)
By this we mean to set focus on how the institutional setting affects the students interaction with monoplant and their inquiry process. By filming and recording one of the groups, they were able to keep an oral discussion without actually writing any answers down on paper. We hoped that this would help us to avoid a test-like situation where the students became interested in answering correct, but rather stimulate discussion and let them negotiate and explore possible solutions.

\subsection{Doing science}
In \emph{excerpt} 7, Fredrik is trying to explain why plant A grows leafs much faster than plant B when Siri asks a control question to check if she understands what he means. He then has to make his explanation more concrete. It is apparent that they negotiate their way through generating an explanation for the observed data. 

Once the students had found that plant B grew much in \emph{excerpt 2}, Siri asked if plant A also had white stems. As mentioned this was probably to check if cress has white stems in general. If this was not the case, the white stems could be used as evidence to prove that plant B did no photosynthesis, as a plant with absolutely no photosynthesis would most likely be white. 

Both \emph{excerpt 7} and \emph{excerpt 2} contain examples of the students doing scientific activities such as evaluating evidence and constructing explanations. This cleary falls into what \citet{jimenez2000doing} calls "doing science". 

\subsection{Doing school}
We have showed some examples test-like situations in our data. Both \emph{excerpt 8} and \emph{9} are examples of situations where questions are asked to the group, and both times the language of the students changes dramatically. In the analysis of \emph{Excerpt 9}, we have presented some possible explanations for why Nora brings in the light-dependent reaction, NADPH and ATP at that moment. 1. Sjur asks a question and creates a test-like situation. 2. The students have noticed that the teacher is observing them, creating expectations to impress him with knowledge of the curriculum. 3. Nora might want to try explaining the phenomena by bringing in a new representation, as they up till now have failed to be able to explain it with their everyday language. Apart from the latter, these explanations shed light on how the educational and institutional setting affects the student talk to become more closely linked to the curriculum. In other words it instantiates rituals, routines and expectations in the educational setting, placing it in the category \citet{jimenez2000doing} calls "doing school". 


\subsection{Holistic perspective}


\subsection{Embeded practices in design}
Monoplant takes hand of datalogging

review homework assignments, take lecture notes, take tests, complete lab activities

embedded stuff: datalogging, -assignments, language,

\begin{itemize}
\item{Institutional settings}
\item{Inquiry learning}
\item{Everyday language \& Scientific language}
\item{Excerpts: 1, 5, 8, 9}
\end{itemize}
