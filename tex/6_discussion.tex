%!TEX root = ../document.tex
\chapter{Discussion}
In this chapter we will discuss our research question by contextualizing our findings to the analytical concepts introduced earlier in this thesis. 

\section{Inquiry process}
What characterizes the students’ inquiry in interaction with monoplant?
Here we can talk about: 
\begin{itemize}
\item{Inquiry learning and the problems. \citep{de1998scientific}}
\item{Misconceptions}
\item{Everyday language \& Scientific language}
\item{Excerpts: 1, 2, 3, 4, 6, 7, 10}
\end{itemize}



\section{Multiple External Representations in Inquiry processes}
How does Monoplant, by visualizing/present photosynthesis differently from the text book, support the inquiry process? 
Here we can talk about: 
\begin{itemize}
\item{how do the textbook represent photosynthesis/external factors}
\item{how do Monoplant represent photosynthesis/external factors}
\item{\nameref{ex:excerpt10} (soilmoist)}
\item{\nameref{ex:excerpt6} (misconception from textbook representation)}
\item{Excerpts: 2, 3, 4, 6, 7, 8, 10}
\item{MER}
\item{Social practices}
\item{Everyday language \& Scientific language}
\end{itemize}




\section{Scaffolding}
How is scaffolding instantiated in the environment?
Here we can talk about: 
\begin{itemize}
\item{ZPD}
\item{Scaffolding}
\item{Inquiry learning}
\item{Misconceptions}
\item{Excerpts: (3), 4, 5, 6, 8, 9}
\end{itemize}



\section{Institutional setting}
How does the institutional setting frame the students inquiry process? (“doing school \& doing science”)
Here we can talk about: 
\begin{itemize}
\item{Institutional settings}
\item{Inquiry learning}
\item{Everyday language \& Scientific language}
\item{Excerpts: 1, 5, 8, 9}
\end{itemize}
