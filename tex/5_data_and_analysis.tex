%!TEX root = ../document.tex
\chapter{Data \& Analysis}
In this chapter we will present the findings from our case study ...

Something worth noting?:
\begin{table}[H]
\begin{center}
	\begin{tabular}{l r r } \toprule
	Who &  Interactions  & Percentage\\ \midrule  
	Linda &	 14  & 3.67\% \\
	Nora&	118 & 30,97\% \\ 
	Siri& 	182 & 47.77\% \\
	Fredrik& 67 & 17.59\% \\ \midrule
	Everybody &	381 & 100\%\\
	\bottomrule
	\end{tabular}
\end{center}
\end{table}
381 beeing the total number of interactions (verbal that is)

\section{Hypothesis generation}


\section{Some headline about scaffolding}

\subsection{Hypothesis generation based on misconception}


\subsubsection*{Context}
In table~\ref{excerpt:disconfirmation1} the students have been looking at the movements of the two plants, and have observed that the plant in the window are moving towards the sun, a so called heliotropism. They are now observing the movements of the plant in the closet and Fredrik have just pointed out that it is growing straight up without any large movement like the other plant. All the students in the group are watching the video closely. Suddenly Nora observes that the plant is growing a lot faster and higher than the window plant.


\subsubsection*{Raw data}

\def\arraystretch{1.5}
\begin{table}[H]
\begin{adjustwidth}{-4em}{-4em}
\begin{center}
\begin{tabular}{r l p{9cm} p{4cm} } \toprule
	Time &  Who &  Speech  & Action\\ \midrule  

	7:46 %time
	&Nora %name
	&\parbox[t]{9cm}{\raggedright Jeg føler at de vokser veldig mye inni ... skapet eller er det? ... %speech 
	}&\parbox[t]{4cm}{\raggedright  %action 
	}\\

	7:51 %time
	&Siri %name
	&\parbox[t]{9cm}{\raggedright Ja det virka som om de vokste ... %speech 
	}&\parbox[t]{4cm}{\raggedright  %action 
	}\\

	7:53 %time
	&Nora %name
	&\parbox[t]{9cm}{\raggedright ... ser ut som de ble lenger lissom ... %speech 
	}&\parbox[t]{4cm}{\raggedright  %action 
	}\\

	7:53 %time
	&Siri %name
	&\parbox[t]{9cm}{\raggedright ... enda mer der. %speech 
	}&\parbox[t]{4cm}{\raggedright  %action 
	}\\

	7:54 %time
	&Fredrik %name
	&\parbox[t]{9cm}{\raggedright ja %speech 
	}&\parbox[t]{4cm}{\raggedright  %action 
	}\\

	7:56 %time
	&Siri %name
	&\parbox[t]{9cm}{\raggedright ... enn ute, at de ble mye lengre. %speech 
	}&\parbox[t]{4cm}{\raggedright  %action 
	}\\

	7:59 %time
	&Fredrik %name
	&\parbox[t]{9cm}{\raggedright mhm. %speech 
	}&\parbox[t]{4cm}{\raggedright  %action 
	}\\

	8:01 %time
	&Siri %name
	&\parbox[t]{9cm}{\raggedright Kanskje de fokuserer veldig på å vokse oppover når lyset er rett over dem.. at de vokser rett oppover ((fører hånden oppover)) i stedet for å følge lyset og gå lissom sånn sakte oppover ((snurrer hånden sakte oppover)) %speech 
	}&\parbox[t]{4cm}{\raggedright  %action 
	}\\
	
	
	\bottomrule
\end{tabular}
\end{center}
\end{adjustwidth}
\caption{Excerpt from disconfirmation of growth}
\label{excerpt:disconfirmation1}
\end{table}

\subsubsection*{Explanation}
When Nora states that the plants are growing more in the closet, she is very cautious when introducing the idea, it seems like an unlikely observation. Siri approves and states that it is indeed growing more than the plant outside of the closet, Fredrik agrees and they all seem a bit puzzled by this observation.
After they have agreed on this, Siri starts to generate a hypothesis for why the plant in the closet are growing more than the one in the window. It might look like it is hard for her to realize that her expectations where wrong and so she is trying to cope with this by finding explanations why the plants grew so different from what she expected. It seems like she is mixing germination and photosynthesis, which is not so weird, because the textbook does not say anything about germination. 
Basically what she is saying is that heliotropism makes the plant in the window grow slower because it has to move after the sun, and since the plant in the closet can grow straight up without following the sun, it can grow faster. 

\subsection{Scaffolding to fix misconception}

\subsubsection*{Context}
In table~\ref{excerpt:scaffold1} the teacher has been talking with the group for a couple of minutes and Siri has talked about her hypothesis that the plant in the closet might have got more than just green light, and if it only got green light it would probably not grow that much.
 
\subsubsection*{Raw data}

\def\arraystretch{1.5}
\begin{table}[H]
\begin{adjustwidth}{-4em}{-4em}
\begin{center}
\begin{tabular}{r l p{9cm} p{4cm} } \toprule
	Time &  Who &  Speech  & Action\\ \midrule  

	13:44 %time
	&Lærer %name
	&\parbox[t]{9cm}{\raggedright ja.. så altså dere tenker at .. sammenhengen mellom \underline{vekst} og fotosyntese den er helt klar ... du kan ikke du tenker at du kan ik et frø kan ikke spire og vokse og bli en plante uten at drives fotosyntese.. tenker dere alle det? %speech 
	}&\parbox[t]{4cm}{\raggedright  %action 
	}\\

	14:00 %time
	&Fredrik %name
	&\parbox[t]{9cm}{\raggedright Det er jo noen planter som ikke har fotosyntese ... og de spirer jo og fordet ikkesant.. det er vel en liten energipakke på en måte i  frøet da? er det ikke det da? %speech 
	}&\parbox[t]{4cm}{\raggedright  %action 
	}\\

	14:14 %time
	&Lærer %name
	&\parbox[t]{9cm}{\raggedright okei, er det? %speech 
	}&\parbox[t]{4cm}{\raggedright  %action 
	}\\

	14:14 %time
	&Nora %name
	&\parbox[t]{9cm}{\raggedright Ja %speech 
	}&\parbox[t]{4cm}{\raggedright nikker annerkjennende %action 
	}\\
	
	\bottomrule
\end{tabular}
\end{center}
\end{adjustwidth}
\caption{Excerpt from teacher talk}
\label{excerpt:scaffold1}
\end{table}

\subsubsection*{Explanation}
In the excerpt the teacher formulates a question which lead the group to think outside of the model of photosynthesis and hints to the germination process. Fredrik starts to answer the question at once, and introduces the notion that there are plant that do not have photosynthesis, and that the seeds have an energy pack. This notion lays the basis for a discussion in the group where the teacher leads the students to find out that seeds have starch as a food reserve, which makes germination possible. 

\section{Language}
\subsection{Scientific language}


\subsubsection*{Context}
Earlier Morten explicitly told the group to look at the light graphs again, however, the group only stated the fact that the plant in the closet got a constant amount of light, where as the window plant got a lot of light during the day, but nothing at night. So Sjur threw in a comment that the plant in the closet has a low level of light all the time. This is where the excerpt starts in table~\ref{excerpt:hypothesis3.1}, the other excerpt (see table~\ref{excerpt:hypothesis3.2}) is from right after this when the teacher arrives to speak with the group.


\subsubsection*{Raw data}

\def\arraystretch{1.5}
\begin{table}[H]
\begin{adjustwidth}{-4em}{-4em}
\begin{center}
\begin{tabular}{r l p{9cm} p{4cm} } \toprule
	Time &  Who &  Speech  & Action\\ \midrule  

	10:49 %time
	&Sjur %name
	&\parbox[t]{9cm}{\raggedright mens den andre gjerne .. nesten ligge på null heile veien da .. (?) %speech 
	}&\parbox[t]{4cm}{\raggedright Fredrik og Nora snur seg. Nora nikker %action 
	}\\

	10:53 %time
	&Siri %name
	&\parbox[t]{9cm}{\raggedright Å ja! det var jo lavere lys der ((refererer til skapplanten)), men så blir det veldig mye lys her ((refererer til vindusplanten)) når det først er lys. %speech 
	}&\parbox[t]{4cm}{\raggedright har et ganske bekymret ansiktsuttryk mens hun prøver å forstå hva hun sier. %action 
	}\\

	11:11 %time
	&Sjur %name
	&\parbox[t]{9cm}{\raggedright Men hvis dere ser på baksiden av det oppgavearket %speech 
	}&\parbox[t]{4cm}{\raggedright Peker mot arket. Nora snur arket %action 
	}\\

	\bottomrule
\end{tabular}
\end{center}
\end{adjustwidth}
\caption{Excerpt from some important stuff}
\label{excerpt:hypothesis3.1}
\end{table}

\def\arraystretch{1.5}
\begin{table}[H]
\begin{adjustwidth}{-4em}{-4em}
\begin{center}
\begin{tabular}{r l p{9cm} p{4cm} } \toprule
	Time &  Who &  Speech  & Action\\ \midrule  

	 %time
	& %name
	&\parbox[t]{9cm}{\raggedright  %speech 
	}&\parbox[t]{4cm}{\raggedright Lærer kommer bort %action 
	}\\

	11:20 %time
	&Lærer %name
	&\parbox[t]{9cm}{\raggedright Går det bra eller %speech 
	}&\parbox[t]{4cm}{\raggedright kommer bort til bordet og lener seg på det. %action 
	}\\

	11:23 %time
	&Siri %name
	&\parbox[t]{9cm}{\raggedright mmm, ja %speech 
	}&\parbox[t]{4cm}{\raggedright  %action 
	}\\

	11:24 %time
	&Lærer %name
	&\parbox[t]{9cm}{\raggedright skjønner dere ... har dere funnet forklaring på alle spørsmålene? %speech 
	}&\parbox[t]{4cm}{\raggedright  %action 
	}\\

	11:26 %time
	&Alle jentene %name
	&\parbox[t]{9cm}{\raggedright *** vi prøver ... %speech 
	}&\parbox[t]{4cm}{\raggedright snakker i munnen på hverandre %action 
	}\\

	11:27 %time
	&Siri %name
	&\parbox[t]{9cm}{\raggedright Jeg tror kanskje jeg har en ide om det med at den her ute ((peker mot vinduet, refererer til planten i vinduet)) ikke vokser like høyt, eller så fort ihvertfall.. fordi atte når det kommer veldig mye sol så blir jo klorofyllmolekylene eksitert, men når alle ... alle klorofyllene blir eksitert i planten, sånn atte det ikke er flere som kan bli eksitert så hjelper det ikke om det er mere lys. %speech 
	}&\parbox[t]{4cm}{\raggedright  %action 
	}\\

	11:55 %time
	&Lærer %name
	&\parbox[t]{9cm}{\raggedright Så det du tenker er rett og slett at den hemmes av for mye lys, at den ikke vokser så mye fordi det er så mye lys? %speech 
	}&\parbox[t]{4cm}{\raggedright  %action 
	}\\

	12:03 %time
	&Siri %name
	&\parbox[t]{9cm}{\raggedright Kanskje ikke hemmes .. det .. hvis det er veldig sterkt lys kan jo pigmentene bli svidd, men  når det er  litt mere lys enn alt det de kan ta opp.. så hjelper det ikke at det er litt mer, for da kan de ikke ta opp det ekstr... %speech 
	}&\parbox[t]{4cm}{\raggedright  %action 
	}\\

	\bottomrule
\end{tabular}
\end{center}
\end{adjustwidth}
\caption{Excerpt from teacher talk}
\label{excerpt:hypothesis3.2}
\end{table}


\subsubsection*{Explanation}
In the excerpt in table~\ref{excerpt:hypothesis3.1} Siri understands that it is a difference in the light intensity, not just when the plants get light. However, when she explains it to the teacher in table~\ref{excerpt:hypothesis3.2}, it seems like she interprets it to mean that the plant in the window get too much light, and that light becomes a limiting factor for the plants growth. When she explains her hypothesis to the teacher, she is using a more scientific language than before, and mentions chlorophyll molecules that gets excited. This might be because Sjur introduced to the representation of the light dependent reaction of the photosynthesis just before the teacher arrived, \sout{or it might be because this is the first time the teacher is listening to the group and she wants to impress him with sciencebable}.


\subsection{Everyday language}


\subsubsection*{Context}
Teacher has left, Morten asked the students to look at the plant videos and see if there is any difference in their appearance. The students have looked at the plant i the cupboard and found that it is mostly the stem that grows, not the leafs. Fredrik has requested that they should check the window plant to compare the two, and Siri has started the video from 29th of October.


\subsubsection*{Raw data}

\def\arraystretch{1.5}
\begin{table}[H]
\begin{adjustwidth}{-4em}{-4em}
\begin{center}
\begin{tabular}{r l p{9cm} p{4cm} } \toprule
	Time &  Who &  Speech  & Action\\ \midrule  

	17:12 %time
	&Siri %name
	&\parbox[t]{9cm}{\raggedright Der åpner jo bladene seg med en gang nesten %speech 
	}&\parbox[t]{4cm}{\raggedright  %action 
	}\\

	17:15 %time
	&Fredrik %name
	&\parbox[t]{9cm}{\raggedright ja ... ((stillhet, venter til video er ferdig)) det kan jo ha noe med at her trenger den jo bladene for fange lyset da, mens den trenger jo ikke det så mye inni skapet.. eh kanskje %speech 
	}&\parbox[t]{4cm}{\raggedright Planten trenger ikke bladene i skapet fordi det ikke er så mye lys? %action 
	}\\

	17:34 %time
	&Siri %name
	&\parbox[t]{9cm}{\raggedright at den bruker næringen fra jorda og frøet mer i skapet? %speech 
	}&\parbox[t]{4cm}{\raggedright  %action 
	}\\

	17:37 %time
	&Fredrik %name
	&\parbox[t]{9cm}{\raggedright ehhhh.. ja. eller at den ikke utnytter den sol.. det sollyset inne i skapet så det den trenger jo ikke da også at bladene spretter ut så tidlig eller at... eh ja. %speech 
	}&\parbox[t]{4cm}{\raggedright Fredrik er ikke helt enig med Siri. Mener at planten i skapet ikke har noe lys å utnytte, derfor ingen blader %action 
	}\\

	\bottomrule
\end{tabular}
\end{center}
\end{adjustwidth}
\caption{Excerpt from some usage of everyday language}
\label{excerpt:everydaylanguage}
\end{table}




\subsubsection*{Explanation}
In the excerpt in table~\ref{excerpt:everydaylanguage} the students are using everyday words like \emph{leaf, sunlight, pops out, makes use of} etc. to explain what is happening with the plant. 


\section{Linking between representations}
\subsection{Soil moisture representation}


\subsubsection*{Context}
The students have read the introduction to assignment 3, Siri has navigated to the soil moisture graph in the system, and got some help from Sjur to expand the graph so they could watch the entire graph. After expanding the graph to include the lifespan of both plants, Siri leaves the system to read assignment 3a: \emph{Is there any difference in the absorption rate?}



\subsubsection*{Raw data}

\def\arraystretch{1.5}
\begin{table}[H]
\begin{adjustwidth}{-4em}{-4em}
\begin{center}
\begin{tabular}{r l p{9cm} p{4cm} } \toprule
	Time &  Who &  Speech  & Action\\ \midrule  

	21:34 %time
	&Nora %name
	&\parbox[t]{9cm}{\raggedright Åja, fra de... den og den ((peker på høyre og venstre side av grafen)) %speech 
	}&\parbox[t]{4cm}{\raggedright Lager v-tegn med fingrene og viser hvilken periode i grafen planten var i vinduet, og hvilken periode den var i skapet %action 
	}\\

	21:36 %time
	&Sjur %name
	&\parbox[t]{9cm}{\raggedright ja. %speech 
	}&\parbox[t]{4cm}{\raggedright  %action 
	}\\

	21:37 %time
	&Siri %name
	&\parbox[t]{9cm}{\raggedright Åja, så det der er den ene planten og det der er den andre.. %speech 
	}&\parbox[t]{4cm}{\raggedright Peker først på venstre side av grafen, så på høyre %action 
	}\\

	21:41 %time
	&Nora %name
	&\parbox[t]{9cm}{\raggedright mhm, den der går litt brattere ned på ... %speech 
	}&\parbox[t]{4cm}{\raggedright Peker på området i grafen hvor planten sto i skapet %action 
	}\\

	21:44 %time
	&Fredrik %name
	&\parbox[t]{9cm}{\raggedright Ja, den går mye brattere ned. %speech 
	}&\parbox[t]{4cm}{\raggedright  %action 
	}\\

	21:46 %time
	&Siri %name
	&\parbox[t]{9cm}{\raggedright Kanskje det betyr at den der andre planten bruker mye mer fuktighet fra jorden %speech 
	}&\parbox[t]{4cm}{\raggedright Peker på området i grafen hvor planten sto i skapet %action 
	}\\

	\bottomrule
\end{tabular}
\end{center}
\end{adjustwidth}
\caption{Excerpt from some important stuff}
\label{excerpt:hypothesis3.1}
\end{table}

\subsubsection*{Explanation}
Here the students are looking at Monoplants representation of the soil moisture over time. They interpret it as how wet the soil is, and when curves in the graph go downwards, it represents when the plant is using water. Since the curves from the plant in the cupboard is much steeper than from the one in the window, Siri interpret this to mean that the plant in the cupboard uses a lot more water.