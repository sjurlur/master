%!TEX root = ../document.tex
\begin{abstract}

As digital sensors become cheaper and more widespread, an opportunity to automate data-logging in the science classroom arise. Through a design experiment we have built a system for monitoring plants and examined it's use in the context of a biology class. 

This thesis reports a qualitative study of students' scientific inquiry in interaction with Monoplant. In order to demonstrate problems and opportunities for learning with Monoplant, a detailed analysis of the students' inquiry process in two experiments have been conducted. Our data material consists of video from a session where groups of students worked with questions related to the experiments. Our analysis indicated that the students had problems interpreting data from Monoplant as it refuted some of their preconceptions regarding photosynthesis. This finding is discussed by applying the concept of inquiry learning and misconceptions within inquiry learning. Further, when engaging with each other, the teacher, Monoplant and the curriculum, indications of different mindsets within the students arose. When exploring plant development within Monoplant, the students externalized their thoughts through an everyday language closely linked to what they observed in the system. However, when the students where talking with their teacher or exposed to textbook material, a more scientific oriented language was used. These findings are discussed by looking at how different representations, institutional settings and social practices affect student talk. 

\end{abstract}
\begin{abstract}

The purpose of this research was to explore how students interact and learn with our scientific inquiry-system, Monoplant. The system is an Internet-connected plant, visualizing different aspects of plant biology through a web-interface. The study investigates how the system affects the educational context and how it supports the students' inquiry process.  We performed a design experiment where a biology class performed science experiments using our system. Our data material consists of video from a session where groups of students worked with questions related to the experiments. We are positioned within the sociocultural perspective and maintain a focus on how the institutional aspects of the school affect the learning process. The findings indicate that multiple representations should be used in scientific inquiry, inquiry learning can lead to scientific misconceptions, and that students have difficulties combining the requirements of the school with the scientific inquiry process.
\end{abstract}