%\documentclass[a4paper,norsk,12pt]{article} % Dokumentopsett.
%\usepackage[utf8]{inputenc} % Tegnsett UTF-8.
%\usepackage{hyperref}



%\begin{document}
\section{Samtykkeskjema}
\label{samtykkeskjema}
\subsection*{Forespørsel om deltakelse i forskningsprosjekt}


\subsubsection*{Bakgrunn og formål}
Vi er to masterstudenter i design, bruk og interaksjon ved Universitetet i Oslo og holder nå på med den avsluttende mastergraden, med tittel "Monoplant - learning with mixed reality". Oppgaven skrives ved institutt for informatikk og institutt for pedagogikk. 

Vi har utviklet et system som registrerer en plantes endringer over lang tid. Systemet består av en plante, ulike sensorer, og ett kamera. Bildene blir satt sammen til en video som vises i rask film, og sensordata blir presentert i grafer. Dermed kan man "se gresset gro" med det blotte øyet, og finne ut hvilke fysiske faktorer som har innvirkning på en plantes vekst. 

Vi er interessert i å finne ut hvordan elever tar i bruk denne teknologien for å undersøke konsepter innenfor fotosyntesen. 

\subsubsection*{Hva innebærer deltakelse i studien?}
Deltakere i studien vil gjennomføre et eksperiment over tid hvor man bruker monoplant til å følge livet til en plante under ulike ytre forhold. Systemet vil være til fri disposisjon både på nett og i klasserommet under hele perioden. Avslutningsvis vil vi samle inn data i form av video- og lydopptak av en gruppe på tre til seks elever. Ett kamera vil være vendt mot ansiktene til deltakerne, og ett kamera vil være vendt mot en datamaskin hvor deltakerne bruker systemet. Dette vil skje i løpet av en skoletime. 

\subsubsection*{Hva skjer med informasjonen om deg?}
Alle personopplysninger vil bli behandlet konfidensielt. Video- og lydopptakene vil kun være tilgjengelige for oss under arbeidet med oppgaven. I den ferdige oppgaven vil all informasjon bli anonymisert. Identifiserende faktorer som skole og sted vil være utelatt. 

Video- og lydopptakene vil kun ligge på våre datamaskiner i ett låsbart rom. Filene vil i tillegg være passordbeskyttet. 

Prosjektet skal etter planen avsluttes 31. oktober 2014. På denne datoen vil alle video- og lydopptak slettes. 

\subsubsection*{Frivillig deltakelse}
Det er frivillig å delta i studien, og du kan når som helst trekke ditt samtykke uten å oppgi noen grunn. Dersom du trekker deg, vil alle opplysninger bli anonymisert. 

Dersom du har spørsmål til studien, ta kontakt med Sjur Seibt (tlf: 99229275 epost: \href{mailto:sjursei@ifi.uio.no}{sjursei@ifi.uio.no}), eller Morten Kjelling (tlf: 48108450 epost \href{mailto:mortenok@ifi.uio.no}{mortenok@ifi.uio.no}). Veileder for prosjektet er professor Anders Mørch ved institutt for pedagogikk (tlf: 22840713 epost: \href{mailto:anders.morch@iped.uio.no}{anders.morch@iped.uio.no}). 

Studien er meldt til Personvernombudet for forskning, Norsk samfunnsvitenskapelig datatjeneste AS

\subsection*{Samtykke til deltakelse i studien}
Jeg har mottatt informasjon om studien, og er villig til å delta
\newline\newline\newline\newline
\line(1,0){250}

(Signert av prosjektdeltaker, dato)
%\end{document}
\newpage