%!TEX root = ../document.tex
\chapter{Analysis}
The analytical procedure employed within this thesis is \emph{Interaction analysis} \citep{jordan1995interaction}, an interdisciplinary method which emerged from fields such as ethnography, sociolinguistics, ethnomethodology, conversation analysis, and sociocultural theories. 

\begin{quote}
An interdisciplinary method for the empirical investigation of the interaction of human
beings with each other and with objects in their environment. It investigates human
activities such as talk, nonverbal interaction, and the use of artifacts and technologies,
identifying routine practices and problems and the resources for their solution \citep[p39]{jordan1995interaction}
\end{quote}

For Interaction analysis to become a reality video and audio technology has been a vital resource. The combination of recording talk as well as nonverbal interaction, the ability to replay a film of interaction as many times as necessary, whenever needed. Combining this micro-level data with ethnographic data gives us a means of analyzing how the interaction is part of the situated context and institutional practices. \citep{furberg2009scientific}.

%Crang & cook: Video recordings can be criticized by pointing out that it is the researcher that selects the frame and focus. Hence the data can become biased. However, by trying to frame interaction generically, and providing a video, getting other people to double check coding and transcription.