%!TEX root = ../document.tex
\section{Oppgaver til forsøk - Foss VGS 13.11.2013}

\subsection{Gruppe 1}

\begin{enumerate}
	\item Oppgave 1
	\begin{enumerate}
		\item Fuktigheten i jorden går ned. Kom til å spire - men ikke like avhengig av lyset
		\item I grønt lys ble karsen mye lenger
		\item Strakk seg antakelig høyere for å "se" etter lys
	\end{enumerate}

	\item Oppgave 2
	\begin{enumerate}
		\item Ja, plante 1 beveger seg frem og tilbake, mens plante 2 vokser rett opp.
		\item Plantene beveger seg etter lyset
	\end{enumerate}	

	\item Oppgave 3
	\begin{enumerate}
		\item Ja, planten i grønt lys absorberte vannet mye fortere enn den i sollys
		\item Dette er antakelig fordi det var mange fler planter i plante 2, og at de vokste mye fortere og høyere. 
		\item Kanskje klarer ikke planten å utnytte vannet like godt når det er lite. Eller ubalanse mellom jorden og luften. Fordamp.
		\item Man skulle tro at jo mer vann en plante tar opp, desto mer fotosyntese driver den. Dette vet vi imidlertid ikke.
	\end{enumerate}

	\item Oppgave 4
	\begin{enumerate}
		\item Den i grønt lys vokser raskest
		\item Antakelig fordi den strekker seg etter lys. 
		\item Bladene på planten som står i sollys krummer seg mer en bladene på planten i grønt lys. 
	\end{enumerate}

	\item Oppgave 5
	\begin{enumerate}
		\item Planten vokser mye fortere i starten enn på slutten
		\item Planten står i grønt lys og vil ikke drive fotosyntese. Derfor vil den naturligvis slutte å vokse når næringen i frøene er brukt opp. 
	\end{enumerate}
\end{enumerate}

\subsection{Gruppe 2}

\subsection{Gruppe 3}