%!TEX root = ../document.tex
\chapter{Data \& Analysis}
In this chapter we will present the findings from our case study ...

\section{Hypothesis generation}

\subsection{Context}
In table~\ref{excerpt:expectations1} the students are discussing assignment 1a, and are talking about what they thought would happen to the plant which only got green light. While discussing they are also describing the what the conditions for the plant was. The plant from the closet is in front of them on the table, but they do not know which plant it is. 
\subsection{Raw data}


\def\arraystretch{1.5}
\begin{table}
\begin{adjustwidth}{-4em}{-4em}
\begin{center}
\begin{tabular}{r l p{9cm} p{4cm} } \toprule
	Time &  Who &  Speech  & Action\\ \midrule  

	2:16 %time
	&Siri %name
	&\parbox[t]{9cm}{\raggedright .. det var det planten stod i skapet også skulle det være bare grønt lys på den ... men det kan jo hende for eksempel at det kom litt annet lys inn i skapet også .. så da er det ikke sikkert at det bare bar grønt lys ..  %speech 
	}&\parbox[t]{4cm}{\raggedright peker på skapet %action 
	}\\

	2:31 %time
	&Nora %name
	&\parbox[t]{9cm}{\raggedright  %speech 
	}&\parbox[t]{4cm}{\raggedright nikker %action 
	}\\

	2:31 %time
	&Siri %name
	&\parbox[t]{9cm}{\raggedright og planten tar jo opp littegrann grønt lys også, men ikke så mye .. så derfor kunne det hende atte den ikke vokste like my.. eller jeg trodde at den ikke ville vokse like mye i skapet .. siden da fikk den bare grønt lys ...  %speech 
	}&\parbox[t]{4cm}{\raggedright  %action 
	}\\

	2:46 %time
	&Nora %name
	&\parbox[t]{9cm}{\raggedright ... mmm ... %speech 
	}&\parbox[t]{4cm}{\raggedright  %action 
	}\\

	2:47 %time
	&Siri %name
	&\parbox[t]{9cm}{\raggedright eller neste bare grønt lys ihvertfall ... men hvor mye vokste den egentlig? er det den ((refererer til planten på bordet)) som stod i skapet? %speech 
	}&\parbox[t]{4cm}{\raggedright peker på planten som står på pulten %action 
	}\\

	2:52 %time
	&Sjur %name
	&\parbox[t]{9cm}{\raggedright ja %speech 
	}&\parbox[t]{4cm}{\raggedright  %action 
	}\\

	2:53 %time
	&Nora %name
	&\parbox[t]{9cm}{\raggedright OJ(!) %speech 
	}&\parbox[t]{4cm}{\raggedright  %action 
	}\\

	2:53 %time
	&Siri %name
	&\parbox[t]{9cm}{\raggedright Den har jo vokst ganske mye %speech 
	}&\parbox[t]{4cm}{\raggedright smiler %action 
	}\\
	
	\bottomrule
\end{tabular}
\end{center}
\end{adjustwidth}
\caption{Excerpt from exercise 1}
\label{excerpt:expectations1}
\end{table}

\subsection{Explanation}
Siri had an expectation that the plant from the closet would not grow as much as the plant from the window, and she is explaining why she thinks that. At the same time, she knows from looking at the system earlier, that the plant have grown, so she tries to explain why it has grown at all. 


\section{Hypothesis generation}

\subsection{Context}
After looking at the plant on the table the student wanted to know if the stems on the plants in the window where white as the stems in the closet. They opened monoplant and navigated to the videolist, where they got an overview over the looks of the two different plants. They checked the color of the stems, and opened a video from 31th of October, then pressing play. This is when Fredrik starts to talk in table~\ref{excerpt:hypothesis1}.



\subsection{Raw data}


\def\arraystretch{1.5}
\begin{table}
\begin{adjustwidth}{-4em}{-4em}
\begin{center}
\begin{tabular}{r l p{9cm} p{4cm} } \toprule
	Time &  Who &  Speech  & Action\\ \midrule  

	3:24 %time
	&Fredrik %name
	&\parbox[t]{9cm}{\raggedright mhm ... mmja så da er det jo egentlig ganske ... ja ikke så stor forskjell da på de som stod ...  i skapet ((peker på planten på border)) og de som stod i vinduskarmen hvis man bare ser på ...  utseende %speech 
	}&\parbox[t]{4cm}{\raggedright Dette sies mens Siri starter videoen, hun stopper også videoen før de har sett den halvferdig. %action 
	}\\

	3:37 %time
	&Siri %name
	&\parbox[t]{9cm}{\raggedright ja .. men da ville jeg kanskje tenke at det kan hende at det kom inn annet lys enn det grønne lyset også. siden de har vokst så bra, og at de vokser bedre hvis de får flere.. lys i flere bølgelengder enn bare grønt lys %speech 
	}&\parbox[t]{4cm}{\raggedright Stemmeleiet går opp mot slutten av setningen, og løfter blikket fra arket for å få bekreftelse %action 
	}\\

	... %time
	&... %name
	&\parbox[t]{9cm}{\raggedright \emph{Intervensjon hvor Sjur introduserer og forklarer bildet av lysspekteret på oppgavearket.} %speech 
	}&\parbox[t]{4cm}{\raggedright  %action 
	}\\

	4:14 %time
	&Siri %name
	&\parbox[t]{9cm}{\raggedright mhm ... der er det jo litt blått lys og sånt også. %speech 
	}&\parbox[t]{4cm}{\raggedright Peker på det blå lyset i illustrasjonen øverst på oppgavearket %action 
	}\\

	4:18 %time
	&Nora %name
	&\parbox[t]{9cm}{\raggedright ja så det er ikke bare rent grønt … %speech 
	}&\parbox[t]{4cm}{\raggedright  %action 
	}\\

	4:20 %time
	&Fredrik %name
	&\parbox[t]{9cm}{\raggedright ... ja det er jo ikke bare på 500 circa ((referer til bølgelengde)), det er jo et stort område %speech 
	}&\parbox[t]{4cm}{\raggedright Holder hendene fra hverandre som om han signaliserer hvor langt noe er. %action 
	}\\

	4:26 %time
	&Siri %name
	&\parbox[t]{9cm}{\raggedright mhm, og planten tar jo ihvertfall opp veldig mye blå .. blårlilla lys ... %speech 
	}&\parbox[t]{4cm}{\raggedright  %action 
	}\\

	4:31 %time
	&Fredrik %name
	&\parbox[t]{9cm}{\raggedright ... mhm ... %speech 
	}&\parbox[t]{4cm}{\raggedright  %action 
	}\\

	4:32 %time
	&Siri %name
	&\parbox[t]{9cm}{\raggedright så da har den sikkert kunnet utnytte mye av dette her. %speech 
	}&\parbox[t]{4cm}{\raggedright peker på det blå spekteret i illustrasjonen øverst på oppgavearket %action 
	}\\


	\bottomrule
\end{tabular}
\end{center}
\end{adjustwidth}
\caption{Excerpt from hypothesis generation 1}
\label{excerpt:hypothesis1}
\end{table}

\subsection{Explanation}
Fredriks observation on the appearance of the plants breaks Siris expectation that the plant in the closet would not grow as much as the on in the window. Siri starts to explain why this could have happened by talking about light in different wavelengths, but without explaining why this has any effect on growth, only stating that it has an effect.
Sjur drops in and introduces the illustration of the green light on the paper as it appears as if they have not seen this yet. This gives the students more hold in Siris explanation that it might not only be green light in the closet, as the green lamp produces some light in the blue specter. So they now have two possible explanations to why the plants have grown (in their eyes) the same amount. Firstly, there might have been some light pollution coming into the closet, and secondly the green light is not purely green.

\section{Hypothesis generation}

\subsection{Context}
In table~\ref{excerpt:disconfirmation1} the students have been looking at the movements of the two plants, and have observed that the plant in the window are moving towards the sun, a so called heliotropism. They are now observing that the plant in the closet is just growing straight up without any large movement like the other plant. Suddenly Nora observes that the plant is growing a lot more than the window plant.
\subsection{Raw data}


\def\arraystretch{1.5}
\begin{table}
\begin{adjustwidth}{-4em}{-4em}
\begin{center}
\begin{tabular}{r l p{9cm} p{4cm} } \toprule
	Time &  Who &  Speech  & Action\\ \midrule  

	7:46 %time
	&Nora %name
	&\parbox[t]{9cm}{\raggedright Jeg føler at de vokser veldig mye inni ... skapet eller er det? ... %speech 
	}&\parbox[t]{4cm}{\raggedright  %action 
	}\\

	7:51 %time
	&Siri %name
	&\parbox[t]{9cm}{\raggedright Ja det virka som om de vokste ... %speech 
	}&\parbox[t]{4cm}{\raggedright  %action 
	}\\

	7:53 %time
	&Nora %name
	&\parbox[t]{9cm}{\raggedright ... ser ut som de ble lenger lissom ... %speech 
	}&\parbox[t]{4cm}{\raggedright  %action 
	}\\

	7:53 %time
	&Siri %name
	&\parbox[t]{9cm}{\raggedright ... enda mer der. %speech 
	}&\parbox[t]{4cm}{\raggedright  %action 
	}\\

	7:54 %time
	&Fredrik %name
	&\parbox[t]{9cm}{\raggedright ja %speech 
	}&\parbox[t]{4cm}{\raggedright  %action 
	}\\

	7:56 %time
	&Siri %name
	&\parbox[t]{9cm}{\raggedright ... enn ute, at de ble mye lengre. %speech 
	}&\parbox[t]{4cm}{\raggedright  %action 
	}\\

	7:59 %time
	&Fredrik %name
	&\parbox[t]{9cm}{\raggedright mhm. %speech 
	}&\parbox[t]{4cm}{\raggedright  %action 
	}\\

	8:01 %time
	&Siri %name
	&\parbox[t]{9cm}{\raggedright Kanskje de fokuserer veldig på å vokse oppover når lyset er rett over dem.. at de vokser rett oppover ((fører hånden oppover)) i stedet for å følge lyset og gå lissom sånn sakte oppover ((snurrer hånden sakte oppover)) %speech 
	}&\parbox[t]{4cm}{\raggedright  %action 
	}\\
	
	
	\bottomrule
\end{tabular}
\end{center}
\end{adjustwidth}
\caption{Excerpt from disconfirmation of growth}
\label{excerpt:disconfirmation1}
\end{table}

\subsection{Explanation}
Siri starts at once to generate a hypothesis for why the plant in the closet are growing more than the one in the window. \sout{It might be because it is hard for her to realize that her expectations where wrong and she is trying to cope with her misconception of plants growing in green light.} Basically what she is saying is that heliotropism makes the plant in the window grow slower because it has to move after the sun, and since the plant in the closet can grow straight up, it can grow faster. 

\section{Hypothesis generation}

\subsection{Context}
Morten has asked if the students has looked at the graphs below the video, and the students have in response looked into the graphs and observed that the light graph is really different in the two enviroments. In  table~\ref{excerpt:hypothesis2} Sjur then asks the question they wondered about earlier again to see if the graphs can help them test their hypothesis. 

\subsection{Raw data}

\def\arraystretch{1.5}
\begin{table}
\begin{adjustwidth}{-4em}{-4em}
\begin{center}
\begin{tabular}{r l p{9cm} p{4cm} } \toprule
	Time &  Who &  Speech  & Action\\ \midrule  

	9:21 %time
	&Sjur %name
	&\parbox[t]{9cm}{\raggedright Men hvorfor tror dere den i skapet strekker seg så mye, den som fikk grønt lys ... %speech 
	}&\parbox[t]{4cm}{\raggedright Nora snur seg mot Sjur som står bak gruppen %action 
	}\\

	9:26 %time
	&Nora %name
	&\parbox[t]{9cm}{\raggedright De skal jo bare vokse oppover da, eller den vokser bare oppover så.. %speech 
	}&\parbox[t]{4cm}{\raggedright Siri snur seg også %action 
	}\\

	9:30 %time
	&Sjur %name
	&\parbox[t]{9cm}{\raggedright ja? %speech 
	}&\parbox[t]{4cm}{\raggedright  %action 
	}\\

	9:31 %time
	&Nora %name
	&\parbox[t]{9cm}{\raggedright Da.. har den mye energi til det? %speech 
	}&\parbox[t]{4cm}{\raggedright  %action 
	}\\

	9:33 %time
	&Siri %name
	&\parbox[t]{9cm}{\raggedright Ja kanskje den fokuserer på å vokse rett oppover ((tar hånden oppover)) når lyset står der hele tiden.. åja! også om natta så er det jo ikke sol, så da … %speech 
	}&\parbox[t]{4cm}{\raggedright  %action 
	}\\

	9:43 %time
	&Nora %name
	&\parbox[t]{9cm}{\raggedright Da vokser den jo ikke opp... %speech 
	}&\parbox[t]{4cm}{\raggedright ser usikkert mot sjur etterhvert %action 
	}\\

	9:44 %time
	&Fredrik %name
	&\parbox[t]{9cm}{\raggedright mhm %speech 
	}&\parbox[t]{4cm}{\raggedright  %action 
	}\\

	9:45 %time
	&Siri %name
	&\parbox[t]{9cm}{\raggedright da vokser den ikke etter lyset på en måte %speech 
	}&\parbox[t]{4cm}{\raggedright litt usikker i stemmen %action 
	}\\

	9:47 %time
	&Nora %name
	&\parbox[t]{9cm}{\raggedright Ja altså den vokste jo dag og natt .. i .. skapet %speech 
	}&\parbox[t]{4cm}{\raggedright  %action 
	}\\

	9:50 %time
	&Siri %name
	&\parbox[t]{9cm}{\raggedright mhm, for det var lys der hele tiden ... så den strakk seg hele tiden etter lyset %speech 
	}&\parbox[t]{4cm}{\raggedright  %action 
	}\\

	\bottomrule
\end{tabular}
\end{center}
\end{adjustwidth}
\caption{Excerpt from hypothesis 2}
\label{excerpt:hypothesis2}
\end{table}

\subsection{Explanation}
Here they found that since the plant in the closet got light all night and all day, it got more light than the plant in the window which got light only during the day.

\section{Hypothesis generation}

\subsection{Context}
Earlier Morten explicitly told the group to look at the light graphs again, however, the group only stated the fact that the window plant got a lot of light during the day, but nothing at night, where as the plant in the closet got a constant amount of light.
table~\ref{excerpt:hypothesis2} 

\subsection{Raw data}

\def\arraystretch{1.5}
\begin{table}
\begin{adjustwidth}{-4em}{-4em}
\begin{center}
\begin{tabular}{r l p{9cm} p{4cm} } \toprule
	Time &  Who &  Speech  & Action\\ \midrule  

	10:49 %time
	&Sjur %name
	&\parbox[t]{9cm}{\raggedright mens den andre gjerne .. nesten ligge på null heile veien da .. (?) %speech 
	}&\parbox[t]{4cm}{\raggedright Fredrik og Nora snur seg. Nora nikker %action 
	}\\

	10:53 %time
	&Siri %name
	&\parbox[t]{9cm}{\raggedright Å ja! det var jo lavere lys der, men så blir det veldig mye lys her når det først er lys. %speech 
	}&\parbox[t]{4cm}{\raggedright har et ganske bekymret ansiktsuttryk mens hun prøver å forstå hva hun sier. %action 
	}\\

	11:11 %time
	&Sjur %name
	&\parbox[t]{9cm}{\raggedright Men hvis dere ser på baksiden av det oppgavearket %speech 
	}&\parbox[t]{4cm}{\raggedright Peker mot arket. Nora snur arket %action 
	}\\

	 %time
	& %name
	&\parbox[t]{9cm}{\raggedright  %speech 
	}&\parbox[t]{4cm}{\raggedright  %action 
	}\\

	 %time
	& %name
	&\parbox[t]{9cm}{\raggedright  %speech 
	}&\parbox[t]{4cm}{\raggedright Lærer kommer bort %action 
	}\\

	11:20 %time
	&Lærer %name
	&\parbox[t]{9cm}{\raggedright Går det bra eller %speech 
	}&\parbox[t]{4cm}{\raggedright kommer bort til bordet og lener seg på det. %action 
	}\\

	11:23 %time
	&Siri %name
	&\parbox[t]{9cm}{\raggedright mmm, ja %speech 
	}&\parbox[t]{4cm}{\raggedright  %action 
	}\\

	11:24 %time
	&Lærer %name
	&\parbox[t]{9cm}{\raggedright skjønner dere ... har dere funnet forklaring på alle spørsmålene? %speech 
	}&\parbox[t]{4cm}{\raggedright  %action 
	}\\

	11:26 %time
	&Alle jentene %name
	&\parbox[t]{9cm}{\raggedright *** vi prøver ... %speech 
	}&\parbox[t]{4cm}{\raggedright snakker i munnen på hverandre %action 
	}\\

	11:27 %time
	&Siri %name
	&\parbox[t]{9cm}{\raggedright Jeg tror kanskje jeg har en ide om det med at den her ute ((peker mot vinduet, refererer til planten i vinduet)) ikke vokser like høyt, eller så fort ihvertfall.. fordi atte når det kommer veldig mye sol så blir jo klorofyllmolekylene eksitert, men når alle ... alle klorofyllene blir eksitert i planten, sånn atte det ikke er flere som kan bli eksitert så hjelper det ikke om det er mere lys. %speech 
	}&\parbox[t]{4cm}{\raggedright  %action 
	}\\

	11:55 %time
	&Lærer %name
	&\parbox[t]{9cm}{\raggedright Så det du tenker er rett og slett at den hemmes av for mye lys, at den ikke vokser så mye fordi det er så mye lys? %speech 
	}&\parbox[t]{4cm}{\raggedright  %action 
	}\\

	12:03 %time
	&Siri %name
	&\parbox[t]{9cm}{\raggedright Kanskje ikke hemmes .. det .. hvis det er veldig sterkt lys kan jo pigmentene bli svidd, men  når det er  litt mere lys enn alt det de kan ta opp.. så hjelper det ikke at det er litt mer, for da kan de ikke ta opp det ekstr... %speech 
	}&\parbox[t]{4cm}{\raggedright  %action 
	}\\

	\bottomrule
\end{tabular}
\end{center}
\end{adjustwidth}
\caption{Excerpt from hypothesis 2}
\label{excerpt:hypothesis2}
\end{table}

\subsection{Explanation}
In the excerpt in table~\ref{excerpt:hypothesis2} Siri understands that it is a difference in the light intensity, not just when the plants get light. However, it seems like she interprets it to mean that the plant in the window get too much light, and that light becomes a limiting factor for the plants growth. When she explains her hypothesis to the teacher later in the excerpt, she is using a more scientific language than before, and mentions chlorophyll molecules that gets excited. This might be because right before this she is introduced to the representation of the light dependent reaction of the photosynthesis, \sout{or it might be because this is the first the teacher is listening to the group}.


\section{Hypothesis generation}

\subsection{Context}

table~\ref{excerpt:scaffold1} 

\subsection{Raw data}

\def\arraystretch{1.5}
\begin{table}
\begin{adjustwidth}{-4em}{-4em}
\begin{center}
\begin{tabular}{r l p{9cm} p{4cm} } \toprule
	Time &  Who &  Speech  & Action\\ \midrule  

	13:33 %time
	&Siri %name
	&\parbox[t]{9cm}{\raggedright nei, men ... hvis de bare hadde fått grønt lys i eh den bølgelengden som de tar opp minst av så hadde kanskje planten vokst veldig lite %speech 
	}&\parbox[t]{4cm}{\raggedright  %action 
	}\\

	13:44 %time
	&Lærer %name
	&\parbox[t]{9cm}{\raggedright ja.. så altså dere tenker at .. sammenhengen mellom vekst og fotosyntese den er helt klar ... du kan ikke du tenker at du kan ik et frø kan ikke spire og vokse og bli en plante uten at drives fotosyntese.. tenker dere alle det? %speech 
	}&\parbox[t]{4cm}{\raggedright  %action 
	}\\

	14:00 %time
	&Fredrik %name
	&\parbox[t]{9cm}{\raggedright Det er jo noen planter som ikke har fotosyntese ... og de spirer jo og fordet ikkesant.. det er vel en liten energipakke på en måte i  frøet da? er det ikke det da? %speech 
	}&\parbox[t]{4cm}{\raggedright  %action 
	}\\

	14:14 %time
	&Lærer %name
	&\parbox[t]{9cm}{\raggedright okei, er det? %speech 
	}&\parbox[t]{4cm}{\raggedright  %action 
	}\\

	14:14 %time
	&Nora %name
	&\parbox[t]{9cm}{\raggedright Ja %speech 
	}&\parbox[t]{4cm}{\raggedright nikker annerkjennende %action 
	}\\
	
	\bottomrule
\end{tabular}
\end{center}
\end{adjustwidth}
\caption{Excerpt from teacher talk}
\label{excerpt:scaffold1}
\end{table}

\subsection{Explanation}
In the excerpt in table~\ref{excerpt:scaffold1} 
\section{Hypothesis testing}

\subsection{Context}
\subsection{Raw data}
\subsection{Explanation}

\section{Questions}

\subsection{Context}
\subsection{Raw data}
\subsection{Explanation}