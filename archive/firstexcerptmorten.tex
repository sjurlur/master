%!TEX root = ../document.tex
\chapter{Data \& Analysis}
In this chapter we will present the findings from our case study ...

\section{Expectations for the experiment}

\subsection{Context}
In table~\ref{excerpt:expectations1} the students have read assignment 1a: \emph{What did you expect would happen?} They have talked about the plant in the window and how they expected it to follow the sun, and how it would use water from the soil. In this sequence they are talking about what they thought would happen to the plant which only got green light. While discussing they are also describing what the conditions for the plant was. The plant from the closet is in front of them on the table, but they do not know which plant it is. 
\subsection{Raw data}


\def\arraystretch{1.5}
\begin{table}[H]
\begin{adjustwidth}{-4em}{-4em}
\begin{center}
\begin{tabular}{r l p{9cm} p{4cm} } \toprule
	Time &  Who &  Speech  & Action\\ \midrule  

	2:16 %time
	&Siri %name
	&\parbox[t]{9cm}{\raggedright .. det var det planten stod i skapet også skulle det være bare grønt lys på den ... men det kan jo hende for eksempel at det kom litt annet lys inn i skapet også .. så da er det ikke sikkert at det bare bar grønt lys ..  %speech 
	}&\parbox[t]{4cm}{\raggedright peker på skapet %action 
	}\\

	2:31 %time
	&Nora %name
	&\parbox[t]{9cm}{\raggedright  %speech 
	}&\parbox[t]{4cm}{\raggedright nikker %action 
	}\\

	2:31 %time
	&Siri %name
	&\parbox[t]{9cm}{\raggedright og planten tar jo opp littegrann grønt lys også, men ikke så mye .. så derfor kunne det hende atte den ikke vokste like my.. eller jeg trodde at den ikke ville vokse like mye i skapet .. siden da fikk den bare grønt lys ...  %speech 
	}&\parbox[t]{4cm}{\raggedright  %action 
	}\\

	2:46 %time
	&Nora %name
	&\parbox[t]{9cm}{\raggedright ... mmm ... %speech 
	}&\parbox[t]{4cm}{\raggedright  %action 
	}\\

	2:47 %time
	&Siri %name
	&\parbox[t]{9cm}{\raggedright eller neste bare grønt lys ihvertfall ... men hvor mye vokste den egentlig? er det den ((refererer til planten på bordet)) som stod i skapet? %speech 
	}&\parbox[t]{4cm}{\raggedright peker på planten som står på pulten %action 
	}\\

	2:52 %time
	&Sjur %name
	&\parbox[t]{9cm}{\raggedright ja %speech 
	}&\parbox[t]{4cm}{\raggedright  %action 
	}\\

	2:53 %time
	&Nora %name
	&\parbox[t]{9cm}{\raggedright OJ(!) %speech 
	}&\parbox[t]{4cm}{\raggedright  %action 
	}\\

	2:53 %time
	&Siri %name
	&\parbox[t]{9cm}{\raggedright Den har jo vokst ganske mye %speech 
	}&\parbox[t]{4cm}{\raggedright smiler %action 
	}\\
	
	\bottomrule
\end{tabular}
\end{center}
\end{adjustwidth}
\caption{Excerpt from exercise 1}
\label{excerpt:expectations1}
\end{table}

\subsection{Explanation}
Siri had an expectation that the plant from the closet would not grow as much as the plant from the window, and she is explaining why she expected that, referring to the green light and that the plant cannot take up as much green light as other light wavelengths. At the same time, she knows from looking at the system earlier that the plant have grown, so she tries to explain why it has grown at all by mentioning that there could have been some other light coming into the closet. She wonders how much the plant actually grew, and asked if the plant on the table was the one in the closet. Both she and Nora shows that they are surprised by how much it has grown.


\section{Hypothesis generation - green is not green}

\subsection{Context}
After looking at the plant on the table the student wanted to know if the stems on the plants in the window where white as the stems in the closet. Sjur told them they could check it in the system, so they opened monoplant and navigated to the videolist where they got an overview over the looks of the two different plants. They checked the color of the stems and found that they where white in the window as well, then Siri opened a video from 31th of October and started it. This is when Fredrik starts to talk in table~\ref{excerpt:hypothesis1.1}.



\subsection{Raw data}


\def\arraystretch{1.5}
\begin{table}[H]
\begin{adjustwidth}{-4em}{-4em}
\begin{center}
\begin{tabular}{r l p{9cm} p{4cm} } \toprule
	Time &  Who &  Speech  & Action\\ \midrule  

	3:24 %time
	&Fredrik %name
	&\parbox[t]{9cm}{\raggedright mhm ... mmja så da er det jo egentlig ganske ... ja ikke så stor forskjell da på de som stod ...  i skapet ((peker på planten på border)) og de som stod i vinduskarmen hvis man bare ser på ...  utseende %speech 
	}&\parbox[t]{4cm}{\raggedright Dette sies mens Siri starter videoen, hun stopper også videoen før de har sett den halvferdig. %action 
	}\\

	3:37 %time
	&Siri %name
	&\parbox[t]{9cm}{\raggedright ja .. men da ville jeg kanskje tenke at det kan hende at det kom inn annet lys enn det grønne lyset også. siden de har vokst så bra, og at de vokser bedre hvis de får flere.. lys i flere bølgelengder enn bare grønt lys %speech 
	}&\parbox[t]{4cm}{\raggedright Stemmeleiet går opp mot slutten av setningen, og løfter blikket fra arket for å få bekreftelse %action 
	}\\

	... %time
	&... %name
	&\parbox[t]{9cm}{\raggedright \emph{Intervensjon hvor Sjur introduserer og forklarer bildet av lysspekteret på oppgavearket.} %speech 
	}&\parbox[t]{4cm}{\raggedright  %action 
	}\\

	\bottomrule
\end{tabular}
\end{center}
\end{adjustwidth}
\caption{Excerpt from observation ..}
\label{excerpt:hypothesis1.1}
\end{table}

\def\arraystretch{1.5}
\begin{table}[H]
\begin{adjustwidth}{-4em}{-4em}
\begin{center}
\begin{tabular}{r l p{9cm} p{4cm} } \toprule
	Time &  Who &  Speech  & Action\\ \midrule  

	4:14 %time
	&Siri %name
	&\parbox[t]{9cm}{\raggedright mhm ... der er det jo litt blått lys og sånt også. %speech 
	}&\parbox[t]{4cm}{\raggedright Peker på det blå lyset i illustrasjonen øverst på oppgavearket %action 
	}\\

	4:18 %time
	&Nora %name
	&\parbox[t]{9cm}{\raggedright ja så det er ikke bare rent grønt … %speech 
	}&\parbox[t]{4cm}{\raggedright  %action 
	}\\

	4:20 %time
	&Fredrik %name
	&\parbox[t]{9cm}{\raggedright ... ja det er jo ikke bare på 500 circa ((referer til bølgelengde)), det er jo et stort område %speech 
	}&\parbox[t]{4cm}{\raggedright Holder hendene fra hverandre som om han signaliserer hvor langt noe er. %action 
	}\\

	4:26 %time
	&Siri %name
	&\parbox[t]{9cm}{\raggedright mhm, og planten tar jo ihvertfall opp veldig mye blå .. blårlilla lys ... %speech 
	}&\parbox[t]{4cm}{\raggedright  %action 
	}\\

	4:31 %time
	&Fredrik %name
	&\parbox[t]{9cm}{\raggedright ... mhm ... %speech 
	}&\parbox[t]{4cm}{\raggedright  %action 
	}\\

	4:32 %time
	&Siri %name
	&\parbox[t]{9cm}{\raggedright så da har den sikkert kunnet utnytte mye av dette her. %speech 
	}&\parbox[t]{4cm}{\raggedright peker på det blå spekteret i illustrasjonen øverst på oppgavearket %action 
	}\\


	\bottomrule
\end{tabular}
\end{center}
\end{adjustwidth}
\caption{Excerpt from hypothesis generation 1}
\label{excerpt:hypothesis1.2}
\end{table}

\subsection{Explanation}
Fredriks observation on the appearance of the plants breaks Siris expectation that the plant in the closet would not grow as much as the on in the window. Siri starts to explain why this could have happened by talking about light in different wavelengths, but without explaining why this has any effect on growth, only stating that it has an effect.
Sjur drops in and introduces the illustration of the green light on the paper as it appears as if they have not seen this yet. This gives the students more hold in Siris explanation that it might not only be green light in the closet, as the green lamp produces some light in the blue specter. So they now have two possible explanations to why the plants have grown (in their eyes) the same amount. Firstly, there might have been some light pollution coming into the closet, and secondly the green light is not purely green.

\section{Hypothesis generation based on misconception}

\subsection{Context}
In table~\ref{excerpt:disconfirmation1} the students have been looking at the movements of the two plants, and have observed that the plant in the window are moving towards the sun, a so called heliotropism. They are now observing that the plant in the closet is just growing straight up without any large movement like the other plant. Suddenly Nora observes that the plant is growing a lot faster and higher than the window plant.
\subsection{Raw data}


\def\arraystretch{1.5}
\begin{table}[H]
\begin{adjustwidth}{-4em}{-4em}
\begin{center}
\begin{tabular}{r l p{9cm} p{4cm} } \toprule
	Time &  Who &  Speech  & Action\\ \midrule  

	7:46 %time
	&Nora %name
	&\parbox[t]{9cm}{\raggedright Jeg føler at de vokser veldig mye inni ... skapet eller er det? ... %speech 
	}&\parbox[t]{4cm}{\raggedright  %action 
	}\\

	7:51 %time
	&Siri %name
	&\parbox[t]{9cm}{\raggedright Ja det virka som om de vokste ... %speech 
	}&\parbox[t]{4cm}{\raggedright  %action 
	}\\

	7:53 %time
	&Nora %name
	&\parbox[t]{9cm}{\raggedright ... ser ut som de ble lenger lissom ... %speech 
	}&\parbox[t]{4cm}{\raggedright  %action 
	}\\

	7:53 %time
	&Siri %name
	&\parbox[t]{9cm}{\raggedright ... enda mer der. %speech 
	}&\parbox[t]{4cm}{\raggedright  %action 
	}\\

	7:54 %time
	&Fredrik %name
	&\parbox[t]{9cm}{\raggedright ja %speech 
	}&\parbox[t]{4cm}{\raggedright  %action 
	}\\

	7:56 %time
	&Siri %name
	&\parbox[t]{9cm}{\raggedright ... enn ute, at de ble mye lengre. %speech 
	}&\parbox[t]{4cm}{\raggedright  %action 
	}\\

	7:59 %time
	&Fredrik %name
	&\parbox[t]{9cm}{\raggedright mhm. %speech 
	}&\parbox[t]{4cm}{\raggedright  %action 
	}\\

	8:01 %time
	&Siri %name
	&\parbox[t]{9cm}{\raggedright Kanskje de fokuserer veldig på å vokse oppover når lyset er rett over dem.. at de vokser rett oppover ((fører hånden oppover)) i stedet for å følge lyset og gå lissom sånn sakte oppover ((snurrer hånden sakte oppover)) %speech 
	}&\parbox[t]{4cm}{\raggedright  %action 
	}\\
	
	
	\bottomrule
\end{tabular}
\end{center}
\end{adjustwidth}
\caption{Excerpt from disconfirmation of growth}
\label{excerpt:disconfirmation1}
\end{table}

\subsection{Explanation}
Siri starts at once to generate a hypothesis for why the plant in the closet are growing more than the one in the window. It might look like it is hard for her to realize that her expectations where wrong and she is trying to cope with her misconception. She is mixing germination and photosynthesis, which is not so weird, because ultimately, she might not have learned anything about germination. Basically what she is saying is that heliotropism makes the plant in the window grow slower because it has to move after the sun, and since the plant in the closet can grow straight up, it can grow faster. 

\section{Hypothesis generation}

\subsection{Context}
Morten has asked if the students has looked at the graphs below the video, so they checked the graphs and observed that the light graph is really different in the two environments. Sjur then asks why the plant which got green light grew so much, a question they wondered about earlier. 

\subsection{Raw data}

\def\arraystretch{1.5}
\begin{table}[H]
\begin{adjustwidth}{-4em}{-4em}
\begin{center}
\begin{tabular}{r l p{9cm} p{4cm} } \toprule
	Time &  Who &  Speech  & Action\\ \midrule  

	9:21 %time
	&Sjur %name
	&\parbox[t]{9cm}{\raggedright Men hvorfor tror dere den i skapet strekker seg så mye, den som fikk grønt lys ... %speech 
	}&\parbox[t]{4cm}{\raggedright Nora snur seg mot Sjur som står bak gruppen %action 
	}\\

	9:26 %time
	&Nora %name
	&\parbox[t]{9cm}{\raggedright De skal jo bare vokse oppover da, eller den vokser bare oppover så.. %speech 
	}&\parbox[t]{4cm}{\raggedright Siri snur seg også %action 
	}\\

	9:30 %time
	&Sjur %name
	&\parbox[t]{9cm}{\raggedright ja? %speech 
	}&\parbox[t]{4cm}{\raggedright  %action 
	}\\

	9:31 %time
	&Nora %name
	&\parbox[t]{9cm}{\raggedright Da.. har den mye energi til det? %speech 
	}&\parbox[t]{4cm}{\raggedright  %action 
	}\\

	9:33 %time
	&Siri %name
	&\parbox[t]{9cm}{\raggedright Ja kanskje den fokuserer på å vokse rett oppover ((tar hånden oppover)) når lyset står der hele tiden.. åja! også om natta så er det jo ikke sol, så da … %speech 
	}&\parbox[t]{4cm}{\raggedright  %action 
	}\\

	9:43 %time
	&Nora %name
	&\parbox[t]{9cm}{\raggedright Da vokser den jo ikke opp... %speech 
	}&\parbox[t]{4cm}{\raggedright ser usikkert mot sjur etterhvert %action 
	}\\

	9:44 %time
	&Fredrik %name
	&\parbox[t]{9cm}{\raggedright mhm %speech 
	}&\parbox[t]{4cm}{\raggedright  %action 
	}\\

	9:45 %time
	&Siri %name
	&\parbox[t]{9cm}{\raggedright da vokser den ikke etter lyset på en måte %speech 
	}&\parbox[t]{4cm}{\raggedright litt usikker i stemmen %action 
	}\\

	9:47 %time
	&Nora %name
	&\parbox[t]{9cm}{\raggedright Ja altså den vokste jo dag og natt .. i .. skapet %speech 
	}&\parbox[t]{4cm}{\raggedright  %action 
	}\\

	9:50 %time
	&Siri %name
	&\parbox[t]{9cm}{\raggedright mhm, for det var lys der hele tiden ... så den strakk seg hele tiden etter lyset %speech 
	}&\parbox[t]{4cm}{\raggedright  %action 
	}\\

	\bottomrule
\end{tabular}
\end{center}
\end{adjustwidth}
\caption{Excerpt from hypothesis 2}
\label{excerpt:hypothesis2}
\end{table}

\subsection{Explanation}
Here they generated a hypothesis that since the plant in the closet got light all night and all day, hence it got more light than the plant in the window which got light only during the day. And they think that this might be the reason why the plant in the closet grew more.

\section{Hypothesis generation gone wrong}

\subsection{Context}
Earlier Morten explicitly told the group to look at the light graphs again, however, the group only stated the fact that the plant in the closet got a constant amount of light, where as the window plant got a lot of light during the day, but nothing at night. So Sjur threw in a comment that the plant in the closet has a low level of light all the time. This is where the excerpt starts in table~\ref{excerpt:hypothesis3.1}, the other excerpt (see table~\ref{excerpt:hypothesis3.2}) is from right after this when the teacher arrives to speak with the group.

\subsection{Raw data}

\def\arraystretch{1.5}
\begin{table}[H]
\begin{adjustwidth}{-4em}{-4em}
\begin{center}
\begin{tabular}{r l p{9cm} p{4cm} } \toprule
	Time &  Who &  Speech  & Action\\ \midrule  

	10:49 %time
	&Sjur %name
	&\parbox[t]{9cm}{\raggedright mens den andre gjerne .. nesten ligge på null heile veien da .. (?) %speech 
	}&\parbox[t]{4cm}{\raggedright Fredrik og Nora snur seg. Nora nikker %action 
	}\\

	10:53 %time
	&Siri %name
	&\parbox[t]{9cm}{\raggedright Å ja! det var jo lavere lys der ((refererer til skapplanten)), men så blir det veldig mye lys her ((refererer til vindusplanten)) når det først er lys. %speech 
	}&\parbox[t]{4cm}{\raggedright har et ganske bekymret ansiktsuttryk mens hun prøver å forstå hva hun sier. %action 
	}\\

	11:11 %time
	&Sjur %name
	&\parbox[t]{9cm}{\raggedright Men hvis dere ser på baksiden av det oppgavearket %speech 
	}&\parbox[t]{4cm}{\raggedright Peker mot arket. Nora snur arket %action 
	}\\

	\bottomrule
\end{tabular}
\end{center}
\end{adjustwidth}
\caption{Excerpt from some important stuff}
\label{excerpt:hypothesis3.1}
\end{table}

\def\arraystretch{1.5}
\begin{table}[H]
\begin{adjustwidth}{-4em}{-4em}
\begin{center}
\begin{tabular}{r l p{9cm} p{4cm} } \toprule
	Time &  Who &  Speech  & Action\\ \midrule  

	 %time
	& %name
	&\parbox[t]{9cm}{\raggedright  %speech 
	}&\parbox[t]{4cm}{\raggedright Lærer kommer bort %action 
	}\\

	11:20 %time
	&Lærer %name
	&\parbox[t]{9cm}{\raggedright Går det bra eller %speech 
	}&\parbox[t]{4cm}{\raggedright kommer bort til bordet og lener seg på det. %action 
	}\\

	11:23 %time
	&Siri %name
	&\parbox[t]{9cm}{\raggedright mmm, ja %speech 
	}&\parbox[t]{4cm}{\raggedright  %action 
	}\\

	11:24 %time
	&Lærer %name
	&\parbox[t]{9cm}{\raggedright skjønner dere ... har dere funnet forklaring på alle spørsmålene? %speech 
	}&\parbox[t]{4cm}{\raggedright  %action 
	}\\

	11:26 %time
	&Alle jentene %name
	&\parbox[t]{9cm}{\raggedright *** vi prøver ... %speech 
	}&\parbox[t]{4cm}{\raggedright snakker i munnen på hverandre %action 
	}\\

	11:27 %time
	&Siri %name
	&\parbox[t]{9cm}{\raggedright Jeg tror kanskje jeg har en ide om det med at den her ute ((peker mot vinduet, refererer til planten i vinduet)) ikke vokser like høyt, eller så fort ihvertfall.. fordi atte når det kommer veldig mye sol så blir jo klorofyllmolekylene eksitert, men når alle ... alle klorofyllene blir eksitert i planten, sånn atte det ikke er flere som kan bli eksitert så hjelper det ikke om det er mere lys. %speech 
	}&\parbox[t]{4cm}{\raggedright  %action 
	}\\

	11:55 %time
	&Lærer %name
	&\parbox[t]{9cm}{\raggedright Så det du tenker er rett og slett at den hemmes av for mye lys, at den ikke vokser så mye fordi det er så mye lys? %speech 
	}&\parbox[t]{4cm}{\raggedright  %action 
	}\\

	12:03 %time
	&Siri %name
	&\parbox[t]{9cm}{\raggedright Kanskje ikke hemmes .. det .. hvis det er veldig sterkt lys kan jo pigmentene bli svidd, men  når det er  litt mere lys enn alt det de kan ta opp.. så hjelper det ikke at det er litt mer, for da kan de ikke ta opp det ekstr... %speech 
	}&\parbox[t]{4cm}{\raggedright  %action 
	}\\

	\bottomrule
\end{tabular}
\end{center}
\end{adjustwidth}
\caption{Excerpt from teacher talk}
\label{excerpt:hypothesis3.2}
\end{table}

\subsection{Explanation}
In the excerpt in table~\ref{excerpt:hypothesis3.1} Siri understands that it is a difference in the light intensity, not just when the plants get light. However, when she explains it to the teacher in table~\ref{excerpt:hypothesis3.2}, it seems like she interprets it to mean that the plant in the window get too much light, and that light becomes a limiting factor for the plants growth. When she explains her hypothesis to the teacher, she is using a more scientific language than before, and mentions chlorophyll molecules that gets excited. This might be because Sjur introduced to the representation of the light dependent reaction of the photosynthesis just before the teacher arrived, \sout{or it might be because this is the first time the teacher is listening to the group and she wants to impress him with sciencebable}.


\section{Scaffolding to fix misconception}

\subsection{Context}
In table~\ref{excerpt:scaffold1} the teacher has been talking with the group for a couple of minutes and Siri has talked about her hyphothesis that the plant in the closet have got more than just green light, and if it only got green light it would probably not grow that much. In this excerpt Fredrik introduces that the seed has an energy pack, which later turns into a scaffolded discussion in the group where Nora and Fredrik finds out that seeds have starch as a food reserve, which makes germination possible.

\subsection{Raw data}

\def\arraystretch{1.5}
\begin{table}[H]
\begin{adjustwidth}{-4em}{-4em}
\begin{center}
\begin{tabular}{r l p{9cm} p{4cm} } \toprule
	Time &  Who &  Speech  & Action\\ \midrule  

	13:44 %time
	&Lærer %name
	&\parbox[t]{9cm}{\raggedright ja.. så altså dere tenker at .. sammenhengen mellom vekst og fotosyntese den er helt klar ... du kan ikke du tenker at du kan ik et frø kan ikke spire og vokse og bli en plante uten at drives fotosyntese.. tenker dere alle det? %speech 
	}&\parbox[t]{4cm}{\raggedright  %action 
	}\\

	14:00 %time
	&Fredrik %name
	&\parbox[t]{9cm}{\raggedright Det er jo noen planter som ikke har fotosyntese ... og de spirer jo og fordet ikkesant.. det er vel en liten energipakke på en måte i  frøet da? er det ikke det da? %speech 
	}&\parbox[t]{4cm}{\raggedright  %action 
	}\\

	14:14 %time
	&Lærer %name
	&\parbox[t]{9cm}{\raggedright okei, er det? %speech 
	}&\parbox[t]{4cm}{\raggedright  %action 
	}\\

	14:14 %time
	&Nora %name
	&\parbox[t]{9cm}{\raggedright Ja %speech 
	}&\parbox[t]{4cm}{\raggedright nikker annerkjennende %action 
	}\\
	
	\bottomrule
\end{tabular}
\end{center}
\end{adjustwidth}
\caption{Excerpt from teacher talk}
\label{excerpt:scaffold1}
\end{table}

\subsection{Explanation}
In the excerpt in table~\ref{excerpt:scaffold1} the teacher asks a question which force the group to think outside of the model of photosynthesis and more to the germination process which is common to all plants regardless of if they've got photosynthesis.

\section{Hypothesis generation}

\subsection{Context}
Teacher has left, Morten asked the students to look at the plant-videos an see if there is any difference in their appearance. see table~\ref{excerpt:observation1} 

\subsection{Raw data}

\def\arraystretch{1.5}
\begin{table}[H]
\begin{adjustwidth}{-4em}{-4em}
\begin{center}
\begin{tabular}{r l p{9cm} p{4cm} } \toprule
	Time &  Who &  Speech  & Action\\ \midrule  

	17:12 %time
	&Siri %name
	&\parbox[t]{9cm}{\raggedright Der åpner jo bladene seg med en gang nesten %speech 
	}&\parbox[t]{4cm}{\raggedright  %action 
	}\\

	17:15 %time
	&Fredrik %name
	&\parbox[t]{9cm}{\raggedright ja ... ((stillhet, venter til video er ferdig)) det kan jo ha noe med at her trenger den jo bladene for fange lyset da, mens den trenger jo ikke det så mye inni skapet.. eh kanskje %speech 
	}&\parbox[t]{4cm}{\raggedright Planten trenger ikke bladene i skapet fordi det ikke er så mye lys? %action 
	}\\

	17:34 %time
	&Siri %name
	&\parbox[t]{9cm}{\raggedright at den bruker næringen fra jorda og frøet mer i skapet? %speech 
	}&\parbox[t]{4cm}{\raggedright  %action 
	}\\

	17:37 %time
	&Fredrik %name
	&\parbox[t]{9cm}{\raggedright ehhhh.. ja. eller at den ikke utnytter den sol.. det sollyset inne i skapet så det den trenger jo ikke da også at bladene spretter ut så tidlig eller at... eh ja. %speech 
	}&\parbox[t]{4cm}{\raggedright Fredrik er ikke helt enig med Siri. Mener at planten i skapet ikke har noe lys å utnytte, derfor ingen blader %action 
	}\\

	17:50 %time
	&Nora %name
	&\parbox[t]{9cm}{\raggedright Ja fordi er det ikke stilken til en plante da består jo mest av sånn stivelse eller cellulose, og det har den jo i frøet sitt, eller.. den lager jo det av fotosyntese %speech 
	}&\parbox[t]{4cm}{\raggedright blir mer usikker mot slutten av setningen og snur seg mot Siri for å få bekrefte. %action 
	}\\

	18:01 %time
	&Siri %name
	&\parbox[t]{9cm}{\raggedright ja ... ja hvis den har det i frøet at .. da virker det som om den bruker mest næringen den får fra jorda og frøet i skapet og at den ikke fokuserer så mye på fotosyntese før ... etterhvert %speech 
	}&\parbox[t]{4cm}{\raggedright  %action 
	}\\


	\bottomrule
\end{tabular}
\end{center}
\end{adjustwidth}
\caption{Excerpt from new observation}
\label{excerpt:observation1}
\end{table}

\subsection{Explanation}
In the excerpt in table~\ref{excerpt:observation1} 


\section{Hypothesis generation}

\subsection{Context}
The students are working with assignment 3a, and have discovered that the soil in the closet dries up much faster than the one in the window. They are puzzled by this because they think that the plant in the window should use more water since it is focusing more on the photosynthesis. Siri and Nora have generated a hypothesis that the plant in the closet might compensate for the lack of light by using more water to grow. After they have said this, Sjur introduces the tool tip function so that they can explore how the plant looks at certain points in the graph.  see table~\ref{excerpt:observation1} 

\subsection{Raw data}

\def\arraystretch{1.5}
\begin{table}[H]
\begin{adjustwidth}{-4em}{-4em}
\begin{center}
\begin{tabular}{r l p{9cm} p{4cm} } \toprule
	Time &  Who &  Speech  & Action\\ \midrule  

	 %time
	& %name
	&\parbox[t]{5cm}{\raggedright  %speech 
	}&\parbox[t]{4cm}{\raggedright Sjur viser tooltip funksjonalitet i systemet %action 
	}\\


	23:47 %time
	&Siri %name
	&\parbox[t]{5cm}{\raggedright ...vi kan se ... hehe, skulle bare se forskjellen på de to. Det kan hende atte, ja her ((referer til 4. nov 09:00 – 5. nov 24:00)) vokste den veldig mye, når den brukte det vannet der. hmm, kanskje den trenger veldig mye vann for å vokse da, den som stod inni skapet.  %speech 
	}&\parbox[t]{4cm}{\raggedright Drar musepekeren langs grafen fra start til slutt for å se på de små bildene som dukker opp på de ulike punktene i grafen. Tar så å fokuserer på området fra 4-6 november. Beveger musepeker frem og tilbake mellom punktet før den ble vannet og etter den ble vannet %action 
	}\\

	24:27 %time
	&Fredrik %name
	&\parbox[t]{5cm}{\raggedright ja, siden den inne i skapet vokste jo mye høyere, eh. %speech 
	}&\parbox[t]{4cm}{\raggedright Peker mot planten på pulten %action 
	}\\

	24:30 %time
	&Nora %name
	&\parbox[t]{5cm}{\raggedright ja, hvis hastigheten er større så må den jo ha mer vann .. %speech 
	}&\parbox[t]{4cm}{\raggedright  %action 
	}\\

	24:35 %time
	&Siri %name
	&\parbox[t]{5cm}{\raggedright ja, og da med en gang den får vann så tar den opp det vannet med en gang og vokser veldig raskt, også blir kanskje, blir det kanskje ganske tørt etter ikke så kort, ikke så lang tid.  %speech 
	}&\parbox[t]{4cm}{\raggedright Drar musepekeren fra en graftopp der planten blir vannet til det har blitt tørrere i jorden. %action 
	}\\

	24:48 %time
	&Fredrik %name
	&\parbox[t]{5cm}{\raggedright mhm. %speech 
	}&\parbox[t]{4cm}{\raggedright  %action 
	}\\


	\bottomrule
\end{tabular}
\end{center}
\end{adjustwidth}
\caption{Excerpt from new observation}
\label{excerpt:hypothesis3}
\end{table}

\subsection{Explanation}
In the excerpt in table~\ref{excerpt:observation1} 

\section{Hypothesis testing}

\subsection{Context}
\subsection{Raw data}
\subsection{Explanation}

\section{Questions}

\subsection{Context}
\subsection{Raw data}
\subsection{Explanation}