%!TEX root = ../document.tex
\chapter{Concluding remarks}
In this thesis we have presented our plant monitoring system, Monoplant, and described how high-school students interacted with the system while working with questions related to photosynthesis and germination in a biology class. The main focus has been how Monoplant as an inquiry based learning tool provided a context for exploration of photosynthesis by the students. 

The Monoplant system consists of three different parts. A physical system of a plant with different sensors, a cloud storage solution to make data available everywhere via Internet, and a user interface in the form of a web page. Our main goal was to visualize different aspects of the life of a plant, not otherwise available for observation and in-depth scrutiny. This was solved using sensors to record environmental changes over time, and timelapse videos to visualize the effect of these changes. 

In order to test our system and answer our research questions, we gathered data in a biology class at a high-school in Oslo. In collaboration with the teacher we designed an experiment where the students changed the plant's light conditions, while keeping water and temperature levels relatively controlled. Our primary data is a one-hour video of four students during a class session where they answered questions regarding the experiment. The data was transcribed and analysed using interaction analysis techniques as described by \citet{jordan1995interaction}. This implied a thorough investigation of spoken utterances, nonverbal interaction and interaction with Monoplant. In addition we gathered written answers from the rest of the students, and made use of observation notes during our analysis. 

The thesis is framed within a sociocultural perspective, leading us to focus on how the students interacted with each other and Monoplant, and how the context of the school affected the students' inquiry process. The analytical perspective has been \emph{dialogic}, meaning that we have looked at how Monoplant provided a context for social interaction. 

The first research question was as follows: \emph{"What characterizes the students’ inquiry in interaction with Monoplant?"}. In order to answer this we looked at other research literature regarding inquiry learning and compared their results with ours. We found that similar to \citeauthor*{klahr1993heuristics} \citetext{\citeyear{klahr1993heuristics}, referenced in \citealp{de1998scientific}} the students had problems with interpreting the data collected in the experiment. This lead to the student confirming an erroneous hypothesis based on faulty interpretation of data. We also found evidence of problems with \emph{hypothesis generation} \citep{de1998scientific}, as the students kept their current hypothesis, despite conflicting evidence, because they could not think of alternatives. 

We also applied the concept of "misconception" in inquiry learning as described by \citet{gomez2008elementary} and \citet{smith1994misconceptions}, and found that the students had preconceptions about photosynthesis gained from their working with the textbook curriculum some weeks prior to our data collection. This in turn led the students to form misconceptions as they tried to apply their prior knowledge of photosynthesis to the experiment. Knowledge that explained the phenomenon in the context of the textbook lead to an erroneous conclusion in the inquiry setting provided by Monoplant. I.e. it could be seen as "faulty extensions of productive prior knowledge" \citep{smith1994misconceptions}. 

We will argue that the students need to be guided through the inquiry process, as inquiry learning places an extra load on the learners. The challenge then becomes to keep the inquiry process open and give the students elbow room to experiment and make "productive" mistakes, but at the same time lead them toward fruitful discoveries. 

%While discussing this we showed examples where the students' preconceptions about photosynthesis affected their reasoning and evaluation of evidence in Monoplant. The main issue being that the plants in the experiment were presented in a real-world context, as opposed to how the textbook model presents plants, making it difficult for them to explain what they saw in Monoplant based on their prior understanding. This in time led to misconceptions as the students pursued ideas and explanations based on an inconsistent understanding.
%maybe rephrase.


The second question we sought to answer was: \emph{"How does Monoplant, by presenting photosynthesis differently from the text book, support the inquiry process?"}. Here we applied \citeauthor{vygotsky2012thought}'s (\citeyear{vygotsky2012thought}) notion of spontaneous and scientific concepts to provide an explanation of why the students' language changed when working with the different types of representations. Vygotski{\u\i}'s theory states that spontaneous concepts work their way from the concrete to the abstract, while scientific concepts work their way from the abstract to the concrete \citep{vygotsky2012thought}. We found that the textbook's and teacher's representations of photosynthesis provided the students with comprehension of the "scientific" explanation of photosynthesis, while Monoplant provided them with real life experiences linked to the concept. This in turn lead some of the students to gain greater concept comprehension. 

On the other hand, we found that the students had problems linking between the different representations (Monoplant and the textbook). This is similar to the results described in \citet{ainsworth1999functions} and \citet{van2006supporting}. The students are faced with extra tasks related to multiple representations which may leave less resources for actual learning. 

We believe this data warrants the assumption that multiple representations should be used to provide students with experiences and knowledge related to different parts of the phenomenon under study. To avoid the costs related to multiple representations, the students should be guided through the task of linking between representations, which leads us to our next research question.

%Especially the link between the textbook model of photosynthesis and the Monoplant model. So while multiple representations may display different aspects of the phenomenon under study, the students are faced with an extra task. The process of linking should therefore be scaffolded so we can have the benefits of multiple representations while reducing the cost. 

%Here we saw evidence of one of the students discoveries leading to a larger conceptual understanding. As Monoplant provided real life experiences related to the phenomenon of photosynthesis and the textbook provided scientific concepts, they complemented each other, which in turn lead to stronger concept comprehension. 



%misconception

Our third research question was: \emph{"How is scaffolding operationalized in the environment?"}.
Here we looked at teacher's and researchers' interventions with \citeauthor{wood1976role}'s (\citeyear{wood1976role}) original study and six steps of scaffolding in mind. We also saw different opportunities for employing a scaffold, one of them that lead to a faulty interpretation of the representation by one of the students. By this we also proved that in line with the zone of proximal development, a scaffold is personal and should therefore be personally adjusted. This is a task that can be done on the fly by a good teacher, but is hard to achieve with a computer based scaffold. 

We also discussed how the inquiry process lead to misconceptions among the students. The problem then becomes to scaffold in a way that the students are lead toward the fruitful discoveries, but at the same time have the freedom to explore and not feel commanded in any way. Computer based scaffolds can be used for this task, as they can be programmed to instruct the students with the procedural aspects of inquiry learning. 

The fourth and last question was: \emph{"How does the institutional setting frame the students' inquiry process?"}. This question was approached with the notion that the inquiry process took place in a context where two different practices of doing science and doing school intersected. We found that the two practices were affecting the students in different ways. When the students were occupied with doing science, interpreting data, exploring Monoplant and discussing evidence, they used an everyday language based on data they saw. While in contact with the teacher the students' language became more scientific oriented, and the concerns of the students went from investigating the assignments to showing their insight in the domain. In line with our observations regarding linking representations, this tentatively suggests that the students master both practices, but have problems combining them.
%social practices

\section{Limitations and directions for further work}
Apart from the themes discussed, there are some limitations in the system and our research, which we will elaborate in the following section. 

\subsection{System design}
Monoplant was designed as a generic system to represent the life of plants. While there was a general idea that it could be used in a school setting, we did not have a specific user group or school curriculum in mind. This meant that peculiarities of the curriculum for \emph{Biology 2} was not integrated into the system, only into the experiments and the attached assignments. This could've been taken into account during the design and could have proven to influence the design. However, in our research we saw evidence of situations where the system could have scaffolded the students' inquiry process, providing scaffolding not necessarily linked to any curriculum.

When the data-collection of plant data is done, Monoplant is only a provider of the data, assignments and a collaborative environment need to be provided alongside Monoplant. Both could have been integrated into the web-interface with possibilities for communication between students and instructors. We think however that such an environment would prove to give a different research focus because of the distinctive nature of computer mediated communication compared to real world communication.
%The students could have been included in the design process

\subsection{Research design}
The research conducted in this thesis were directed on high performing students in one class at one school, limiting the study to a small selection of the actual user group. While this was due to time limitations, we acknowledge that a larger and broader study could and should be done as it would be interesting to see results from similar experiments conducted with students at other performing levels.

As mentioned, the inquiry learning session took place when the students were done with lectures on the textbook chapter concerning photosynthesis. As inquiry learning is normally used in order for students to gain motivation for learning, this process was introduced quite late according to traditional inductive learning methods. While our study proved to be informing, we think that it would be interesting to compare a study where Monoplant was introduced before the students learned the peculiarities of the curriculum.

We have earlier in this thesis stated that we have been doing design based research. At the same time we have presented a case study where we have introduced Monoplant in an educational setting. The contradiction is of course that DBR as an action research-like methodology is supposed to last over several iterations, and not by any means end up in a case study. As always we blame it on the boogie, also known as time limitations, and as such; acknowledge this study as the first iteration of a design experiment and thereby that there is still plenty of work to be pursued.

\subsection{Further work}

\begin{itemize}
\item{Implement a process oriented scaffold within the Monoplant system.}
\item{Redesign the Monoplant interface with the science curriculum in mind.}
\item{Continue iterations in design and research}
\item{Write up this thesis}
\item{Drink coffee}
\item{play ping pong}
\item{Start working at our respective workplaces}
\end{itemize}