%!TEX root = ../document.tex
\chapter{Concluding remarks}
In this thesis we have presented our plant monitoring system Monoplant and described how high-school students interacted with the system while working with questions related to photosynthesis and germination. The main focus has been how Monoplant as an inquiry based learning tool provided a context for exploration of photosynthesis by the students. 

The Monoplant system consists of three different parts. A physical system consisting of a plant and different sensors, a cloud storage solution to make data available everywhere, and a user interface in the form of a web page. Our main goal when we created the system was to visualize different aspects of the life of plants, not otherwise available for observation. This was solved using sensors to record environmental changes over time, and timelapse videos to visualize the effect of these changes. 

In order to test out our system and answer our research questions, we gathered data in a biology class at a high-school in Oslo. In collaboration with the teacher we designed an experiment where the students changed the plant's light conditions, while keeping water and temperature levels relatively controlled. Our primary data consisted of video of four students during a class session where they answered questions regarding the experiment. The data was transcribed and analyzed using interaction analysis techniques as described by \citet{jordan1995interaction}. This implied thorough investigation of speech, nonverbal interaction and interaction with Monoplant, in our corpus of data. 

The thesis is framed within a sociocultural perspective, leading to a focus on how the students interacted with each other and Monoplant, and how the context of the school affected the students inquiry process. The analytical perspective has been \emph{dialogic}, meaning that we have looked at how Monoplant has provided a context for social interaction. 

The first research question was as follows: \emph{"What characterizes the students’ inquiry in interaction with Monoplant?"} While discussing this we showed examples where the students' preconceptions about photosynthesis affected their reasoning and evaluation of evidence in Monoplant. The main issue being that the plants in the experiment were presented in a real-world context, as opposed to how the textbook model presents plants, making it difficult for them to explain what they saw in Monoplant based on their prior understanding. This in time led to misconceptions as the students pursued ideas and explanations based on an inconsistent understanding.
%maybe rephrase.


The second question we sought to answer was: \emph{"How does Monoplant, by presenting photosynthesis differently from the text book, support the inquiry process?"}. Here we saw evidence of one of the students discoveries leading to a larger conceptual understanding. As Monoplant provided real life experiences related to the phenomenon of photosynthesis and the textbook provided scientific concepts, they complemented each other, which in turn lead to stronger concept comprehension. 

We also found that the students had problems understanding the link between the different representations. Especially the link between the textbook model of photosynthesis and the Monoplant model. So while multiple representations may display different aspects of the phenomenon under study, the students are faced with an extra task. The process of linking should therefore be scaffolded so we can have the benefits of multiple representations while reducing the cost. 

%misconception

Our third research question was: \emph{"How is scaffolding operationalized in the environment?"}. Here we looked at how \citeauthor{wood1976role}'s (\citeyear{wood1976role}) six steps of scaffolding was used by the instructors during the session. We also saw how there were multiple opportunities for employing a scaffold, and that scaffolds should be personally adjusted. 

The main finding in this section was how the inquiry process lead to misconceptions among the students. The problem then becomes to scaffold in a way that the students are lead toward the fruitful discoveries, but at the same time have the freedom to explore and not feel commanded in any way. Computer based scaffolds can be used for this task, as they can be programmed to instruct the students with the procedural aspects of inquiry learning. 

The fourth and last question was: \emph{"How does the institutional setting frame the students' inquiry process?"}. This question was approached with the notion that the inquiry process took place in a context where two different practices of doing science and doing school intersected. We found that the two practices were affecting the students in different ways. At times when the students were occupied with doing science, interpreting data, exploring Monoplant and discussing evidence, they students used an everyday language based on data they saw. While in contact with the teacher the students' language became more scientific, and the concerns of the students went from investigating the assignments to showing their insight in the theme. In resemblance with our observations regarding linking representations, this suggests that the students master both the practices, but have problems combining them.
%social practices

\section{Limitations and directions for further work}
Apart from the themes discussed, there are some limitations in the system and our research which we will elaborate in the following section. 

\subsection{Generic system}
Monoplant is a generic system which monitors environmental variables in a pot and presents the observed data to the user. The system was not designed for the specific user group of high school students, rather for students of any age. This meant that the curriculum for Biology 2 was not taken into account during the design of the system. 

%The session and experiments needs to be carefully designed.

\subsection{Inverted inductive learning}
As mentioned, the inquiry learning session took place when the students were done with the chapter on photosynthesis. 

\subsection{High end students}
The experiments were conducted with high performance students, how does other students use Monoplant?
