%!TEX root = ../document.tex
\chapter{Empirical settings and methods}
In this chapter...

\section{Empirical setting}
The collection of data material used in this thesis took part in the late autumn, 2013, at a high school located in the centre of Oslo. The school is in the upper third on the grade scale, with a limit of 43.5 points for admission in 2010/2011 \citep{utdanningsetaten}. Therefore the students at this school are mostly high achievers. 

Contact with the school was first initiated through Intermedia, and a presentational flyer was sent as an explanation of the project (_____appendix_____). Luckily our request coincided perfectly with a 2 week time frame for reviewing photosynthesis in one of the teacher's classes, and he was therefore quite eager to use our application in class. 

The class selected for the experiment was a biology class at the highest level offered at the school, biologi 2, which has an extensive curriculum covering e.g photosynthesis, enzymes and energy transmitters. The students were between 17 and 18 years of age, and for the main part most of the 20 students taking the class were present. All the students aggred to participate in the study, but due to technical limitations, data collection was only done with a small sample of the group. 

\subsection{Planning}
The planning of the project was done by us in conjuction with the teacher responsible for the for the biology class. An initial plannign meeting was held at the school around one week before the experiments started. There we gave the teacher a thorough introduction of the system, and presented some ideas for expirements the students could perform using our system as platform. This involved:

\begin{enumerate}
\item{Present the application in class}
\item{Initiate an experiment using the application. Related to e.g soil moisture, light intensity, light quality, or temperature}
\item{Have a one hour session where the students work with text tasks realted to the experiment}
\end{enumerate}

The teacher then suggested that we could conduct two experiments, so the students could work on the relations between the different external factors effect on photosynthesis. As the system records a range of different variables, it would be possible to keep the environment relatively controlled, or at least point to factors which could be sources of error in the experiment. 

We agreed that the factor where we could get the most interesting result was to vary the light intesity and the light quality (wavelength). The first experiment would then involve keeping the plant located in a window facing west, receiving sunlight and light from the fluorescent indoor-lighting. In the second experiment we would plant new seeds, and relocate the plant to a (presumably) light proof cupboard. The plant would then only be given green light, with a known wavelength. Each of the experiments would have a duration of approximately one week. 

\subsection{Execution pow pow ?}
The project was presented and the first experiment initiated by one of the students on friday 25th of october. This went on until friday 1st of november when the second experiment was initiated. The second experiment went on until wednesday 13th of november when the primary data collection session started. During the experiments we were present at four separate occations, observing what the teacher was focusing on, and what kind of questions/which part of the photosynthesis the students found most difficult. In addition to answering questions about the system, and if/how the system was used in the education. 

\subsection{Data collection}


Our main data material consists of x hours of video from the session... blabla

\section{Analytical Procedures}

\subsection{Ethics}



